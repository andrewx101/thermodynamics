\documentclass[main.tex]{subfiles}
\begin{document}
本讲义旨在帮助高分子专业的同学联系大二学的物理化学知识和大三学的高分子物理知识。现有内容是笔者在日常工作中的零星积累,所以主题是分散的。

本讲义假定大二物理化学指定课本是《物理化学(第六版)》(傅献彩,侯文华),大三高分子物理指定课本是《高分子物理(第三版)》(何曼君,张红东,陈维孝,董西侠)。在讲义当中,凡不加说明地提到“《物理化学》”或“《高分子物理》”就默认指这两本书。

作为中文物理化学课本的补充,可参考:
\begin{itemize}
\item Callen, Herbert B. (1985), \emph{Thermodynamics and an Introduction to Thermostatistics}, 2nd ed., John Wiley \& Sons
\item 王竹溪(1960), \emph{热力学(第二版)}, 高等教育出版社
\end{itemize}
作为中文高分子物理课本的补充,可参考:
\begin{itemize}
\item Flory, P. J. (1953), \emph{Principles of Polymer Chemistry}, Cornell University Press
\item van Dijk M. A., Wakker, A. (1997), \emph{Concepts of Polymer Thermodynamics}, ChemTec Publishing
    \item Gnanou, Y., Fontanille, M. (2008), \emph{Organic and Physical Chemistry of Polymers}, John Wiley \& Sons
    \item Hiemenzm, P., Lodge, T. (2007), \emph{Polymer Chemstry, 2nd ed.}, CRC Press
\end{itemize}



\begin{flushright}
孙尉翔\\
2020年10月
\end{flushright}
\end{document}

