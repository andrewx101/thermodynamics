% Generated by GrindEQ Word-to-LaTeX 
\documentclass{article} % use \documentstyle for old LaTeX compilers

\usepackage[utf8]{inputenc} % 'cp1252'-Western, 'cp1251'-Cyrillic, etc.
\usepackage[english]{babel} % 'french', 'german', 'spanish', 'danish', etc.
\usepackage{amsmath}
\usepackage{amssymb}
\usepackage{txfonts}
\usepackage{mathdots}
\usepackage[classicReIm]{kpfonts}
\usepackage{graphicx}

% You can include more LaTeX packages here 


\begin{document}

%\selectlanguage{english} % remove comment delimiter ('%') and select language if required


\noindent \textbf{$ $等温三元相图的坐标}


{\bf  三组分混合物的组分坐标常画在二维平面上的三角形区域。这种坐标体系时重心坐标(Barycentric coordinates system)的一种,三元相图中,三个组分的质量、摩尔或体积分数之和为1。因此,关于组成的函数$f\left(x_{1} x_{2} ,x_{{}_{3} } \right)$,其实是二元函数$Fx_{1} ,x_{2} )\equiv f(x_{1} ,x_{2} ,1-x_{1} -x_{2} )$}

\begin{enumerate}
\item  设等边三角形在直角坐标系中如下图放置,某组成表示为三角形区域内一点P,它代表的组成为$\left(x_{1} ,x_{2} ,x_{3} \right)$,在直角坐标系中的坐标是$\left(X,Y\right)$。根据几何关系得 
\end{enumerate}

\noindent $\begin{array}{l} {\left\{\begin{array}{c} {X=\frac{1}{2} (1+x_{1} -x_{2} )} \\ {Y=\frac{\sqrt{3} }{2} (1-x_{1} -x_{2} )} \end{array}\right. } \\ {0\le x_{1} \le 1,0\le x_{2} \le 1} \end{array}$   $\begin{array}{l} {\left\{\begin{array}{c} {x_{1} =X-\frac{\sqrt{3} }{3} Y} \\ {x_{2} =1-X-\frac{\sqrt{3} }{3} Y,} \end{array}\right. } \\ {\frac{\sqrt{3} }{3} Y\le X\le 1-\frac{\sqrt{3} }{3} Y,0\le Y\le \frac{\sqrt{3} }{2} } \end{array}$    \includegraphics*[width=2.03in, height=1.47in]{image1} $ $

\begin{enumerate}
\item  在三元坐标系中的两点$P\left(x_{sp} ,x_{2p} ,x_{sp} \right),Q\left(x_{1\underline{q}} ,x_{2\underline{q}} ,x_{3\underline{q}} \right)$所确定的直线在直角坐标系下的方程
\end{enumerate}

\noindent 由$\begin{array}{l} {\begin{array}{cc} {X_{p} =\frac{1}{2} \left(1+x_{1p} -x_{2p} \right),} & {X_{q} =\frac{1}{2} \left(1+x_{1q} -x_{2q} \right)} \end{array}} \\ {\begin{array}{cc} {Y_{p} =\frac{\sqrt{3} }{2} \left(1-x_{1p} -x_{2p} \right),} & {Y_{q} =\frac{\sqrt{3} }{2} \left(1-x_{1q} -x_{2q} \right)} \end{array}} \end{array}$

\noindent ?两点式?:$Y=\frac{Y_{p} -Y_{q} }{X_{p} -X_{q} } \left(X-X_{q} \right)+Y_{q} $

\noindent 代入上面的关系式:
\[\begin{array}{c} {\frac{\sqrt{3} }{2} \left(1-x_{1} -x_{2} \right)=\sqrt{3} \frac{x_{1q} -x_{1p} +x_{2q} -x_{2p} }{x_{1p} -x_{1q} +x_{2q} -x_{2p} } \cdot \frac{1}{2} \left(x_{1} -x_{1q} +x_{2q} -x_{2} \right)+\frac{\sqrt{3} }{2} \left(1-x_{1q} -x_{2q} \right)} \\ {\Rightarrow x_{2} =\frac{x_{2p} -x_{2q} }{x_{1p} -x_{1q} } x_{1} +\frac{x_{1p} x_{2q} -x_{1q} x_{2p} }{x_{1p} -x_{1q} } } \end{array}\] 


{\bf  直线上有意义的点不能超出等边三角形区域。}

\noindent \textbf{高分子溶液}

\begin{enumerate}
\item \textbf{ }考虑格子体积恒定的网格,$v_{0} \equiv V/N$,$N$是格子总数,一条高分子链会占据多个格子。考虑双组分,并将分子的?链节?定为一个网格大小的部分。各组分均为由这种链节组成的线形链状分子。

\item  假设已有$k$个高分子占据了格子,高分子的的聚合度$M_{2} $,则总共已占据 $kM_{2} $个格子,还剩$N-kM_{2} $个空格子。我们考虑放置第$k+1$个高分子。这个分子的第一个链节可在 $N-kM_{2} $个空格子中自由选取一个。但第二个链节就只能选头一个空格子周围的$z$个格子,且这$z$个格子中有比例为$f_{k} $的被之前的分子链节所占据。于是第二个链节只能在$z\left(1-f_{k} \right)$个空格子中选一个。之后的链节每个都有$z\left(1-f_{k} \right)$个空格子,忽略因?绕回来?而遇到被之前链节占据格子的情况,这也684532771等假于684532771684532771129647410等价于?假设链不是自回避的不允许的仅仅是原方向逆反。
\end{enumerate}

\noindent 总共可供选择放置这一个高分子的格子数都乘起来就是整条分子链的可能构象数:$V_{k+1} =\left(N-kM_{2} \right)z\left(z-1\right)^{M_{2} -2} \left(1-f_{k} \right)^{M_{2} -1} $

\noindent 如果这$N_{i} $条链在放入格子的时候都是互不干扰的,即每条都如上述情况,则它们的总构象数是

\noindent $\mathop{\prod }\limits_{k=1}^{N_{2} } V_{k} $   其中$V_{k} $中含有的$f_{k} $是变化的684532597?684532597684532597129647407问号的使用可能有误$N_{2} $是高分子总数

\noindent 在这个总数中,我们把放置过$N_{2} $条链的先后不同也当作不同放法了,实际我们认为不同次序是等价的,所以要从上面的总数中$N_{2} !$扣掉,故放法数变成
\[\frac{1}{N_{2} !} \mathop{\prod }\limits_{k=1}^{N_{i} } \nu _{k} \] 

\begin{enumerate}
\item  $f_{k} $的估计。当一个格子已经放好上一个链节,现在当前链节在面临$z$个相邻格子,它们还会有多大比例$f_{k} $被占据呢?主要因为被之前$k$条链的链节占了,假设这个概率$\overline{f_{k} }$等于总比例$\overline{f_{i} }=\left(N-M_{i} k\right)/N$。这是Flory的简单假设。实际上还要考虑,此时的悬着受已知信息(上一个链节占据了一格)所影响,以下文献进行了更仔细的考虑,但没有得出与实验更相符的预测。
\end{enumerate}

\noindent \includegraphics*[width=1.46in, height=0.63in]{image2}

\noindent Huggins:J.Phys.Chem.46:151(1942),Ann.N.Y.Acad.Sci.41:1(1942).JACS64:1712(1942)

\noindent Miller :Poc.Cambridge Phil.Soc,39:54(1943)

Miller(1948)The Theory of Solution of High Polymers, Clarendon Press

\noindent Orr: Trans\_Faraday Soc,40:320(1944)

\noindent Guggenheim : Proc.R.Soc.A 183:203(1944)

\noindent Flory在书P.501-502的footnote有一个考虑过程,当$z\to \infty $时 $f_{k} \to \overline{f_{k} }$,所以用$\overline{f_{k} }$当$f_{k} $是更符合实际的。

\noindent 把$\left(1-f_{k} \right)=\left(N-M_{2} k\right)/N$
\[V_{k+1} =(N-M_{2} k)^{M_{2} } [\frac{\left(z-1\right)}{N} ]^{M_{2} -1} \] 
Flory又作了一个近似$V_{k+1} =\frac{\left(N-M_{2} k\right)!}{\left[N-M_{2} \left(k+1\right)\right]!} \left[\frac{\left(Z-1\right)}{N} \right]^{M_{2} -1} $

\noindent 故高分子的总构象是:
\[\Omega =\frac{N!}{\left(N-M_{i} k\right)!N_{2} !} \left(\frac{Z-1}{N} \right)^{N_{2} \left(M_{2} -1\right)} \] 
剩下的溶剂分子若每个放一个格子,那全放好就只有一种放法。故$\Omega $就是高分子溶液的总构象数。

\noindent 溶液终态的熵:
\[S_{c} =-k_{B} \left(N_{1} {\rm ln}\frac{N_{2} }{N_{1} +M_{2} N_{2} } +N_{2} {\rm ln}\frac{N_{2} }{N_{1} +M_{2} N_{2} } -N_{2} \left(M_{2} -1\right){\rm ln}\frac{Z-1}{e} \right)\] 
其中代入了$N\equiv N_{2} +M_{2} N_{2} $ 

\noindent 溶解过程的初态是纯溶剂和完美结晶结构的纯聚合物。后者是伸直链平行排列的构象。溶解过程可视为两步:1)聚合物先打乱,2)加入溶剂分子。上式的$S_{c} $中令$N_{1} =0$就是第一步的熵变
\[{\rm \Delta }S_{{\rm disorientation}} =k_{B} N_{2} \{ \ln M_{2} +(M_{2} -1)\ln \frac{z-1}{e} \} \] 
当$M_{2} $很大时,第一项比第二项小得多,$M_{2} -1\approx M_{2} $ 
\[{\rm \Delta }S_{disorientation} \approx k_{B} N_{2} M_{2} \ln \frac{z-1}{e} \] 
由于本讨论假定所设置的网格是总能恰好被所有溶剂和高分子链节放满的,因此链节数$M_{2} $就是高分子的分子体积与溶剂分子的分子体积之比,$M_2\equiv V_p/V_s$,$V_{s} $也就是网格体积。上式的 ${\rm \Delta }S_{disorientation} $含$M_{2} $,于是该熵既依赖高分子也依赖溶剂分子,不完全是纯高分子的混乱熵变。${\rm \Delta }S_{disorientation} $的更实际考虑应是连续空间中的自由连结链,链段数是$M_{2} {}^{{'} } $,此时难以确定所谓?配位数$z$?。只能说上述格子讨论下的${\rm \Delta }S_{disorientation} $可能小于实际(更自由的构象空间,更大的$z$等)。但本讨论仍保持格子模型的讨论。

\noindent 我们要讨论的混合熵变,不包括${\rm \Delta }S_{disorientation} $,因此\includegraphics*[width=3.22in, height=1.29in]{image3}
\[\begin{array}{c} {\boxed{\Delta S_{mix} } =S_{c} -\Delta S_{disorientation} } \\ {\boxed{=-k_{B} (N_{1} \ln \phi _{1} +N_{2} \ln \phi _{2} )} } \end{array}\] 
其中$\phi _{1} =\frac{N_{1} }{N_{1} +M_{2} N_{2} } $,$\phi _{2} =\frac{N_{2} M_{2} }{N_{1} +M_{2} N_{2} } $,考虑到$M_{2} $ 的摩尔体积比意义,$\phi _{1} $、$\phi _{2} $就是体积分数。

\begin{enumerate}
\item  比较该式与小分子正则溶液的分子式,我们发现原本是摩尔分数的地方换成了体积分数。这就是处理分子尺寸不同的混合熵的一般后果。

\item  推广到多组分:
\[{\rm \Delta }S_{m} =-k_{B} {\mathop{\sum }\limits_{k}} N_{k} \ln \phi _{k} \] 
\end{enumerate}
J.Chem.Phys.12:425(1944),13:172(1945)

\begin{enumerate}
\item  混合焓 在Flory的著作中称为?heat of mixing?。按照正则溶液的讨论,不依赖构象,故形式不变。但是在讨论中用的格子占据比例,现应为体积比。故
\end{enumerate}

\noindent ${\rm \Delta }H_{{\rm mix}} =z\vartriangle \varepsilon N_{1} \phi _{2} $(若溶剂只占一格)         Flory的书P.549

\noindent ${\rm \Delta }H_{{\rm mix}} =z\vartriangle \varepsilon M_{1} N_{1} \phi _{2} $(若溶剂占$M_{1} $格)

\noindent 令$\chi _{12} =z{\rm \Delta }\varepsilon M_{1} /(k_{B} T)$ ,则
\[\boxed{{\rm \Delta }H_{{\rm mix}} =k_{B} T\chi _{12} N_{2} \phi _{2} } \] 
对于多组分,${\rm \Delta }H_{mix} =k_{B} T{\mathop{\sum }\limits_{i}} \sum _{j>i}\chi _{ij} N_{i} \phi _{j}  $ 
\[\begin{array}{rcl} {\boxed{{\rm \Delta }G_{mix} } } & {=} & {{\rm \Delta }H_{mix} -T{\rm \Delta }S_{mix} } \\ {} & {=} & {k_{B} T\chi _{12} N_{1} \phi _{2} +k_{B} T\left(N_{1} \ln \phi _{1} +N_{2} \ln \phi _{2} \right)} \\ {} & {=} & {\boxed{k_{B} T\left(N_{1} \ln \phi _{1} +N_{2} \ln \phi _{2} +\chi _{12} N_{1} \phi _{2} \right)} } \end{array}\] 
重新用我习惯的notation表示,对于不同分子大小的两分子混合:

\noindent $\frac{{\rm \Delta }Gmix}{RT} =n_{1} \ln \phi _{1} +n_{2} \ln \phi _{2} +\chi _{12} n_{1} \phi _{2} $         该式的严格推到尚待补充

\noindent  记组分1,2的摩尔体积是$v_{1} $,$v_{2} $,则$\phi _{i} =\frac{n_{i} v_{i} }{n_{1} v_{1} +n_{2} v_{2} } $。凑成全用$\phi _{i} $表示:
\[\begin{array}{rcl} {\frac{\vartriangle G_{mix} }{RT} } & {=} & {\frac{n_{1} v_{1} }{V} \frac{V}{v_{1} } \ln \phi _{1} +\quad \frac{n_{2} v_{2} }{V} \quad \frac{V}{v_{2} } \ln \phi _{2} +\chi _{12} \frac{n_{1} v_{1} }{V} \frac{V}{v_{1} } \phi _{2} } \\ {} & {=} & {V\left(\frac{\phi _{1} }{v_{1} } \ln \phi _{1} +\frac{\phi _{2} }{v_{2} } \ln \phi _{2} +\chi _{12} \frac{\phi _{1} }{v_{1} } \phi _{2} \right)} \\ {} & {=} & {\frac{V}{v_{1} } \left(\phi _{1} \ln \phi _{1} +\frac{v_{1} }{v_{2} } \phi _{2} \ln \phi _{2} +\chi _{12} \phi _{1} \phi _{2} \right)} \end{array}\] 
记$x=\frac{v_{2} }{v_{1} } $ 则
\[\frac{{\rm \Delta }G_{mix} }{RT} =\frac{V}{v_{1} } \left(\phi _{1} \ln \phi _{1} +\frac{1}{x} \cdot \phi _{2} \ln \phi _{2} +\chi _{12} \phi _{1} \phi _{2} \right)\] 
对于溶剂与大分子的混合,$x$很大。常说$x$正比于分子量,但它更严格而言是正比于$\left[\eta \right]$或$C^{*} {}^{-1} $。

\noindent \eject 

\noindent \textbf{$ $高斯链的正则系综}


{\bf  自由结合链的末端距$\vec{R}$在正则系综平衡态下的概率密度:
\[{\rm \Phi }(\vec{R})\simeq (\frac{3}{2\pi nb^{2} } )^{\frac{3}{2} } \exp (-\frac{3||\vec{R}||^{2} }{2nb^{2} } )\quad (n\to \infty \to Ñ)\] }


{\bf  考虑键势能为$\mathrm{{\mathcal U}}(\vec{r})=\frac{3{\rm k}_{{\rm B}} {\rm T}}{2b^{2} } \parallel \vec{r}\parallel ^{2} $ 的链。一条n+1个原子的链的势能}

\noindent $V(\{ \overrightarrow{r_{i} }\} )=\mathop{\sum }\limits_{i=1}^{n} \frac{3k_{B} T}{2b^{2} } ||\overrightarrow{r_{i} }||^{2} =\mathrm{{\mathcal H}}$   over damped?


{\bf  再分配函数
\[\begin{array}{rcl} {Z} & {=} & {\smallint d\{ \vec{r}_{i} \} \exp \left(-\frac{1}{k_{B} T} \mathop{\sum }\limits_{i=1}^{n} \frac{3k_{B} T}{2b^{2} } \left\| \vec{r}_{i} \right\| ^{2} \right)} \\ {} & {=} & {\smallint d\{ \vec{r}_{i} \} \exp \left(-\mathop{\sum }\limits_{i=1}^{n} \frac{3\left\| \vec{r}_{i} \right\| ^{2} }{2b^{2} } \right)} \\ {} & {=} & {\left(\smallint d\vec{r}\exp \left(-\frac{3\left\| \vec{r}_{i} \right\| ^{2} }{2b^{2} } \right)\right)^{n} } \\ {} & {=} & {(\frac{2\pi b^{2} }{3} )^{\frac{3n}{2} } } \end{array}.\] }


{\bf  高分子取构象$\{ \vec{r}_{i} \} $的概率密度(正则系综)
\[\begin{array}{rcl} {\Psi (\{ \vec{r}_{i} \} )} & {=} & {Z^{-1} \exp (-\mathop{\sum }\limits_{i=1}^{n} \frac{3\parallel \vec{r}_{i} \parallel ^{2} }{2b^{2} } )} \\ {} & {=} & {(\frac{2\pi b^{2} }{3} )^{-\frac{3n}{2} } \exp (-\mathop{\sum }\limits_{i=1}^{n} \frac{3\parallel \vec{r}_{i} \parallel ^{2} }{2b^{2} } )} \\ {} & {=} & {\mathop{\prod }\limits_{i=1}^{n} \left(\frac{2\pi b^{2} }{3} \right)^{-\frac{3}{2} } \exp (-\mathop{\sum }\limits_{i=1}^{n} \frac{3\parallel \vec{r}_{i} \parallel ^{2} }{2b^{2} } )} \\ {} & {=} & {\mathop{\prod }\limits_{i=1}^{n} {\rm \Psi }(\vec{r}_{i} ),\quad \psi (\vec{r})=(\frac{2\pi b^{2} }{3} )^{-\frac{3}{2} } \exp \left(-\frac{3\parallel \vec{r}\parallel ^{2} }{2b^{2} } \right)} \end{array}\] }
可见,我们所考虑的链,正则系统下的键概率$\psi (\vec{r})$ 是高斯分布。故称该键是高斯链。


{\bf  高斯链的末端均概率密度(正则系统)
\[\begin{array}{rcl} {\Phi \left(\vec{R}\right)} & {=} & {{\mathop{\smallint }\limits_{\{ \vec{r}_{i} \} }} \delta \left(\vec{R}-\sum _{i=1}^{n}\vec{r}_{i}  \right){\rm \Psi }\left(\{ \vec{r}_{i} \} \right)d\{ \vec{r}_{i} \} } \\ {} & {=} & {\frac{1}{(2\pi )^{3} } \smallint d\vec{k}d\{ \vec{r}_{j} \} \exp [i\cdot \vec{k}\cdot (\vec{R}-{\rm \Sigma }_{j=1}^{n} \vec{r}_{j} )]\mathop{\prod }\limits_{n}^{n=1} (\frac{2\pi b^{2} }{3} )^{-\frac{3}{2} } \exp (-\frac{3||\vec{r}_{j} ||^{2} }{2b^{2} } )} \\ {} & {=} & {\frac{1}{(2\pi )^{3} } (\frac{3}{2\pi b^{2} } )^{\frac{3n}{2} } \smallint d\vec{k}\exp (i\vec{k}\cdot \vec{R})\smallint d\{ \vec{r}_{j} \} \mathop{\prod }\limits_{j=1}^{n} \exp (-i\vec{k}\cdot \vec{r}_{j} -\frac{3||\vec{r}_{j} ||^{2} }{2b^{2} } )} \\ {} & {=} & {\frac{1}{(2\pi )^{3} } \left(\frac{3}{2\pi b^{2} } \right)^{\frac{3h}{2} } \smallint d\vec{k}\exp \left(i\vec{k}\cdot \vec{R}\right)\left[\smallint d\vec{r}\exp \left(-i\vec{k}\cdot \vec{r}-\frac{3||\vec{r}||^{2} }{2b^{2} } \right)\right]^{n} } \end{array}\] }
其中
\[\begin{array}{rcl} {\smallint d\vec{r}\exp (-i\vec{k}\cdot \vec{r}-\frac{3||\vec{r}^{2} ||}{2b^{2} } )} & {=} & {\smallint dr_{1} dr_{2} dr_{3} \exp (-i{\rm \Sigma }_{j} k_{j} r_{j} -\frac{3{\rm \Sigma }_{j} r_{j} {}^{2} }{2b^{2} } )} \\ {} & {=} & {\mathop{\prod }\limits_{j=1}^{3} \smallint _{-\infty }^{\infty } dr_{j} \exp (-ik_{j} r_{j} -\frac{3r_{j}^{2} }{2b^{2} } )} \\ {} & {=} & {\mathop{\prod }\limits_{j=1}^{3} \frac{2\pi b^{2} }{3} e^{-\frac{1}{6} b^{2} k_{j} {}^{2} } } \\ {} & {=} & {(\frac{2\pi b^{2} }{3} )^{\frac{3}{2} } e^{-\frac{1}{6} b^{2} k^{2} } } \end{array}\] 
\[\begin{array}{l} {\mathrm{\therefore \Phi }(\vec{R})=\frac{1}{(2\pi )^{3} } (\frac{3}{2\pi b^{2} } )^{\frac{3n}{2} } (\frac{2\pi b^{2} }{3} )^{\frac{3n}{2} } \smallint d\vec{k}\exp (i\vec{k}\cdot \vec{R})\exp (-\frac{1}{6} nb^{2} k^{2} )} \\ {=\frac{1}{(2\pi )^{3} } \mathop{\prod }\limits_{j=1}^{3} \smallint \exp (ik_{j} R_{j} -\frac{1}{6} nb^{2} k_{j}^{2} )dk_{j} } \\ {=\frac{1}{(2\pi )^{3} } (\frac{6\pi }{nb^{2} } )^{\frac{3}{2} } \exp \left(-\frac{3\parallel \vec{R}\parallel ^{2} }{2nb^{2} } \right)} \\ {=(\frac{3}{2\pi nb^{2} } )^{\frac{3}{2} } \exp \left(-\frac{3\parallel \vec{R}\parallel ^{2} }{2nb^{2} } \right)} \end{array}\] 
可见任意有限n个链节的高斯链的末端距概率分布是高斯分布(无需$k_{B} \ll 1$近似)。
\[\begin{array}{rcl} {\left\langle \vec{R}^{2} \right\rangle } & {=} & {\smallint \vec{R}^{2} {\rm \Phi }(\vec{R})d\vec{R}} \\ {} & {=} & {\left(\frac{3}{2\pi nb^{2} } \right)^{\frac{3}{2} } \smallint d\vec{R}\cdot \vec{R}^{2} \exp (-\frac{3\parallel \vec{R}\parallel ^{2} }{2nb^{2} } )} \\ {} & {=} & {\left(\frac{3}{2\pi nb^{2} } \right)^{\frac{3}{2} } 4\pi \smallint _{0}^{\infty } \rho ^{4} \exp (-\frac{3\rho ^{2} }{2nb^{2} } )d\rho } \\ {} & {=} & {\left(\frac{3}{2\pi nb^{2} } \right)^{\frac{3}{2} } 4\pi \frac{1}{3} \sqrt{\frac{n}{6} } \left(nb^{2} \right)^{\frac{5}{2} } } \\ {} & {=} & {nb^{2} } \end{array}\] 
\textbf{共聚交联网络}


{\bf  对于共聚交联网络,考虑编号1:溶剂1;2:溶剂2;3:共聚单体A;4:共聚单体B。}

\noindent $v_{i} ,n_{i} ,\phi _{i} $分别是组分i得分摩尔体积,摩尔数和体积分数,对于$i=3,4$,我们考虑的是重复单元。

\noindent 
{\bf 设共聚组成为$F_{3} \equiv \frac{n_{3} }{n_{3} +n_{4} } =\frac{n_{3} }{n_{p} } $ ,$n_{p} \equiv n_{3} +n_{4} $是聚合物总单元数,则$n_{3} =n_{p} F_{3} $ ,$n_{4} =n_{p} (1-F_{3} )$,设体积分数$\phi _{3} =\frac{n_{3} v_{3} }{V} ,\phi _{4} =\frac{n_{4} v_{4} }{V} ,\phi _{p} =\frac{n_{p} v_{p} }{V} $ ,故
\[v_{p} =\frac{n_{3} v_{3} +n_{4} v_{4} }{n_{3} +n_{4} } =F_{3} v_{3} +(1-F_{3} )v_{4}    \phi _{p} =\phi _{3} +\phi _{4} \] 
\[\phi _{3} =\frac{n_{3} v_{3} }{V} =\frac{n_{p} F_{3} v_{3} }{V} =\frac{n_{p} v_{p} }{V} \frac{F_{3} v_{3} }{v_{p} } =\phi _{p} F_{3} \frac{v_{3} }{v_{p} } ,\phi _{4} =\phi _{p} (1-F_{3} )\frac{v_{4} }{v_{p} } \] }

{\bf 混合自由能:            高分子网络没有平动熵
\[\begin{array}{rcl} {\frac{{\rm \Delta }G_{mix} }{RT} } & {=} & {n_{1} \ln \phi _{1} +n_{2} \ln \phi _{2} +n_{3} \ln \phi _{3} +n_{4} \ln \phi _{4} +x_{12} n_{1} \phi _{2} +x_{13} n_{1} \phi _{3} +x_{14} n_{1} \phi _{4} } \\ {} & {} & {+x{}_{23} n{}_{2} \phi {}_{3} +x{}_{24} n{}_{2} \phi {}_{4} +x{}_{34} n{}_{3} \phi {}_{4} } \\ {} & {=} & {n_{1} \ln \phi _{1} +n_{2} \ln \phi _{2} +x_{12} n_{1} \phi _{2} +n_{1} (x_{13} \phi _{3} +x_{14} \phi _{4} )+n_{2} (x_{23} \phi _{3} +x_{24} \phi _{4} )} \\ {} & {} & {+x_{34} n{}_{p} F{}_{3} \phi {}_{p} (1-F{}_{3} )v{}_{4} /v{}_{p} } \\ {} & {=} & {n_{1} \ln \phi _{1} +n_{2} \ln \phi _{2} +x_{12} n_{1} \phi _{2} +x_{1p} n_{1} \phi _{p} +x_{2p} n_{2} \phi _{p} +x_{34} n_{p} \phi _{p} F_{3} (1+_{3} )v_{4} /v_{p} } \end{array}\] }

{\bf 其中$x_{1p} =(x_{13} \phi _{3} +x_{14} \phi _{4} )/(\phi _{3} +\phi _{4} )$ ,$x_{2p} =(x_{23} \phi _{3} +x_{24} \phi _{4} )/(\phi _{3} +\phi _{4} )$是由上式定义的。与将聚合物当作684693203聚均物6846932036846932031203393112均聚物?相比,多出一项$x_{34} n_{p} \phi _{p} F_{3} (1-F_{3} )v_{4} /v_{p} $ 。}

\noindent 
{\bf 上式中的相互作用参数}

\noindent 
{\bf $x_{ij} =\frac{\nu _{i} }{RT} \left(\delta _{i} -\delta _{j} \right)^{2} $ ,其中$\delta _{i} $是组分$i$的溶度参数,与内聚能密度$E_{c,i} $关系$\delta _{i} =(\frac{E_{c,i} }{v_{i} } )^{\frac{1}{2} } $ ,$E_{c,i} $用基团贡献法预测。}

\noindent \eject 

\noindent \textbf{混合物的化学势}


{\bf  $T,p,\mu _{i} $$T,p,\mu _{i} $多组分体系内能,在孤立时,是$(S,V,\{ n_{i} \} )$的函数$U=U(S,V,\{ n_{i} \} )$
\[dU=\left. \frac{\partial U}{\partial S} \right|_{V,\{ n_{i} \} } dS+\left. \frac{\partial U}{\partial V} \right|_{S,\{ n_{i} \} } dV+\sum i \left. \frac{\partial U}{\partial n_{i} } \right|_{S,V,\{ n_{j\ne i} \} } dn_{i} \] }

{\bf 与第一、二定律比较得(准静态过程)$dU=TdS-pdV+\mu _{i} dn_{i} $
\[T=\left. \frac{\partial U}{\partial S} \right|_{V,\{ n_{i} \} } ,-p=\left. \frac{\partial U}{\partial V} \right|_{S,\{ n_{i} \} } ,\mu _{i} =\left. \frac{\partial U}{\partial n_{i} } \right|_{S,V,\{ n_{j\ne i} \} } \] }

{\bf $U_{i} $是由于组分$i$分子数变化$dn_{i} $造成的内能变化,引入为化学势。}


{\bf  $(T,p,\{ n_{i} \} )$$(T,p,\{ n_{i} \} )$常考虑体系平衡态由$(T,p,\{ n_{i} \} )$ 决定的情况,此时热力学势是吉布斯自由能,
\[\begin{array}{rcl} {G{\mathop{=}\limits^{{\rm det}}} U-TS+pV{\rm ,d}G} & {=} & {{\rm d}U-T{\rm d}S-S{\rm d}T+p{\rm d}V+Vdp} \\ {} & {=} & {TdS-pdV+\mu _{i} dn_{i} -TdS-SdT+pdV+Vdp} \\ {} & {=} & {-SdT+Vdp+\mu _{i} dn_{i} } \\ {} & {=} & {\left. \frac{\partial G}{\partial T} \right|_{p,\{ n_{i} \} } dT+\left. \frac{\partial G}{\partial P} \right|_{T,\{ n_{i} \} } dp+\left. \frac{\partial G}{\partial ni} \right|_{T,p,\{ n_{j\ne i\} } } dn_{i} } \end{array}\] 
\[\quad \Rightarrow \quad -S=\left. \frac{\partial G}{\partial T} \right|_{p,\{ n_{i} \} } ,\quad V=\left. \frac{\partial G}{\partial p} \right|_{T,\{ n_{i} \} } ,\quad \mu _{i} =\left. \frac{\partial G}{\partial n_{i} } \right|_{T,p,\{ n_{j\ne i} \} } \] }

{\bf Maxwell关系:
\[\left. \left. \frac{\partial S}{\partial p} \right|_{T,\{ n_{i} \} } =\left. -\frac{\partial }{\partial p} {\rm (}\left. \frac{\partial G}{\partial T} \right|_{p,\{ n_{i} \} } \right)\right|_{T,\{ n_{i} \} } =\left. \left. -\frac{\partial }{\partial T} {\rm (}\left. \frac{\partial G}{\partial p} \right|_{\pi ,\{ n_{i} \} } \right)\right|_{p,\{ n_{i} \} } =-\left. \frac{\partial V}{\partial T} \right|_{p,\{ n_{i} \} } \] 
\[\left. \frac{\partial S}{\partial n_{i} } \right|_{T,p,\left\{n_{j\ne i} \right\}} =-\frac{\partial }{\partial n_{i} } \left. \left(\left. \frac{\partial G}{\partial T} \right|_{p,\left\{n_{i} \right\}} \right)\right|_{T,p,f\left\{n_{j\ne i} \right\}} =-\frac{\partial }{\partial T} \left. \left(\left. \frac{\partial G}{\partial n_{i} } \right|_{T,p,\left\{n_{j\ne i} \right\}} \right)\right|_{p,\left\{n_{i} \right\}} =-\left. \frac{\partial \mu i}{\partial T} \right|_{p,} {}_{\left\{n_{i} \right\}} \] 
\[\left. \left. \frac{\partial V}{\partial n_{i} } \right|_{T,p,\{ n_{j\ne i} \} } =\frac{\partial }{\partial n_{i} } \left(\left. \frac{\partial G}{\partial p} \right|_{T,\{ n_{i} \} } \right)\right|_{T,p,\{ n_{j\ne i} \} } =\frac{\partial }{\partial p} \left. \left(\left. \frac{\partial G}{\partial n_{i} } \right|_{T,p,\{ n_{j\ne i} \} } \right)\right|_{T,\{ n_{i} \} } =\left. \frac{\partial \mu _{i} }{\partial p} \right|_{T,\{ n_{i} \} } \] 
\[\left. \left. \left. \frac{\partial \mu _{i} }{\partial n_{j} } \right|_{T,p,\{ n_{k\ne j} \} } =\frac{\partial }{\partial n_{j} } \left(\left. \frac{\partial G}{\partial n_{i} } \right|_{T,p,\left\{n_{l\ne i} \right\}} \right)\right|_{T,p,\{ n_{k\ne j} \} } =\frac{\partial }{\partial n_{i} } \left(\left. \frac{\partial G_{i} }{\partial n_{j} } \right|_{T,p,\{ n_{k\ne j} \} } \right)\right|_{{}_{T,p,\left\{n_{{}_{l\ne i} } \right\}} } =\left. \frac{\partial \mu _{j} }{\partial n_{i} } \right|_{T,p,\{ n_{k\ne i} \} } \] }

{\bf 其中定义$V_{i} {\mathop{=}\limits^{{\rm det}}} \left. \frac{\partial V}{\partial n_{i} } \right|_{T,p} {}_{,\{ n_{j\ne i} \} } $ 为组分$i$在$(T,p,\{ n_{i} \} )$ 下的偏摩尔体积。}

\noindent 
{\bf         $S_{i} {\mathop{=}\limits^{{\rm det}}} \left. \frac{\partial S}{\partial n_{i} } \right|_{T,p} {}_{,\{ n_{j\ne i} \} } $ 为组分$i$在$(T,p,\{ n_{i} \} )$ 下的偏摩尔熵。}

\noindent 
{\bf 则$\mu _{i} =\mu _{i} (T,p,\{ n_{i} \} )$ 第$i$个组分的化学势全微分
\[\begin{array}{rcl} {d\mu {}_{i} } & {=} & {\left. \frac{\partial \mu _{i} }{\partial T} \right|_{p,\{ n_{i} \} } dT+\left. \frac{\partial \mu _{i} }{\partial P} \right|_{T,\{ n_{i} \} } dp+{\mathop{\sum }\limits_{j}} \left. \frac{\partial \mu _{i} }{\partial n_{j} } \right|_{T,p,\{ n_{k\ne j} \} } dn_{j} } \\ {} & {=} & {-S_{i} dT+V_{i} dp+{\mathop{\sum }\limits_{j}} \frac{\partial \mu _{i} }{\partial n_{j} } |_{T,p,\{ n_{k\ne j} \} } dn_{j} } \end{array}\] }


{\bf  理想气体单组份,化学势的变化是}

\noindent 
{\bf $d\mu =-S_{m} dT+V_{m} dp$,$S_{m} ,V_{m} $分别是气体的摩尔熵和摩尔体积。}

\noindent 
{\bf 理想气体 $pV=nRT,\quad V_{m} =\frac{V}{n} =\frac{RT}{p} =V_{m} (T,p)$      }

\noindent 
{\bf 理想气体由$\left(T,p_{1} \right)$到$\left(T,p_{2} \right)$的化学势变化(等温变压)
\[\mu (T,p_{2} )-\mu (T,p_{1} )=\int _{p_{1} }^{p_{2} }\left. \frac{\partial \mathrm{{\mathcal U}}}{\partial p} \right|_{T}  dp=\int _{p_{1} }^{p_{2} }V_{m} (T,p) dp=RT\ln {\raise0.7ex\hbox{$ p_{2}  $}\!\mathord{\left/ {\vphantom {p_{2}  p_{1} }} \right. \kern-\nulldelimiterspace}\!\lower0.7ex\hbox{$ p_{1}  $}} \] }

{\bf 记标准状态$p^{\theta } \equiv 10^{5} Pa$下理想气体的化学势(必为常量)为$\mu ^{\theta } (T)\equiv \mu (T,p^{\theta } )$,则任一$\left(T,p\right)$下理想气体的化学势}

\noindent 
{\bf $\mu ^{id} (T,p)=\mu ^{\theta } (T)+{\rm RT}\ln \frac{p}{p^{\theta } } $ ,该式是确定的通式}

\noindent 
{\bf 对$\mu ^{id} $ 全微分有简单形式
\[d\mu ^{{\rm id}} =RT{\rm dlnp}\] }

{\bf 对于真实气体我们定义逸度$f$使得真实气体的化学势$\mu $也形如上式:}

\noindent 
{\bf $\mu (T,p)=\mu ^{\theta } (T)+RT{f\mathord{\left/ {\vphantom {f p^{\theta } }} \right. \kern-\nulldelimiterspace} p^{\theta } } $ ,所以逸度的定义:实际气体的逸度是相同条件下与其化学势相同的理想气体的压强。}

\noindent 
{\bf 从而$f$的定义是
\[f(T,p){\mathop{=}\limits^{{\rm def}}} p^{\theta } \exp \frac{{\rm \Delta }\mu (T,p)}{RT} \] }

{\bf $p\to 0$${\mathop{\lim }\limits_{p\to 0}} \gamma =1$$p\to 0$${\mathop{\lim }\limits_{p\to 0}} \gamma =1$使得 $\mu (T,p)=\mu ^{\theta } (T)+RT\ln \frac{f}{p^{\theta } } $ ,}

\noindent 
{\bf 逸度因子$\gamma ={f\mathord{\left/ {\vphantom {f p}} \right. \kern-\nulldelimiterspace} p} $ ,则$\mu (T,p)=\mu ^{{\rm id}} (\tau ,p)+RT\ln \gamma $ }


{\bf  理想气体混合物:设定存在只透组分$i$的半透膜,
\[\sum _{i}p _{i} =p_{i}^{{'} } \sum _{i}p _{i} =p_{i}^{{'} } \left. \begin{array}{c} {p_{1} ,...} \\ {n_{1} ,...} \\ {\mu _{1} ,...} \end{array}\right|p_{i} {}^{'} ,n_{i} {}^{'} ,\mu _{i} {}^{'} \] }

{\bf 则膜两边$i$的分压相等$p_{i} =p_{i}^{{'} } $ ,化学势相等$\mu _{i} =\mu _{i}^{{'} } $
\[\mu _{i}^{{'} } =\mu _{i}^{*} (T,p_{i} )=\mu _{i}^{\theta } (T)+RT\ln ^{2} \frac{p_{i} }{p^{\theta } } =\mu _{i} (T,p,\left\{n_{i} \right\})\] }

{\bf 由分压定律 $p_{i} =px_{i} $ 
\[\begin{array}{c} {\mathrm{\therefore }\mu _{i} (\tau ,p,\{ n_{i} \} )=\mu _{i}^{\theta } (T)+RT\ln \frac{p}{p^{\theta } } +RT\ln x_{i} } \\ {=\mu _{i}^{*} (T,p)+RT\ln x_{i} } \end{array}\] }


{\bf  非理想气体混合物:$\mu _{i} \left(T,p,\{ n_{i} \} \right)=\mu _{i}^{\theta } (T)+RT\ln {f_{i} \mathord{\left/ {\vphantom {f_{i}  p^{\theta } }} \right. \kern-\nulldelimiterspace} p^{\theta } } $ }

\noindent 
{\bf 其中组分$i$在混合物中的逸度$f_{i} (T,p,\{ n_{i} \} )$ 顺带定义$f_{i} {\mathop{=}\limits^{{\rm det}}} p^{\theta } \exp \frac{\mu _{i} (T,p)-\mu ^{\theta } (T)}{RT} $ 组分$i$在混合物中的逸度因子$\gamma _{i} {\mathop{=}\limits^{det}} {f_{i} \mathord{\left/ {\vphantom {f_{i}  p}} \right. \kern-\nulldelimiterspace} p} _{i} ={f_{i} \mathord{\left/ {\vphantom {f_{i}  px_{i} }} \right. \kern-\nulldelimiterspace} px_{i} } $ ,则$\mu _{i} (T,p,\{ n_{i} \} )={\mathop{\mu _{i}^{id} (T,p,\{ n_{i} \} )}\limits_{\mathrm{\sim \sim \sim \sim \sim \sim \sim \sim \sim \sim }}} +RT\ln \gamma _{i} $ ,$i$在理想气体混合物中的化学势。}


{\bf  我们考虑混合液相,拉乌尔定律:$p_{i} =p_{i}^{*} x_{i} $,在$x_{j\ne i} \to 0$ 时满足,即$p_{i} =p_{i}^{*} (1-{\rm \Sigma }_{j\ne i} x_{j} )$,${\rm \Delta }p=p_{i}^{*} -p_{i} =p_{i}^{*} \sum _{j\ne i}x_{j}  $。$p_{i} {}^{*} $:纯$i$组分蒸气压; $p_{i} $:混合物中$i$组分蒸气压;可见由于组分$j\ne i$造成的蒸气压降与组分$j\ne i$浓度呈一次相似。}

\noindent 
{\bf 假设拉乌尔定律对每一组分在全浓度范围内称量(该对象被定义成?理想液体混合物?),在气液平衡时,假设气相是理想气体,
\[\begin{array}{rcl} {\mu {}_{i}^{l} (T,p{}_{i} )} & {=} & {\mu _{i}^{g} (T,p_{i} )v-pi/Äi„‹} \\ {} & {=} & {\mu _{i}^{\theta } (T)+RT\ln {\raise0.7ex\hbox{$ p_{i}  $}\!\mathord{\left/ {\vphantom {p_{i}  p^{\theta } }} \right. \kern-\nulldelimiterspace}\!\lower0.7ex\hbox{$ p^{\theta }  $}} =\mu _{i} {}^{\theta } (T)+RT\ln ({\raise0.7ex\hbox{$ p_{i}^{*} x_{i}  $}\!\mathord{\left/ {\vphantom {p_{i}^{*} x_{i}  p^{\theta } }} \right. \kern-\nulldelimiterspace}\!\lower0.7ex\hbox{$ p^{\theta }  $}} )=\mu _{i} {}^{\theta } (T)+RT\ln ({\raise0.7ex\hbox{$ p_{i}^{*}  $}\!\mathord{\left/ {\vphantom {p_{i}^{*}  p^{\theta } }} \right. \kern-\nulldelimiterspace}\!\lower0.7ex\hbox{$ p^{\theta }  $}} )+RT\ln x_{i} } \\ {} & {=} & {\mu _{i}^{*,g} (T,p_{i}^{*} )+RT\ln x_{i} } \end{array}\] }

{\bf 其中$\mu _{i}^{*,g} (T,p_{i} {}^{*} )=\mu _{i}^{\theta } (T)+RT{\raise0.7ex\hbox{$ p_{i} {}^{*}  $}\!\mathord{\left/ {\vphantom {p_{i} {}^{*}  p_{\theta } }} \right. \kern-\nulldelimiterspace}\!\lower0.7ex\hbox{$ p_{\theta }  $}} $是$i$气体在其饱和蒸气压$p_{i} {}^{*} $下的化学势(已假设理想气体)。纯$i$组分在温度T下达到气液平衡时,气相压力就是$p_{i} {}^{*} $,化学势就是$\mu _{i}^{*,g} (T,p_{i} {}^{*} )$,且$\mu _{i}^{*,l} (T,p_{i} {}^{*} )=\mu _{i}^{*,g} (T,p_{i} {}^{*} )$。而纯组分液相在封闭条件下(没有气相跟它达成气液平衡,外压自由施加给它的大小$p$)的化学势$\mu _{i}^{*} (T,p)$。若把外压增加至$p_{i} {}^{*} $,则有}

\noindent 
{\bf }

\noindent 
{\bf 故原式$\begin{array}{rcl} {\Leftrightarrow \mu {}_{i}^{l} (T} & {,} & {p).=\mu _{i}^{*,l} (T,p_{i}^{*} )+RT\ln x_{i} } \\ {=\mu {}_{i}^{*,l} (T} & {,} & {p)-\int _{p_{i} }^{p}V_{i}  dp'+RT\ln x_{i} } \end{array}$ }

\noindent 
{\bf 再要求$\int _{p_{i} }^{p}V_{i}  dp'=0$时$\mu _{i}^{l} (T,p)=\mu _{i}^{*,l} (T,p)+RT\ln x_{i} $,则该式定义了理想混合物。}


{\bf  非理想混合物1)每个组分偏离拉乌尔定律;2)气相不是理想气体;3)$\int _{p_{i} }^{p}V_{i}  dp'$不可忽略。}


{\bf  引入纯物质$i$液相活度$a_{i} \left(T,p,\{ n_{i} \} \right)$,使得
\[\mu _{i}^{l} (T,p,\{ n_{i} \} )=\mu _{i}^{*,l} (T,p)+RT\ln a_{i} \] }

{\bf 即$a_{i} (\tau ,p,\{ n_{i} \} ){\mathop{=}\limits^{\det }} x_{i} \exp \quad \frac{u_{i}^{l} (T,p,\{ n_{i} \} -\mu _{i}^{l,id} (T,p,\{ n_{i} \} )}{RT} $}

\noindent 
{\bf $i$$i$组分$i$在混合物中的活度因子$\gamma _{i} {\mathop{=}\limits^{{\rm def}}} {a_{i} \mathord{\left/ {\vphantom {a_{i}  x_{i} }} \right. \kern-\nulldelimiterspace} x_{i} } $,则$\mu _{i} {}^{l} (T,p,\{ n_{i} \} )={\mathop{\mu _{i}^{{\rm l,id}} (T,p,\{ n_{i} )}\limits_{\mathrm{\sim \sim \sim \sim \sim \sim \sim \sim \sim }}} +RTln\gamma _{i} $ }

\noindent \eject 

\noindent \textbf{结构与动力学的标度律}

\noindent 主要关注的是$S(q,t)$和$G^{*} \left(\omega \right)$的标度律,Martin等人做的工作最全面。


{\bf  静态光散射(静态结构的标度律应该通过Fisher指数$\tau $联系道德普适类。原始指数应是$g(r,\varepsilon )\propto r^{-(d-2+y)} F_{G} (r\xi ^{-1} )$(这是来自动态结构因子,686026906与$K_{T} $的关系的6860269066860269061740295412的关系的?字体分辨不清)是分形的,才有$q(r)\propto r^{d_{f} -d} $,进而有$\eta \equiv 2-d_{f} $)}


{\bf  一个分形团簇在其内部满足}

\noindent 
{\bf $g(r)\propto r^{d_{f} -d} $ ,$a\ll r<Rduster$,统计$g(r)$时要选择远离团簇表面的部分。}

\noindent 
{\bf 其中$a$是粒径}

\noindent 
{\bf 若用团簇的均方回转半径$R_{g} $作为$Rduster$,按定义$S\propto R_{g} {}^{df} $}

\noindent 
{\bf \includegraphics*[width=1.19in, height=0.99in]{image4}由}

\noindent 
{\bf $S(q)=1+4\pi \rho q^{-1} \int dr r\sin (qr)[g(r)-1]$这个可见Wikipedia的?radial distribution function?$\Rightarrow S(q)\propto q^{d-3-df} $(Mathematical可以处理那个积分)}

\noindent 
{\bf            当$d=3$时,$S\left(q\right)\propto q^{-df} $,$a\ll q^{-1} <Rduster$}

\noindent 
{\bf 单团簇散射$I(q)\propto M^{2} S(q)$,故$I(q)\propto q^{-df} $(相同$q$范围)}

\noindent 
{\bf PRL52:2371(1984)}

\noindent 
{\bf SLS和SAXS实验,稀团簇悬浮液,在$q>a^{-1} $(SAXS)时$I\left(q\right)$进入porod区$\propto q^{-4} $(作者讨论认为$I\left(q\right)$没受$N\left(m\right)$分布影响(后详))。}


{\bf  当$q$与团簇间距相当(团簇较浓,多团簇散射),$I\left(q\right)$就不仅依赖单团簇结构因子,现沦为$S_{M} \left(q\right)$,还依赖团簇质量分布$N\left(m\right)$(或$n_{s} $),体系的结构因子:}

\noindent 
{\bf $S(q)={\smallint M^{2} N(M)S_{M} (q)dM\mathord{\left/ {\vphantom {\smallint M^{2} N(M)S_{M} (q)dM \smallint M^{2} N(M)dM}} \right. \kern-\nulldelimiterspace} \smallint M^{2} N(M)dM} \left(\equiv â„ÍGPÏM_{w} \right)$,此处$S_{M} \left(q\right)$就相当于平时说的?形状因子?(from factor名委)。}

\noindent 
{\bf (参考:MRS Online Proc.Library 367:447(1994),Phys.Rev.A 31:1180(1985))}

\noindent 
{\bf 其中,团簇都是有限大小,故引入截断函数$f\left(x\right)$,$g(r)=f({r\mathord{\left/ {\vphantom {r Rs}} \right. \kern-\nulldelimiterspace} Rs} )r^{df-d} $,其中$Rs$是某$S-$团簇的半径,当$x>1$时$f\left(x\right)$比所有幂律都快地(即指数式)衰减到零,$f\left(x\right)\sim 1,x\ll 1$。}


{\bf  按照逾渗理论的标度律假设,$N(M)=M^{-\tau } h(\varepsilon M^{\sigma } )$,其中$h$也是个截断函数。若考虑$\tau $,$\sigma $与$\beta $的关系,由于
\[P_{\infty } \propto \left\{\begin{array}{c} {\varepsilon ^{\beta } ,\varepsilon >0} \\ {0,\varepsilon <0} \end{array}\right. \] }

{\bf 故$h\left(x\right)$需满足约束$k\smallint z^{-\beta } h'(-z)dz=0$,$h'\left(x\right)$是$h\left(x\right)$的导函数}


{\bf  多团簇散射光强}

\noindent 
{\bf $I(q)\propto \smallint M^{2} N(M)S_{M} (q)dM$,其中${\rm S}_{{\rm M}} (\vec{q})=M^{-1} \smallint d^{d} \vec{r}g(\vec{r})e^{i\vec{q}\cdot \vec{r}} $与之前定义形式略有不同但等价。代入$N\left(m\right)$的开式,在$\varepsilon \to 0$时,$I(q)\propto q^{-\mu } $,$\mu =df(3-\tau )+3-d$,当$d=3$时$\mu =df(3-\tau )$。(与Fisher指数联系之后,各普适类下取之?是否有实验验证)}

\noindent 
{\bf }


{\bf  扩散系数Phuys.Rev A 43:858(1991)($n={\raise0.7ex\hbox{$ z $}\!\mathord{\left/ {\vphantom {z y}} \right. \kern-\nulldelimiterspace}\!\lower0.7ex\hbox{$ y $}} $那么这里的$n$与$z,y$模型取得一致吗?)}


{\bf  来自de Gennes的思想J.Phys.Left.40:197(1979),一个半径为R的团簇在其它团簇存在下的扩散,认为在扩散时间$\tau _{p} $内,比R大的团簇近似精致,比R小的团簇近乎平均化为均相液体,粘度$\eta $依赖讨论的R,$\eta =\eta (R)$故当该团簇尚未碰到比其大的团簇前(相当于考虑短时扩散),可视为在$\eta (R)$液体中自扩散,符合Stokes-Einstein扩散系数式}

\noindent 
{\bf $D$$\eta $$d_{f} $$g(r)$$\eta $$D$$\eta $$d_{f} $$g(r)$$\eta $$D\propto [\eta (R)R^{d-2} ]^{-1} $,一般维数Stokes-Einstein关系的表达式参见例如J.Chem.Phys.139:/64102(2013)}


{\bf  若考虑逾渗模型,团簇的幂率分布体现一种自相似性,具体地,由}

\noindent 
{\bf $N(M)=M^{-\tau } h(\varepsilon M^{\sigma } )$,在$M\sim M+dM$区间的团簇数是$N(M)dM$,在$\varepsilon \to 0$$N(M)dM\alpha M^{-2} dM$由强标度率$\frac{d}{df} +1=\tau $,$N(M)dM\propto M^{-1-d/df} dM\propto R^{-df-d} d(R^{df} )\alpha R^{-df-d+df-1} dR\propto R^{-d} d\ln R$}

\noindent 
{\bf 故该区间团簇数可记作$N(R)d\ln R$其中$N(R)\alpha R^{-d} $,是尺寸在$R\sim R+dlnR$区间的团簇数。}

\noindent 
{\bf 考虑尺寸为R的团簇的间距也满足自相似性,即不同尺寸的团簇感受到相似的拥护程度,诚然,上述强标度律结果使得。}

\noindent 
{\bf ${\rm dcluster}\propto \frac{1}{N(R)^{{1\mathord{\left/ {\vphantom {1 d}} \right. \kern-\nulldelimiterspace} d} } } \propto R$ 其中,R是数密度,且$N\left(R\right)\propto R^{-d} $ }

\noindent 
{\bf 即每个尺寸的团簇都感受到相似的环境。上述推导在相应条件下的等价性反过来说明了强标度律来自自相似的物理图象。}

\noindent 
{\bf 有了团簇间距${\rm dcluster}\propto R$的结果,尺寸为R的团簇在扩散时间$\tau _{p} $内是自由扩散的。}


{\bf  现考虑$\eta \left(R\right)$,假设已知宏观(零切)粘度$\eta $,实验表明$\eta \propto \varepsilon ^{-k} $(这是的$k$定义),$\varepsilon \to 0$,只是$k$的值与$\alpha ,\beta ,...$的标度关系尚未明确。我们假设,在某$\varepsilon $下某$R$对应的$\eta \left(R\right)$近似于以R为团簇分布截断($N\left(R\right)$是以$z$的尺寸$R_{z} $来截断的)的另一反应程度$\varepsilon '$下的宏观粘度,则由$\eta \propto \varepsilon ^{-k} $,(与$k$相关,只是$k$本身依赖模型了)$R_{z} \propto e^{-\nu } \Rightarrow g(R)\propto R^{{k\mathord{\left/ {\vphantom {k \nu }} \right. \kern-\nulldelimiterspace} \nu } } $ ,所以$D(R)\alpha [\eta (R)R^{d-2} ]^{-1} \alpha R^{-(d-2+k/\nu )} $}


{\bf  上述考虑反过来提供了讨论k的方式。假如团簇扩散的物理模型可给出$InD/InR$,则$\ln D/\ln R=-(d-2+k/\nu )$。}

\noindent 
{\bf 例如Rouse动力学(hydrodynamic introduction completely screened)预测$D\propto R^{-df} $,故给出}

\noindent 
{\bf ${\rm 0}\le k\le v\left(df-f+2\right)$${\rm 0}\le k\le v\left(df-f+2\right)$$k=\nu (df-d+2)$  HI completely screened                  }

\noindent 
{\bf 反之Zimm动力学预测$D\alpha R^{-(d-2)} $,故  }

\noindent 
{\bf $k=0$  HI completely screened}

\noindent 
{\bf 进一步借用Isaacson and Lubensky关于聚合物分形的结果,例如无扰支化聚合物团簇满员$d_{f} =\frac{d+2}{d} $(phantom cluster),$\Rightarrow 0\le k/\nu \le (6-d)/2$这说明平均场极限($d=6$)$k=0$且粘度仅按对数发散,而$d=3$的情况$0\mathrm{\} }k\mathrm{\} }1.35$。 }


{\bf  动态结构因子}


{\bf  Martin(PRA 43:858)的讨论:作为逾渗转变的临界凝胶转变不是液液相分离(进而不是热力学相变),在临界转变时不会有散射奇点(换句话说就是逾渗是二级/连续相变)。除非是稀释了的溶液且折射率未分配。这是假设的是团簇的折射率的dn/dc都相同。}


{\bf  由此设定,单体级别的密度涨落在整个凝胶化过程中是恒定背景。质量为M的团簇造成的结构因子$S(q)\propto M$而不是$S(q)\propto M^{2} f(qR)$,其中$f(qR)$是相干团簇形状因子,故
\[S_{self} (q,t)=\smallint _{1}^{\infty } MN(M)e^{-q^{2} Dt} dM\] }

{\bf 在$\varepsilon \to 0$时,$N(M)=M^{-1-d/d_{f} } {\mathop{e^{-M/M_{z} } }\limits_{\mathrm{\sim \sim \sim \sim }}} $(截断),利用$D\alpha R^{-1-k/\nu } $和强标度$d_{f} =d-\beta /v$
\[S_{self} (q,t)\alpha t^{-\beta /(\nu +k)} ,I^{-1} <t<I_{z} ,\] }

{\bf $\omega \frac{-\gamma }{y} $$\tau _{z} $$\eta $$\begin{array}{l} {\omega _{0} \propto \xi ^{-z} \propto \varepsilon ^{-z} } \\ {\overline{z}=s+t} \end{array}$$\begin{array}{rcl} {\xi } & {\propto } & {\varepsilon ^{v} } \\ {\xi {}^{-1-k/v} } & {=} & {\varepsilon ^{v(-1-k/v)} } \\ {} & {=} & {\varepsilon ^{-v-k} } \end{array}$$\begin{array}{l} {\eta _{0} \propto \varepsilon ^{-s} ,\omega 0={\raise0.7ex\hbox{$ y_{0}  $}\!\mathord{\left/ {\vphantom {y_{0}  G_{0} }} \right. \kern-\nulldelimiterspace}\!\lower0.7ex\hbox{$ G_{0}  $}} ?} \\ {G_{0} \propto \varepsilon ^{t} ,n={\raise0.7ex\hbox{$ t $}\!\mathord{\left/ {\vphantom {t \omega _{0} vs\varepsilon }} \right. \kern-\nulldelimiterspace}\!\lower0.7ex\hbox{$ \omega _{0} vs\varepsilon  $}} } \end{array}$$\omega \frac{-\gamma }{y} $$\tau _{z} $$\eta $$\begin{array}{l} {\omega _{0} \propto \xi ^{-z} \propto \varepsilon ^{-z} } \\ {\overline{z}=s+t} \end{array}$$\begin{array}{rcl} {\xi } & {\propto } & {\varepsilon ^{v} } \\ {\xi {}^{-1-k/v} } & {=} & {\varepsilon ^{v(-1-k/v)} } \\ {} & {=} & {\varepsilon ^{-v-k} } \end{array}$$\begin{array}{l} {\eta _{0} \propto \varepsilon ^{-s} ,\omega 0={\raise0.7ex\hbox{$ y_{0}  $}\!\mathord{\left/ {\vphantom {y_{0}  G_{0} }} \right. \kern-\nulldelimiterspace}\!\lower0.7ex\hbox{$ G_{0}  $}} ?} \\ {G_{0} \propto \varepsilon ^{t} ,n={\raise0.7ex\hbox{$ t $}\!\mathord{\left/ {\vphantom {t \omega _{0} vs\varepsilon }} \right. \kern-\nulldelimiterspace}\!\lower0.7ex\hbox{$ \omega _{0} vs\varepsilon  $}} } \end{array}$$S_{self} $是外差(heterodyne)相关函数(即它是$g^{\eqref{GrindEQ__1_}} (t)$)。它与零差(homodyne)相关函数的关系可设定Siegert关系成立,$g^{\eqref{GrindEQ__2_}} (t)=B(1+fg^{(1)} (t)|^{2} )$,故$g^{\eqref{GrindEQ__2_}} (t)\propto t^{-2\beta /(\nu +k)} $}


{\bf  截断时间尺度$\tau _{z} =\frac{1}{q^{2} D_{z} } $,$D_{z} $是z均扩散系数,$D_{z} \propto \xi ^{-1-k/\nu } \Rightarrow \tau _{z} \propto \varepsilon ^{-\nu -k} $}


{\bf  线性粘弹性 PRL61:2620(1991) Phys.Rev,A 39:1325(1989)}


{\bf  实验表明,在$\varepsilon \to 0$时,$G(t)\propto t^{-n} ,G'\sim G''\sim \omega ^{n} ,H(\tau )d\ln \tau \propto \tau ^{-n} d\ln \tau $}


{\bf  考虑一个分子团簇的松弛时间分布,类似线形链的Rouse模型考虑放法,第j个简正模的松弛时间
\[\begin{array}{c} {\tau _{z} =\frac{1}{qDz} \propto \frac{1}{q^{2} \xi ^{-1} {}^{-kN} } } \\ {T_{z} \propto \xi ^{1+{\raise0.7ex\hbox{$ k $}\!\mathord{\left/ {\vphantom {k v}} \right. \kern-\nulldelimiterspace}\!\lower0.7ex\hbox{$ v $}} } ,z=1+{\raise0.7ex\hbox{$ k $}\!\mathord{\left/ {\vphantom {k v}} \right. \kern-\nulldelimiterspace}\!\lower0.7ex\hbox{$ v $}} } \\ {T_{z} \propto \xi ^{z} } \end{array}\begin{array}{c} {\tau _{z} =\frac{1}{qDz} \propto \frac{1}{q^{2} \xi ^{-1} {}^{-kN} } } \\ {T_{z} \propto \xi ^{1+{\raise0.7ex\hbox{$ k $}\!\mathord{\left/ {\vphantom {k v}} \right. \kern-\nulldelimiterspace}\!\lower0.7ex\hbox{$ v $}} } ,z=1+{\raise0.7ex\hbox{$ k $}\!\mathord{\left/ {\vphantom {k v}} \right. \kern-\nulldelimiterspace}\!\lower0.7ex\hbox{$ v $}} } \\ {T_{z} \propto \xi ^{z} } \end{array}\tau _{j} \alpha j^{-\alpha } \tau _{R} (1\le j\le M)\] }

{\bf 记团簇扩散系数是$D\propto R^{-b} $,之前已经给过b的具体表达式
\[T_{R} \propto \frac{R^{2} }{D} \propto R^{2+b} ,T_{j} \propto j^{-\alpha } R^{2+b} (1\le j\le M)\] }

{\bf 又考虑最大的简振模$\tau _{m} $应当不依赖团簇大小,即$\alpha $应当使$j=M$时
\[\tau _{m} \propto M^{-\alpha } R^{2+b} \propto M^{-\alpha } M^{(2+b)/df} \alpha M^{o} \Rightarrow \alpha =-\frac{2+b}{df} \] 
\[\therefore \tau _{j} \alpha (j/m)^{(2+b)/d_{f} } \] }

{\bf 对于线性聚合物的Rouse模型(HI screened),$b=d_{f} $,对于无扰链$d_{f} =2$,这也退回到Rouse模型中简振模时间
\[\tau _{j} \propto ({\raise0.7ex\hbox{$ j $}\!\mathord{\left/ {\vphantom {j M}} \right. \kern-\nulldelimiterspace}\!\lower0.7ex\hbox{$ M $}} )^{2} \] }

{\bf 的结果。}


{\bf  根据Boltzmann叠加原理,在线性响应中的每个简正模的松弛线性叠加或总松弛,即}

\noindent 
{\bf $G_{M} (t)=\smallint _{1}^{m} e^{-t/\tau _{i} } dj\propto M/t^{df/(2+b)} ,\tau _{0} \mathrm{\} }t\mathrm{\} }\tau _{R} $ 线性无扰Rouse是$G(t)\propto {\raise0.7ex\hbox{$ M $}\!\mathord{\left/ {\vphantom {M t^{-0.5} }} \right. \kern-\nulldelimiterspace}\!\lower0.7ex\hbox{$ t^{-0.5}  $}} $}


{\bf  当b去之前得到的一般式$b=d-2+k{\rm /}\nu $时,${\rm G}_{{\rm M}} (t)\propto M/t^{{\raise0.7ex\hbox{$ d_{f}  $}\!\mathord{\left/ {\vphantom {d_{f}  (d+k/\nu )}} \right. \kern-\nulldelimiterspace}\!\lower0.7ex\hbox{$ (d+k/\nu ) $}} } $ 。单团簇松弛时间谱
\[H(\lambda )d\ln \lambda \propto M\lambda ^{-d_{f} v/(d\nu +k)} d\ln \lambda ,\tau _{0} \le \lambda \le \tau _{k} \] }

{\bf 同时$\tau _{R} \propto R^{2+b} \propto R^{d+k/\nu } \propto M^{(d\nu +k)/(d_{f} \nu )} $}


{\bf  假设多团簇松弛时间谱是按照分布$N(M)$的加权$H(\lambda )={\mathop{\sum }\limits_{M}} N(M)H_{M} (\lambda )$}

\noindent 
{\bf 由$N(M)\propto M^{-\tau } \propto M^{-1-d/d_{f} } $,把求和积改为积分,}

\noindent 
{\bf $H(\lambda )\propto \smallint N(M)H_{M} (\lambda )dM\propto \smallint M^{-1-d/d_{f} } M\lambda ^{-d_{f} v/(dv+k)} \boxed{\exp (-\frac{\lambda }{\tau _{k} } )} $ (截断函数)}

\noindent 
{\bf $\tau _{R} $$\lambda $$\tau _{R} $$\lambda $情况1:Rouse,$k=\nu \left(df-d+2\right)$,$\begin{array}{rcl} {\tau {}_{R} } & {\propto } & {M^{(d\nu +k)/(d_{f} \nu )} } \\ {} & {\propto } & {\lambda ^{-d_{f} +\nu /(d\nu +k)} \smallint M^{-d/d_{f} } \exp \left(-\lambda /M^{(d\nu +k)/(d_{f} v)} \right)} \\ {} & {\propto } & {\lambda ^{-d_{f} \nu /(d\nu +k)} {}^{+\nu (d_{f} -d)/(d\nu +k)} } \\ {} & {\propto } & {\lambda ^{-dv/(dv+k)} \propto \lambda ^{-d/(d_{f} +2)} } \end{array}$}

\noindent 
{\bf $H(\lambda )\propto \lambda ^{-d\nu /(d\nu +k)} F(\lambda /\lambda _{z} )$$\lambda z\propto \xi ^{2} /D(\xi )$$D$$\propto \lambda ^{-d_{f} {}^{{'} } /d} $$T_{z} \propto \left\{\begin{array}{l} {|p-p_{c} |^{-d\nu -k} {\rm conc}.} \\ {|p-p_{c} |^{-d\nu } {}^{(d/d_{f} ')} {\rm dilute.}} \end{array}\right. $$H(\lambda )\propto \lambda ^{-d\nu /(d\nu +k)} F(\lambda /\lambda _{z} )$$\lambda z\propto \xi ^{2} /D(\xi )$$D$$\propto \lambda ^{-d_{f} {}^{{'} } /d} $$T_{z} \propto \left\{\begin{array}{l} {|p-p_{c} |^{-d\nu -k} {\rm conc}.} \\ {|p-p_{c} |^{-d\nu } {}^{(d/d_{f} ')} {\rm dilute.}} \end{array}\right. $情况2:Zimm,$k=0$,$\tau _{R} \propto M^{d/d_{f} '} $ ,同样推导得 
\[H(\lambda )\propto \lambda ^{-d_{f} {}^{{'} } /d} \] }

{\bf 常默认Rouse时用non swollen的$d_{f} $,Zimm时则用swollen的$d_{f} '$。$d_{f} '$与$d_{f} $的关系已在上一笔记给出:}

\noindent 
{\bf $d_{f} =\frac{2d_{f} '}{2+d-2d_{f} '} \Leftrightarrow d_{f} '=\frac{d_{f} (d+2)}{2d_{f} +2} $故Rouse情况下$n=\frac{d}{d_{f} +2} =\frac{d(2+d-2d_{f} ')}{2(2+d-d_{f} ')} $}

\noindent 
{\bf 这个式子之所以有用是因为聚合物团簇分形维数常在稀释条件下测量,故测量得到的是$d_{f} '$。此时如果是凝胶化体系却属于浓稠的$n$就要用Rouse情况式。如果先知道$n$,该用哪个情况需额外决定,计算的$d_{f} $和$d_{f} '$在两个情况之间是不同的。具体如下
\[\left\{\begin{array}{l} {{\rm Rouse}:n=\frac{d}{d_{f} +2} =\frac{d(z+d-zd_{f} ')}{2(z+d-d_{f} ')} } \\ {{\rm Zimm}:n=\frac{d_{f} (d+2)}{d(zd_{f} +2)} =d_{f} '/d} \end{array}\right. \] }

{\bf J.Non-Cryst.Solids 172-274:1151(1994)}


{\bf  Doi\&Onuki.J.Phys.ⅡFr.2:1631(1992)用two-fluid model(考虑了高分子溶液动力学,并认为对于逾渗点前的分形团簇,动态结构因子的指数$g^{\eqref{GrindEQ__2_}} (t)\propto t^{-{\rm \Phi }} ,{\rm \Phi }=2n$。但按照Martin的结果,$\Phi =2\beta /(\nu +k)$}

\noindent 
{\bf Rouse:$\Phi =2\beta /(\nu +\nu (d_{f} -d+2))=\frac{2\beta }{\nu (3-d+d_{f} )} $把$\beta $转成含有$d$,$d_{f} $和$v$两式,$\beta =\nu \left(d-df\right)$得${\rm \Phi }=\frac{2v(d-d_{f} )}{v(3-d+d_{f} )} =\frac{2(d-d_{f} )}{3-d+d_{f} } $}

\noindent 
{\bf Zimm:$\Phi =d-d_{f} $ 都不等于2n,且强行令等式成立将给出不合理的$d_{f} $。对于Doi\&Onuki结论对凝胶化的不适用性也被S.Richter确认(见其综述Macomel.Chem.Phys.208 :1495(2007))}


{\bf  Adam and Lairez(1996)(原文难看到)说$\Phi =n$。}


{\bf  独立实验测量$n$与$\Phi $返现$n$与$\Phi $没有固定关系。}

\noindent \eject 

\noindent \textbf{聚合物的分形维数和Fisher指数${}^{*}$}

\noindent 参考资料:Stauffer Phys Rep.54:1(1979) Adv. Polym, Sci 44:103 (1982)

\noindent Flory-Stockmayer模型  JAC8 63:3083,63:3091,63:3096(1941) J.Phys.Chem,46:132(1942) J.Chem.Phys.11:45(1943) 12:125(1944) 17:1301(1949)

\noindent Hiemenz and Lodge的书 JACS 58:1877(1936)

\noindent 单分散长链高分子的硫化


{\bf  考虑聚合度均为$N$的长链分子,每个单元都能发生交联反应,发生多少个交联时才形成网络(凝胶点?)}


{\bf  设$p$为参与交联的单元数与总单元数之比,将其视为概率,平均每条链上有几个交联点?设一共有$M$条链,则总交联单元数是$MNp$,平均每条链上的交联单元数是$Np$,为了每条链都连上网络,平均每条链上都至少有一个交联单元,即$Np>1$,定义$p_{c} =N^{-1} $为凝胶点。}


{\bf  以上估算忽略了交联点分布的不均匀之性,loops/dangling,以及多分散性。}


{\bf  Hiemenz and Lodge书上考虑多分散性,其余理想假设保持,得到$p_{c} ={1\mathord{\left/ {\vphantom {1 \overline{N_{w} }}} \right. \kern-\nulldelimiterspace} \overline{N_{w} }} $,$\overline{N_{w} }$是重均聚合度。线性缩聚的分子量分布(JACS 58:1877(1936),Flory书Ch.Ⅷ)}


{\bf  假设反应集团的反应活性是恒定的。缩聚单体体系有两类:1)A-B$\mathrm{\to}$A$\mathrm{\sim}$$\mathrm{\sim}$$\mathrm{\sim}$B;2)A-A+B-B$\mathrm{\to}$A$\mathrm{\sim}$$\mathrm{\sim}$$\mathrm{\sim}$B,,其中基团A只跟基团B反应。设$p$为已反应的A与总A的数量之比。}


{\bf  考虑第1)种体系,并且?把浓度当概率?,任一单元为已反应单元的概率就是$p$。}


{\bf  要形成一条聚合度为$x$的链相当于连续$x-1$个单元为已反应单元(聚合度为$x$的链只需成$x-1$个键),概率就是$p^{x-1} $。要使该链聚合度停在$x$,第$x$个单元必须是未反应的,这一概率是$1-p$。因此,聚合度为$x$的链出现概率为(任选一单元,抽起来是一根链,这个做法可以遍历所有单元,虽然与很多次抽出来的是同一条链。在所有次数中,抽出了一条聚合度为$x$的链的次数之比)
\[p^{x-1} (1-p)\] }

{\bf 这是任一单体属于一条聚合度为$N$的链的概率。?把概率当浓度?,这就是$x$聚合体的数量分数。}


{\bf  设总共有$N$条链,则$x$聚体个数为$N_{x} =N(1-p)p^{x-1} $}


{\bf  $Sû„sGZ| ={;UCp\mathord{\left/ {\vphantom {;UCp {\rm p} p}} \right. \kern-\nulldelimiterspace} {\rm p} p} =1/(1-p)$,设总单元数为$N_{0} $,则$N=N_{0} (1-P)$,$N_{x} =N_{0} (1-p)^{2} p^{x-1} $}


{\bf  忽略缩聚反应掉的小分子产物,$x$聚合体分子量与$x$成正比,故$x$聚合体质量分数
\[w_{x} =xNx/N_{0} =x(1-p)^{2} p^{x-1} \] }


{\bf  对于第2)类体系,推导结果相同。此时$x$代表A与B的结合数。对于这类体系还有必要讨论A-A与B-B不等化学计量的情况。}

\noindent 线性聚合物的分形维数


{\bf  线性聚合物的均方回转半径$\left\langle \overrightarrow{R}\right\rangle \propto N^{2\nu } $,这里的$v$与1065713229渝渗1065713229cdchengcheng deng1065713229666608339逾渗?模型讨论的$v$不同,只是按照高分子资料的惯例。实际上这里的$v={1\mathord{\left/ {\vphantom {1 d_{f} }} \right. \kern-\nulldelimiterspace} d_{f} } $,$d_{f} $是分形维数。以下讨论假设$d=3$}


{\bf  $v=\frac{1}{2} (d_{f} =2)$,高斯链,$\theta $条件下的链,无扰链 ?excluded volume completely screened?}

\noindent 
{\bf $\nu =\frac{3}{5} (d_{f} =\frac{5}{3} )$,自回避,稀溶液(良溶剂)}

\noindent 多官能度单体的缩聚


{\bf  考虑这一问题的重点是先要计算成对交联点的浓度(概率),或者说,任一夺冠能单元,沿一头支链能够找到下一个多官能单元(而不是到了链端)的概率。}


{\bf  设多官能单元集团为A,二官能度单体为A-A,B-B,只有A-B之间可反应。则问题如下图所示}

\noindent 
{\bf \includegraphics*[width=2.36in, height=0.40in]{image5}(以三官能度为例)}

\noindent 
{\bf 设$p_{A} $是反应了的集团A数与总基团A数之比,$p_{B} $类似,$i$是两交联点间重复单元数,$\rho $是属于多官能单元的基团A与总基团A数之比。一个B基团与多官能单元基团A反应的概率$p_{B} \rho $:一个B基团与线性单体的A基团反应的概率是$p_{B} (1-\rho )$。任选一个基团A,其为上图这总结构中的第一个基团A的概率是
\[p_{A} [p_{B} (1-\rho )p_{A} ]^{i} p_{B} \rho \] }

{\bf 设$\alpha $是任选一个多官能基团A顺一头链找到的是交联点而不是线形链端的概率,它是上述链间聚合度$i=0,1,\cdots ,\infty $的子概率之和:
\[\begin{array}{rcl} {\alpha } & {=} & {\mathop{\sum }\limits_{i=0}^{\infty } \left[p_{A} p_{B} (1-\rho )\right]^{i} p_{A} p_{B} \rho } \\ {} & {=} & {\frac{p_{A} p_{B} \rho }{1-p_{A} p_{B} (1-\rho )} \quad (Mathematicaïô\yen úÓœ{\rm )}} \end{array}\] }

{\bf 设A基团与B基团数量比为$r$,则$p_{B} =rp_{A} $,
\[\alpha =\frac{rp_{A}^{2} \rho }{1-rp_{A}^{2} (1-\rho )} =\frac{p_{B}^{2} \rho }{r-p_{B}^{2} (1-\rho )} \] }

{\bf $r$和$\rho $可由投料情况计算}


{\bf  存在临界$\alpha $值$\alpha _{c} $体系恰好形成了网络,以3官能度为例,若$\alpha <{1\mathord{\left/ {\vphantom {1 2}} \right. \kern-\nulldelimiterspace} 2} $,则无法满足每个3官能单体有两条支链都不是dangling------临界凝胶条件。一般地若多官能度单团官能度为$f$,则$\alpha _{c} =1/(f-1)$}

\noindent 
{\bf (继续推到分子量分布和$\tau $未完待续)}

\noindent 线性与支化聚合物的分形维数

\noindent 以下是旧笔记,待完善。另外要参看Soft Matter13:1223(2017)

\noindent The $d_{f} $ of polymeric fractal:

\noindent In   $\thetaup$-solvent/condition                     Flory-Stockmayer(Caley tree) 

\noindent      excluded volume completely screened           result : $d_{f} =4,d=3$

\noindent Numerical result of standard percolation:$d_{f} =\cdots $ 

\noindent Issacson and Lubesky(1980)J.Phys.Left.41:149 Considered the excluded

\noindent volume effect by Flory-Rehner type model, retaining the Caley tree or Bethe Cattice Selfter

\noindent $F_{el} $ : elastic energy $F_{el} \sim \frac{R^{2} }{R_{0}^{2} } $

\noindent $R_{0} $ : the Gaussian/ideal chain radius of gyration.

\noindent $F_{el} =1$ as$R=R_{0} $

\noindent It is known that

\noindent $R_{0} \propto N^{{1\mathord{\left/ {\vphantom {1 df_{0} }} \right. \kern-\nulldelimiterspace} df_{0} } } $ for linear polymer in $d=3,d_{f0} =2$(phantom chain)

\noindent In food solvent/with excluded volume effect, another energy:

\noindent $F_{req} \sim \frac{N^{2} }{R^{d} } $ (dilute solutions,这个式子待确认) $N$ is the $DP$.

\noindent $F_{req} \sim \frac{N^{2} }{R^{d} } \frac{1}{N_{W} } $(melts) $N_{W} $ is the weight-averaged $DP$.

\noindent So, dilute solutions corresponds to $N_{W} \sim N^{0} $, monodispersed melts

\noindent correspond to $N_{W} \sim N$, in gental $N_{W} \sim N^{\rho } $ $\rho $:screening exponent

\noindent 

\noindent The upper-critical dimension, below which mean-field theory breaks down, is determined by $F_{req} (R=R_{0} )\sim 1{\rm \; \; \; }\therefore d_{c} =(1-\rho )d_{f_{0} } $ which is used to find the values of $d_{c} $ for dilute solution, monodispersed welt and gelation.

\noindent 

\noindent Now to find $d_{f} $, minimizing $F=F_{el} +F_{rep} $, with respect to$R$, we find
\[R^{d+2} \sim R_{0}^{2} \frac{N^{2} }{N_{w} } \] 
$\boxed{d_{f} =\frac{d_{t} {}^{2} }{2(1+1/d_{f_{0} } )-\rho } } $ Basic relation by $F=F_{el} +F_{rep} $

\noindent Now we discuss cases of $\rho $, $d_{f_{0} } $ and hence $d_{c} $ 


{\bf  Dilute linear polymers: $df_{0} =2,\rho =0,d_{c} =4,\quad d_{f} =\frac{d+2}{3} (\boxed{=\frac{5}{3} ,d=3} )$ (Flory result)}


{\bf  Dilute branched polymers: $\boxed{df_{0} =4} ,\rho =0,d_{c} =8,\quad d_{f} =\frac{2\left(d+2\right)}{5} (=2,d=3)$(Caley free)}


{\bf  Monodispersed linear melts: $df_{0} =2,\rho =1,d_{c} =2,\quad d_{f} =\frac{d+2}{2} (=\frac{5}{2} ,d=3)$}


{\bf  Monodispersed branched melts: $df_{0} =4,\rho =1,d_{c} =4,\quad d_{f} =\frac{2\left(d+2\right)}{3} (=\frac{10}{3} ,d=3)$}

\noindent Comments:


{\bf  Cafes (1984) PRL 53:926 pointed out the generalized form of $F_{req} $:}

\noindent 
{\bf $F_{{\rm rep}} \sim {\raise0.7ex\hbox{$ N^{n}  $}\!\mathord{\left/ {\vphantom {N^{n}  R^{(1-n)d} }} \right. \kern-\nulldelimiterspace}\!\lower0.7ex\hbox{$ R^{(1-n)d}  $}} $consider n-body interaction.}

\noindent 
{\bf Issacson \& Lubensky `s result is the special case of$n=2$(two-body interaction )}


{\bf  Issacson \& Lubensky separated the effect of dilute/concentrated with swollen/non-swollen. Minimizing $F=F_{el} +F_{rep} $ corresponds to swollen case (as oppose to Flory- stockmayer ) bat $\rho =0{\rm \; }or{\rm \; }1$ corresponds to dilute/concentrated. In other literatures however, if seems that dilute = swollen, concentrated = non-swollen .So in the language of Lubensky, $\rho \equiv 0$ for single polymeric fractal. For concentrated polymeric fractals, once consider the fractal not swollen at all($F_{rep} =0,R=R_{0} $).例如Muthukumar的文章, esp, M 22:4656中eq\eqref{GrindEQ__1_}两情况的意义。}

\noindent \eject 

\noindent \textbf{聚合物-溶剂A-溶剂B}

\noindent $\frac{{\rm \Delta }G_{mix} }{RT} =\mathop{\sum }\limits_{i=1}^{m} n_{i} \ln \varphi _{i} +\sum _{i=1}^{m}\sum _{j>i}x_{ij} n_{i} \varphi _{j}   $ (这一式子来自Flory的书P.549脚注,J.Chem.Phys.12:425(1944)同时推导了双组份和多组分)

\noindent $m=3$时,
\[\frac{{\rm \Delta }Gmix}{RT} =n_{1} \ln \varphi _{1} +n_{2} \ln \varphi _{2} +n_{3} \ln \varphi _{3} +\chi _{12} n_{1} \varphi _{2} +\chi _{13} n_{1} \varphi _{3} +\chi _{33} n_{2} \varphi _{3} \] 
$\varphi _{1} =\frac{n_{1} \nu _{1} }{n_{1} \nu _{1} +n_{2} \nu _{2} +n_{3} \nu _{3} } \ldots $ 等等是体积分数,$v_{i} $是摩尔体积(不考虑偏摩尔体积)

\noindent $\frac{\partial \varphi _{1} }{\partial n_{1} } =(1-\varphi _{1} )\frac{v_{1} }{V} ,\quad \frac{\partial \varphi _{1} }{\partial n_{2} } =-\varphi _{1} \frac{v_{2} }{V} $,等等

\noindent 另外$x_{i} {}_{j} $也常常依赖组成$u_{j} $,其中$u_{j} =\varphi _{j} /(\varphi _{i} +\varphi _{j} )$。$x_{ij} (u_{j} )$的数据要查资料。

\noindent $\frac{\partial x_{12} }{\partial n_{1} } =\frac{\partial x_{12} }{\partial u_{2} } \frac{\partial u_{2} }{\partial n_{1} } $,$\begin{array}{c} {\begin{array}{rcl} {\frac{\partial u_{2} }{\partial n_{1} } } & {=} & {\frac{\partial }{\partial n_{1} } \left(\frac{\varphi _{2} }{\varphi _{1} +\varphi _{2} } \right)=-u_{2} \frac{v_{1} }{V(\varphi _{1} +\varphi _{2} )} } \\ {\frac{\partial u_{2} }{\partial n_{2} } } & {=} & {\frac{v_{2} }{V} \frac{u_{1} }{\varphi _{1} +\varphi _{2} } ,\frac{\partial u_{2} }{\partial n_{3} } =0} \end{array}} \end{array}$ 

\noindent 假设只有$x_{12} $依赖$u_{2} $。(这是Altena and Smdders,M 15:1491(1982)的做法)
\[\begin{array}{rcl} {\boxed{\frac{\vartriangle \mu _{1} }{RT} } } & {=} & {\frac{\partial \vartriangle Gmix}{RT\partial n_{1} } =\ln \varphi _{1} +n_{1} \varphi _{1}^{-1} \frac{\partial \varphi _{1} }{\partial n_{1} } +n_{2} \varphi _{2}^{-1} \frac{\partial \varphi _{2} }{\partial n_{1} } +n_{3} \varphi _{3}^{-} \frac{\partial \varphi _{3} }{\partial n_{1} } +\frac{\partial \chi _{12} }{\partial n_{1} } n_{1} \varphi _{2} +n_{1} \chi _{12} \frac{\partial \varphi _{2} }{\partial n_{1} } } \\ {} & {} & {+\chi _{12} \varphi {}_{2} +n{}_{1} \chi {}_{13} \frac{\partial \varphi _{3} }{\partial n_{1} } +\chi _{13} \varphi {}_{3} +\chi {}_{23} n{}_{2} \frac{\partial \varphi _{3} }{\partial n_{1} } } \\ {} & {=} & {\ln \varphi _{1} +1-\varphi _{1} -s\varphi _{2} -r\varphi _{3} -u_{1} u_{2} \varphi _{2} \frac{\partial \chi _{12} }{\partial u_{2} } -\chi _{12} \varphi _{1} \varphi _{2} +\chi _{12} \varphi _{2} } \\ {} & {} & {-\chi {}_{13} \varphi {}_{1} \varphi {}_{3} +\chi {}_{13} \varphi {}_{3} -s\chi {}_{23} \varphi {}_{2} \varphi {}_{3} } \\ {} & {} & {\boxed{\begin{array}{l} {=\ln \varphi _{1} +1-\varphi _{1} -s\varphi _{2} -r\varphi _{3} -u_{1} u_{2} \varphi _{2} \frac{\partial \chi _{12} }{\partial u_{2} } +\chi _{12} \varphi _{2} \left(1-\varphi _{1} \right)} \\ {+\chi _{13} \varphi _{3} (1-\varphi _{1} )-s\chi _{23} \varphi _{2} \varphi _{3} } \end{array}} } \end{array}\] 
\[\begin{array}{rcl} {\frac{{\rm \Delta u}_{{\rm 2}} }{RT} } & {=} & {n_{1} \varphi _{1}^{-1} \frac{\partial \varphi _{1} }{\partial n_{3} } +n_{2} \varphi _{2} +n_{1} q_{2}^{-1} \frac{\partial \varphi _{2} }{\partial n_{3} } +n_{3} \varphi _{3}^{-1} \frac{\partial \varphi 3}{\partial n_{2} } +n_{1} \varphi _{2} \frac{\partial \chi _{12} }{\partial n_{2} } +n_{1} \chi _{12} \frac{\partial \varphi _{2} }{\partial n_{2} } +\chi _{13} n_{1} \frac{\partial \varphi _{3} }{\partial n_{2} } } \\ {} & {} & {+\chi _{23} \varphi {}_{3} +n{}_{2} \chi {}_{23} \frac{\partial \varphi _{3} }{\partial n_{2} } } \\ {} & {=} & {-s^{-1} \varphi _{1} +\ln \varphi _{2} +1-\varphi _{2} -\frac{r}{s} \varphi _{3} +s^{-1} \varphi _{1} u_{1} u_{2} \frac{\partial \chi _{12} }{\partial u_{2} } +s^{-1} \varphi _{1} (1-\varphi _{2} )\chi _{12} -\chi _{13} \varphi _{3} \varphi _{1} s^{-1} } \\ {} & {} & {+\chi {}_{23} \varphi {}_{3} -\chi {}_{23} \phi {}_{2} \phi {}_{3} } \end{array}\] 
\[\boxed{\frac{s{\rm \Delta }\mu _{2} }{RT} =s\ln \varphi _{2} +s-\varphi _{1} -s\varphi _{2} -r\varphi _{3} +\left(1-\varphi _{2} \right)\left(\chi _{12} \varphi _{1} +s\chi _{23} \varphi _{3} \right)-\chi _{13} \varphi _{1} \varphi _{3} +u_{1} u_{2} \varphi _{1} \frac{\partial \chi _{12} }{\partial u_{2} } } \] 
\[\begin{array}{c} {\begin{array}{l} {\frac{{\rm \Delta }U_{3} }{RT} =n_{1} \varphi _{1}^{-1} \frac{\partial \varphi _{3} }{\partial n_{3} } +n_{2} \varphi _{2}^{-1} \frac{\partial \varphi _{3} }{\partial n_{3} } +\ln \varphi _{3} +n_{3} \varphi _{3}^{-1} \frac{\partial \varphi _{3} }{\partial n_{3} } +n_{1} \chi _{12} \frac{\partial \varphi _{2} }{\partial n_{3} } +n_{1} \chi _{13} \frac{\partial \varphi _{3} }{\partial n_{3} } +n_{2} \chi _{23} s\frac{\partial \varphi _{3} }{\partial n_{3} } } \\ {=-\varphi _{1} r^{-1} -\varphi _{2} \frac{s}{r} +\ln \varphi _{3} +1-\varphi _{3} -r^{-1} \chi _{12} \varphi _{1} \varphi _{2} +r^{-1} \chi _{13} \varphi _{1} (1-\varphi _{3} )+\chi _{23} \varphi _{2} (1-\varphi _{3} )\frac{s}{r} } \end{array}} \end{array}\] 
\[\boxed{\frac{r{\rm \Delta }\mu _{3} }{RT} =r\ln \varphi _{3} +r-\varphi _{1} -s\varphi _{2} -r\varphi _{3} -\chi _{12} \varphi _{1} \varphi _{2} +(\chi _{13} u_{2} +s\chi _{23} \varphi _{2} )(1-\varphi _{3} )} \] 
其中$s=\frac{v_{1} }{v_{2} } ,\quad r=\frac{v_{1} }{v_{3} } $ 

\noindent 设液液相分离有两相,组成分别为$\varphi _{1} ',\varphi _{2} ',\varphi _{3} ',\varphi _{1} '',\varphi _{2} '',\varphi _{3} ''$,三个组分的化学式分别为$\vartriangle \mu _{1} ',\vartriangle \mu _{1} '',\vartriangle \mu _{2} '',\vartriangle \mu _{3} ''$,相平衡时
\[\left\{\begin{array}{c} {{\rm \Delta }\mu _{1}^{{'} } ={\rm \Delta }\mu _{2}^{{'} } } \\ {{\rm \Delta }\mu _{2}^{{'} } ={\rm \Delta }\mu _{2}^{{'} {'} } } \\ {{\rm \Delta }\mu _{3}^{{'} } ={\rm \Delta }\mu _{3}^{{'} {'} } } \\ {\varphi _{1}^{{'} } +\varphi _{2}^{{'} } +\varphi _{3}^{{'} } =1} \\ {\varphi _{1}^{{'} {'} } +\varphi _{2}^{{'} {'} } +\varphi _{3}^{{'} {'} } =1} \end{array}\right. \] 
固定$\varphi _{1} '$的值,则其余5个体积分数都可以确定。度化$\varphi _{1} '$的值可以描出双节线

\noindent 假设体系被半透膜隔开,只允许组分1,2通过。设半透膜外溶液组成$\varphi _{1}^{{\rm bath}} ,\varphi _{2}^{{\rm bath}} $,

\noindent $\varphi _{2}^{{\rm bath}} =\frac{n_{1}^{{\rm bath}} v_{2} }{n_{1}^{{\rm ath}} v_{1} +n_{2}^{{\rm bath}} v_{2} } $,$\varphi _{2}^{bath} =1-\varphi _{1}^{bath} $,$\frac{\partial \varphi _{1}^{bath} }{\partial n_{1}^{bath} } =(1-\varphi _{1}^{bath} )\frac{v_{1} }{V^{bath} } $,$\frac{\partial \varphi _{1}^{bath} }{\partial n_{2} } =\varphi _{1}^{bath} \frac{v_{2} }{V^{bath} } $ 
\[\frac{{\rm \Delta }G_{mix}^{bath} }{RT} =n_{1}^{bath} \ln \varphi _{1}^{bath} +n_{2}^{bath} \ln \varphi _{2}^{bath} +\chi _{12} n_{1}^{bath} \varphi _{2}^{bath} \] 
\[\begin{array}{rcl} {\boxed{\frac{{\rm \Delta }W_{2}^{bath} }{RT} } } & {=} & {\ln \phi _{1}^{b} +\phi _{2}^{b} -s\phi _{2}^{b} -\frac{\partial \chi _{12} }{\partial \phi _{2}^{b} } \phi _{2}^{b} \left(\phi _{2}^{b} \right)^{2} +\chi _{12} \phi _{2}^{b} -\chi _{12} \phi _{1} \phi _{2} } \\ {} & {} & {\boxed{=\ln \phi _{1}^{b} +(1-s)\phi _{2}^{b} +\chi _{12} (\phi _{2}^{b} )^{2} -\frac{\partial \chi _{1} {}_{2} }{\partial \phi _{2}^{b} } \phi _{1}^{b} (\phi _{2}^{b} )^{2} } } \end{array}\] 
\[\begin{array}{c} {\frac{{\rm \Delta }\mu _{2}^{badh} }{RT} =n_{1}^{b} \varphi _{1}^{b-1} \frac{\partial \varphi _{1}^{b} }{\partial n_{2}^{b} } +\ln \varphi _{2}^{b} +n_{2}^{b} \varphi _{3}^{b-1} \frac{\partial \varphi _{2}^{b} }{\partial n_{2}^{b} } +n_{1}^{b} \frac{\partial \chi _{12} }{\partial n_{2}^{b} } \varphi _{2}^{b} +n_{1}^{b} \chi _{12} \frac{\partial \varphi _{2}^{b} }{\partial n_{2}^{b} } } \\ {=\ln \phi _{2}^{b} -s^{-1} \phi _{1}^{b} +\phi _{1}^{b} +s^{-2} \frac{\partial \chi _{12} }{\partial \phi _{2}^{b} } (\phi _{1}^{b} )^{2} \phi _{2}^{b} +s^{-1} \chi _{12} \left(\varphi _{1}^{b} \right)^{2} } \end{array}\] 
\[\boxed{\frac{{\rm s\Delta }\mu _{2}^{b} }{RT} =s\ln \phi _{2}^{b} -(1-s)\phi _{1}^{b} +\chi _{12} (\phi _{1}^{b} )^{2} +\frac{\partial \chi _{12} }{\partial \varphi _{2}^{b} } (\varphi _{1}^{b} )^{2} \phi _{2}^{b} } \] 
这两个化学势通过$\varphi _{1}^{b} =1-\varphi _{2}^{b} $可表达为仅含有$\varphi _{2}^{b} $的表达式,消去$\varphi _{2} $,则${\rm \Delta }\mu _{1}^{b} $与${\rm \Delta }\mu _{2}^{b} $满足一个方程。知道${\rm \Delta }\mu _{1}^{b} $能求${\rm \Delta }\mu _{2}^{b} $。相平衡时,由于
\[\left\{\begin{array}{c} {{\rm \Delta }\mu _{1}^{{'} } ={\rm \Delta }\mu _{1}^{{'} {'} } ={\rm \Delta }\mu _{1}^{b} } \\ {{\rm \Delta }\mu _{2}^{{'} } ={\rm \Delta }\mu _{2}^{n} ={\rm \Delta }\mu _{2}^{b} } \end{array}\right. \] 
因此${\rm \Delta }\mu _{1} '$与${\rm \Delta }\mu _{2} '$之间也互相关联并非独立,这为旋节线分离添加了约束。

\noindent 一些计算时间问题:

\noindent Macromal.Theory Simul.4:449(1995)5:789(1996)

\noindent \eject 

\noindent \textbf{聚合物网络的溶胀平衡}


{\bf  $\chi ={1\mathord{\left/ {\vphantom {1 2}} \right. \kern-\nulldelimiterspace} 2} $$\chi ={1\mathord{\left/ {\vphantom {1 2}} \right. \kern-\nulldelimiterspace} 2} $按照通常的考虑方式,?溶胀前后?的自由能变化$\vartriangle G$写成
\[\vartriangle G=\vartriangle G_{mix} +\vartriangle G_{el} \] }

{\bf 其中$\vartriangle G_{mix} $是无线长线形同种聚合物与溶剂的混合自由能,初态是纯溶剂与纯聚合物熔体。$\vartriangle G_{el} $是聚合物网络形变的自由能,初态是干网络。}


{\bf  按之前的推导,聚合物-溶剂二元体系        1:溶剂;2:聚合物}

\noindent 
{\bf $\frac{{\rm \Delta }G_{mix} }{RT} =n_{1} \ln \phi _{1} +n_{2} \ln \phi _{2} +\chi _{12} n_{1} \phi _{2} $     $n_{i} $:$i$分子的摩尔数}

\noindent 
{\bf 但对于交联聚合物网络,高分子链段不能完全自由运动。Flory吧混合自由能中设$n_{2} $为零。就是假设聚合物的平动熵(translational entropy)与溶剂的相比小到可以忽略。(Flory书P.577式(34))。因此
\[\frac{{\rm \Delta }G_{mix} }{RT} =n_{1} \ln \phi _{1} +\chi _{12} n_{1} \phi _{2} \] }


{\bf  弹性能表达式照旧:}

\noindent 
{\bf ${\rm \Delta }G_{el} =-T{\rm \Delta }S_{el} =\frac{1}{2} n_{ch} RT\left[tr\boldsymbol{\mathrm{B}}-3-\ln \left(\det \boldsymbol{\mathrm{F}}\right)\right]$ Flory的书Ch.Ⅺ Eq.\eqref{GrindEQ__41_}}

\noindent 
{\bf 其中$\boldsymbol{\mathrm{F}}$是形变梯度张量,参考构型是无应力干网络,$\boldsymbol{\mathrm{{\rm B} }}=\boldsymbol{\mathrm{FF}}^{T} $。对于各向同性膨胀,设干凝胶体积$V_{0} $,终态体积$V$,则
\[\boldsymbol{\mathrm{F}}=(\frac{V}{V_{0} } )^{\frac{1}{3} } \boldsymbol{\mathrm{I}}=\phi _{0}^{-\frac{1}{3} } \boldsymbol{\mathrm{I}},\quad \boldsymbol{\mathrm{B}}=\phi _{2}^{-\frac{2}{3} } \boldsymbol{\mathrm{I}},\quad {\rm tr}\boldsymbol{\mathrm{B}}=3\phi _{2}^{-\frac{2}{3} } \quad {\rm det}\boldsymbol{\mathrm{F}}=\phi _{2}^{-1} \] 
\[\therefore \frac{{\rm \Delta }G_{el} }{RT} =\frac{1}{2} n_{ch} \left(3\phi _{2}^{-\frac{2}{3} } -3+\ln \phi _{2} \right)\] 
\[\therefore \frac{\vartriangle G}{RT} =n_{1} \ln \phi _{1} +\chi _{12} n_{1} \phi _{2} +\frac{1}{2} n_{ch} \left(3\phi _{2}^{\frac{2}{3} } -3+\ln \phi _{2} \right)\] }


{\bf  实际常通过测量溶胀凝胶的平衡模量,$G_{e} =\frac{nch}{V} {\rm RT}$,得到}

\noindent 
{\bf $n_{ch} =\frac{G_{e} }{RT} \phi _{2,m}^{-1} $,其中$\phi _{2,m} $是测量平衡模量的样品的体积分数,也是实验测量。}

\noindent 
{\bf 弹性形变的自由能
\[{\rm \Delta }G_{el} =\frac{1}{2} \frac{G_{e} }{\phi _{2,m} } V_{0} \left(3\phi _{2}^{-\frac{2}{3} } -3+\ln \phi _{2} \right)\] }

{\bf 再由$\phi _{2} =\frac{V_{0} }{n_{1} v_{1} +V_{0} } \Rightarrow n_{1} =V_{0} \frac{1-\phi _{2} }{\phi _{2} v_{1} } $
\[{\rm \Delta }G=RTV_{0} \left[\frac{1-\phi _{2} }{\phi _{2} v_{1} } \ln (1-\phi _{2} )+\frac{1-\phi _{2} }{V_{1} } \chi _{12} +\frac{1}{2} \frac{G_{e} }{\phi _{2,m} } \left(3\phi _{2}^{-\frac{2}{3} } -3+\ln \phi _{2} \right)\right]\] }


{\bf  溶剂的化学势变化$\frac{{\rm \Delta }\mu _{1} }{RT} =\frac{\partial {\rm \Delta }G}{RT\partial n_{1} } $。这也就是溶剂在凝胶中的化学势与在纯溶剂中的化学势之差,溶胀平衡时${\rm \Delta }\mu _{1} =0$的,
\[\frac{\partial {\rm \Delta }G}{RT\partial n_{1} } =\ln (1-\phi _{2} )+\phi _{2} +\chi _{12} \phi _{2}^{2} +\frac{n_{ch} }{V_{0} } v_{1} \left(\phi _{2}^{\frac{1}{3} } -\frac{1}{2} \phi _{2} \right)=\frac{{\rm \Delta }\mu _{2} }{RT} \] }

{\bf 该式与Flory书P.578式(38)相同。}

\noindent 
{\bf 由于$\vartriangle G$考虑的初态是干凝胶与纯溶剂,故$\vartriangle \mu _{1} $是溶剂在凝胶中的化学势与纯物质的化学势之差。}

\noindent 
{\bf 利用测得的平衡模量则
\[\frac{{\rm \Delta }\mu _{1} }{RT} =\ln (1-\phi _{2} )+\phi _{2} +\chi _{12} \phi _{2}^{2} +\frac{G_{e} }{\phi _{2,m} } \nu _{1} \left(\phi _{2}^{\frac{1}{3} } -\frac{1}{2} \phi _{2} \right)\] }

{\bf 注意:$n_{ch} {\rm /V}_{{\rm 0}} $是体系的化学结构特性,可通过任何溶胀度的样品测得的$G_{e} $和$\phi _{2,m} $计算,但是本身是常数。$V_{0} $也只是样品裁出来的大小。当且仅当测$G_{e} $的样品处于溶胀平衡(${\rm \Delta }\mu _{1} =_{0} $)时,$\phi _{2,m} =\phi _{2} $ ,此时}

\noindent 
{\bf $\ln (1-\phi _{2} )+\phi _{2} +\chi _{12} \phi _{2}^{2} +\frac{G_{e} }{RT} v_{1} (\phi _{2} {}^{-\frac{2}{3} } -\frac{1}{2} )=0$,用到了$\phi _{2,m} =\phi _{2} $ }

\noindent 
{\bf 通过测量平衡溶胀时的$\phi _{2} $和$G_{e} $可由上式估算表观的$\chi _{12} $。}

\noindent 
{\bf ${n_{{\rm ch}} \mathord{\left/ {\vphantom {n_{{\rm ch}}  V_{0} }} \right. \kern-\nulldelimiterspace} V_{0} } ={\rm G}_{{\rm e}} /\phi _{2,m} $的理由:}

\noindent 
{\bf 上面的讨论中,网络的形变是各向同性膨胀($F=\phi _{2}^{-\frac{1}{3} } I$)。对于一般形变的讨论,详见另一笔记。这里需要提的是,按照Flory-Rehner的考虑方式${\rm \Delta }G={\rm \Delta }G_{mix} +{\rm \Delta }G_{el} $,且$G_{mix} $不依赖形变。因此溶胀网络的平衡模量仍然是橡胶网络机理贡献,$G_{e} =\left({n_{ch} \mathord{\left/ {\vphantom {n_{ch}  V}} \right. \kern-\nulldelimiterspace} V} \right)RT$。许多资料默认了这一点}

\noindent 
{\bf ---------------------------------------------------------------------------------------------------------------}

\noindent 
{\bf 三组分体系仅需改动${\rm \Delta }G_{mix} $。记1:非溶剂;2:溶剂,3:聚合物网络。同理,${\rm \Delta }G_{mix} $在前面的表达式中要设$n_{3} =0$,得到:
\[\frac{{\rm \Delta }G_{mix} }{RT} =n_{1} \ln \phi _{1} +n_{2} \ln \phi _{2} +\chi _{12} n_{2} \phi _{2} +\chi _{13} n_{1} \phi _{3} +\chi _{23} n_{2} \phi _{3} \] }

{\bf $\frac{1}{RT} \frac{\partial {\rm \Delta }G_{mix} }{\partial n_{1} } =\ln \phi _{1} +1-\phi _{1} -s\phi _{2} +(1-\phi _{1} )(\chi _{12} \phi _{2} +\chi _{13} \phi _{3} )-s\chi _{23} \phi _{2} \phi _{3} -u_{1} u_{2} \phi _{2} \frac{\partial \chi _{12} }{\partial u_{2} } $$\mathrm{\sqrt{}}$}

\noindent 
{\bf $\frac{s}{RT} \frac{\partial \Delta G_{mix} }{\partial n_{2} } =s\ln \phi _{2} +s-\phi _{1} -s\phi _{2} +(1-\phi _{2} )\left(\chi _{12} \phi _{1} +s\chi _{23} \phi _{3} \right)-\chi _{13} \phi _{1} \phi _{3} +u_{1} u_{2} \phi _{1} \frac{\partial \chi _{12} }{\partial u_{2} } $$\mathrm{\sqrt{}}$
\[\chi _{12} \chi _{23} \chi _{13} \begin{array}{c} {\frac{{\rm \Delta }G}{RTV_{0} } =\frac{\phi _{1} }{v_{1} \phi _{3} } \ln \phi _{2} +\frac{\phi _{2} }{v_{2} \phi _{3} } \ln \phi _{2} +\chi _{12} \frac{\phi _{1} \phi _{2} }{v_{1} \phi _{3} } +\chi _{13} \frac{\phi _{1} }{v_{1} } +} \\ {X_{23} \frac{\phi _{2} }{v_{2} } +\frac{1}{2} \frac{Ge}{\phi _{3,m} } \left(3\phi _{3}^{-\frac{2}{3} } -3+\ln \phi _{3} \right)} \end{array}\chi _{12} \chi _{23} \chi _{13} \begin{array}{c} {\frac{{\rm \Delta }G}{RTV_{0} } =\frac{\phi _{1} }{v_{1} \phi _{3} } \ln \phi _{2} +\frac{\phi _{2} }{v_{2} \phi _{3} } \ln \phi _{2} +\chi _{12} \frac{\phi _{1} \phi _{2} }{v_{1} \phi _{3} } +\chi _{13} \frac{\phi _{1} }{v_{1} } +} \\ {X_{23} \frac{\phi _{2} }{v_{2} } +\frac{1}{2} \frac{Ge}{\phi _{3,m} } \left(3\phi _{3}^{-\frac{2}{3} } -3+\ln \phi _{3} \right)} \end{array}\frac{{\rm \Delta }G_{el} }{RT} =\frac{1}{2} n_{{\rm ch}} (3\phi _{3}^{-\frac{2}{3} } -3+\ln \phi _{3} )\] 
\[\frac{1}{RT} \frac{\partial \Delta Gel}{\partial n_{1} } =\frac{n_{ch} }{V_{0} } \nu _{1} \left(\varphi _{3}^{\frac{1}{3} } -\frac{1}{2} \varphi _{3} \right)\] 
\[\left. \frac{S}{RT} \frac{\partial {\rm \Delta }Gel}{\partial n_{2} } =\frac{n_{ch} }{V_{0} } v_{1} \left(\begin{array}{c} {\varphi _{3}^{\frac{1}{3} } -\frac{1}{2} \varphi _{3} } \end{array}\right. \right)\] 
\[\begin{array}{l} {\therefore \frac{{\rm \Delta }\mu _{1} }{RT} =\ln \phi _{1} +1-\phi _{1} -s\phi _{2} +(1-\phi _{2} )\left(\chi _{12} \phi _{2} +\chi _{13} \phi _{3} \right)-s\chi _{23} \phi _{2} \phi _{3} -u_{1} u_{2} \phi _{2} \frac{\partial \chi _{12} }{\partial u_{2} } } \\ {+\frac{n_{ch} }{V_{0} } v_{1} \left(\phi _{3}^{\frac{1}{3} } -\frac{1}{2} \phi _{3} \right)} \\ {s\frac{{\rm \Delta }\mu _{2} }{RT} =s\ln \phi _{2} +s-\phi _{1} -s\phi _{2} +(1-\phi _{2} )\left(\chi _{12} \phi _{1} +s\chi _{23} \phi _{3} \right)-\chi _{13} \phi _{1} \phi _{3} +u_{1} u_{2} \phi _{1} \frac{\partial \chi _{12} }{\partial u_{2} } } \\ {+\frac{n_{ch} }{V_{0} } v_{1} \left(\phi _{3}^{\frac{1}{3} } -\frac{1}{2} \phi _{3} \right)} \end{array}\] }

{\bf ${\rm \Delta }G$只用$\phi _{i} $来表出:}

\noindent 
{\bf 由:$\left\{\begin{array}{c} {\phi _{1} =\frac{n_{1} v_{1} }{V} } \\ {\phi _{2} =\frac{n_{2} v_{2} }{V} } \\ {\phi _{3} =V_{0} /V} \\ {\mathop{\sum }\limits_{i=1}^{3} \phi _{i} =1} \end{array}\right. \Rightarrow \begin{array}{c} {n_{1} =\frac{\phi _{1} V_{0} }{\phi _{3} v_{1} } } \\ {n_{2} =\frac{\phi _{2} V_{0} }{\phi _{3} v_{2} } } \end{array}$ }

\noindent 
{\bf 代入原式得:
\[\frac{{\rm \Delta }G_{mix} }{{\rm RT}V_{0} } =\phi _{3}^{-1} \left(\frac{\phi _{1} }{v_{1} } \ln \phi _{1} +\frac{\phi _{2} }{v_{2} } \ln \phi _{2} +\chi _{12} \frac{\phi _{2} }{v_{1} } \phi _{2} +\chi _{13} \frac{\phi _{1} }{v_{1} } \phi _{3} +\chi _{23} \frac{\phi _{2} }{v_{2} } \phi _{3} \right)\] 
\[\frac{{\rm \Delta }G_{el} }{{\rm RT}V_{0} } =\frac{1}{2} \frac{n_{ch} }{V_{0} } (3\phi _{3}^{-\frac{2}{3} } -3+\ln \phi _{3} )\] }

{\bf 另外,$\mu _{1} $与$\mu _{2} $式中要求$\phi _{i} >0$,若$\phi _{i} =0$则变成二元体系。}


{\bf  $\begin{array}{c} {\mu _{1}^{{'} } =\mu _{1}^{{'} {'} } =\mu _{1}^{b} } \\ {\mu _{2}^{{'} } =\mu _{2}^{''} =\mu _{2}^{b} } \end{array}$$\mu _{1}^{b} =F(\mu _{2}^{b} )\therefore \mu _{1}^{{'} } =F(\mu _{2}^{{'} } )$$\mu _{1}^{b} =\mu _{1}^{b} (\phi _{2}^{b} )\quad \mu _{2}^{b} =\mu _{2}^{b} (\phi _{2}^{b} )$$\begin{array}{c} {\mu _{1}^{{'} } =\mu _{1}^{{'} {'} } =\mu _{1}^{b} } \\ {\mu _{2}^{{'} } =\mu _{2}^{''} =\mu _{2}^{b} } \end{array}$$\mu _{1}^{b} =F(\mu _{2}^{b} )\therefore \mu _{1}^{{'} } =F(\mu _{2}^{{'} } )$$\mu _{1}^{b} =\mu _{1}^{b} (\phi _{2}^{b} )\quad \mu _{2}^{b} =\mu _{2}^{b} (\phi _{2}^{b} )$计算相图时,假设相分离形成贫聚合物相和富聚合物相,分别用上标prime和double prime标示,则相平衡时,有
\[\mu _{i}^{{'} } =\mu _{i}^{{'} {'} } =\mu _{i}^{{\rm bath}} \quad ,\quad i=1,2\quad ,\quad {\mathop{\sum }\limits_{i}} \phi _{i}^{{'} } =1,\quad {\mathop{\sum }\limits_{i}} \phi _{i}^{{'} {'} } =1\] }


{\bf  溶剂浴的始态是两种纯物质。终态与凝胶网络平衡。对于溶剂浴,
\[\left. \begin{array}{l} {\frac{{\rm \Delta }\mu _{1}^{ba+h} }{RT} =\ln (1-\phi _{2}^{b} )+(1-s)\varphi _{2}^{b} +\chi _{12} (\phi _{2}^{b} )^{2} -(1-\phi _{2}^{b} )(\phi _{2}^{b} )^{2} \frac{\partial \chi _{12} }{\partial \phi _{2}^{b} } \mathrm{\surd }} \\ {S\frac{{\rm \Delta }\mu _{2}^{bath} }{RT} =s\ln \phi _{2}^{b} -(1-s)(1-\phi _{2}^{b} )+\chi _{12} (1-\phi _{2}^{b} )^{2} +(1-\phi _{2}^{b} )^{2} \phi _{2}^{b} \frac{\partial \chi _{1} {}_{2} }{\partial \phi _{2}^{b} } \mathrm{\surd }} \end{array}\right\}{\mathop{\Rightarrow }\limits^{ˆ\gg \phi _{2}^{b} }} {\rm \Delta }\mu _{1}^{bath} =F({\rm \Delta }\mu _{2}^{bath} )\] }

{\bf 其中$\phi _{2}^{b} $是溶剂浴中的组成。上式实际是给溶胀平衡下的$\mu _{1} $与$\mu _{2} $之间增加了约束关系。因为相平衡时还要求$\mu _{i}^{{'} } =\mu _{i}^{{'} {'} } =\mu _{i}^{bath} ,i=1,2$,并同时确定相应的溶剂浴组成$\phi _{2}^{b} $。}

\noindent 
{\bf ---------------------------------------------------------------------------------------------------------------}


{\bf  与渗透压相关的热力学推导:}

\noindent 
{\bf 设凝胶在温度$T$,外压$p$下与无限多溶剂达到平衡。自由能$G$,熵$S$,体积$V$,?}

\noindent 
{\bf 由于这些状态函数的绝对值是不可知测的,理论讨论的变化是视此时的溶剂分数$n_{1} $与高分子网络完全分开为初态。得出了${\rm \Delta }G=G-G_{0} $}

\noindent 
{\bf 故$d{\rm \Delta }G=dG-dG_{0} $  由热力学基本知识,在初终态分别有
\[dG=-SdT+Vdp+\mu _{1} dn_{1}    dG_{0} =-S_{0} dT+V_{0} dp+\mu _{1}^{0} dn_{1} \] }

{\bf 故有${\rm d\Delta }G=-{\rm \Delta }S{\rm d}T+{\rm \Delta }V{\rm d}p+{\rm \Delta }\mu _{1} {\rm d}n_{1} $}

\noindent 
{\bf Remarks:}


{\bf  常假设混合前后体积不变,${\rm \Delta }V=0$;}


{\bf  恒压过程$dp=0$;}


{\bf  不考虑$n_{2} $是由于聚合物是交联网络不是可自由扩散的分子。}

\noindent 此时,由摩尔体积定义和上述假设有

\noindent ${\rm d}n_{1} =\nu _{1}^{-1} {\rm d}V$,凝胶体积变化仅因得失溶剂。

\noindent 故$d{\rm \Delta }F=-{\rm \Delta }SdT+{\rm \Delta }\mu _{1} \nu _{1}^{-1} dV$

\noindent 又由渗透压定义,视凝胶造成的效应为一种半透膜,等效渗透压就是
\[{\rm \Pi }=-\nu _{1}^{-1} {\rm \Delta }\mu _{1} \] 
故${\rm d\Delta }G=-{\rm \Delta }S{\rm d}T-{\rm \Pi d}V$ 其中$\Pi $是来自含?$\Delta $?的一个量。在这里,形式上很像一个?压强?。

\noindent 我们仿造原本的等温压缩系数定义

\noindent $\left. \kappa =-V^{-1} \frac{\partial V}{\partial \Pi } \right|_{T} $,$K\equiv \kappa ^{-1} =\left. -V\frac{\partial \Pi }{\partial V} \right|_{T} $ 例如:Flory------Rehner $\Pi =-V_{1}^{-1} {\rm \Delta }U_{1} $ $\left. \frac{\Pi }{RT} =-\nu _{1}^{-1} \ln (1-\phi _{2} )+\phi _{2} +\chi _{12} \phi _{2}^{2} +\frac{n_{ch} }{V} \nu _{1} \left(\phi _{2}^{\frac{1}{2} } -\frac{1}{2} \phi _{2} \right)\right]$ $\begin{array}{c} {V_{e} =15.2907mol\cdot m^{-3} } \\ {V_{1} =1.8\times 10^{-5} m^{3} \cdot mol^{-1} } \end{array}$

\noindent 由$d{\rm \Delta }F$表达式有$\Pi =-\left. \frac{\partial {\rm \Delta }F}{\partial V} \right|_{T} $

\noindent 所谓体积相变,就是视体积为序参量。这与平时讨论液液相变时以组分为序参量其实等价。

\noindent ${\rm \Delta }FvsV$曲线,在某些条件下,是否有亚稳态?于是可画出相应的体积分相。但这种分相在空间尺度具体如何发生。是难以(无法)在总体考虑方式下的热力学推导中考虑。必须采用场论。


{\bf  Ginzberg -Landau Fru energy of inhomogeneous gels}

\noindent 
{\bf 可见Adu.Polym,Sci,109:63(1993)}

\noindent \eject 

\noindent \textbf{凝胶的相平衡问题}


{\bf  聚合物网络---溶剂 二元体系溶胀平衡的相关热力学函数:
\[\frac{{\rm \Delta }G}{RT} =n_{1} \ln \phi _{1} +\chi _{12} n_{1} \phi _{2} +\frac{1}{2} n_{ch} (3\phi _{2}^{-\frac{2}{3} } -3+\ln \phi _{2} )\] }

{\bf 由$n_{1} =\frac{1-\phi _{2} }{\phi _{2} v_{2} } V_{0} $代入上式得
\[\frac{{\rm \Delta }G}{RTV_{0} } =\frac{1-\phi _{2} }{\phi _{2} v_{1} } \ln (1-\phi _{2} )+\chi _{12} \frac{1-\phi _{2} }{v_{1} } +\frac{1}{2} \frac{n_{ch} }{V_{0} } \left(3\phi _{2}^{-\frac{2}{3} } -3+\ln \phi _{2} \right)\] 
\[\frac{{\rm \Delta }\mu _{1} }{RT} =\ln (1-\phi _{2} )+\phi _{2} +\chi _{12} \phi _{2}^{2} +\frac{n_{{\rm ch}} }{V_{0} } v_{1} \left(\phi _{2}^{\frac{1}{3} } -\frac{1}{2} \phi _{2} \right)\] 
\[\Pi =-v_{1}^{-1} {\rm \Delta }\mu _{1} =\frac{n_{ch} }{V_{0} } \left(\frac{1}{2} \phi _{2} -\phi _{2}^{\frac{1}{3} } \right)-\nu _{1}^{-1} \left[\ln (1-\phi _{2} )+\phi _{2} +\chi _{12} \phi _{2}^{2} \right]\] }


{\bf  以$\phi _{2} $为自变量考虑平衡态稳定性判据:$\Pi =0$}

\noindent 
{\bf 考察不同$\chi _{12} $的$\Pi vs\phi _{2} $曲线:}

\noindent 
{\bf \includegraphics*[width=1.97in, height=1.58in]{image6}\includegraphics*[width=1.89in, height=1.51in]{image7}}

\noindent 
{\bf $\Pi =0$总是唯一解。论文中的情况:}

\noindent 
{\bf 这不是原始$F-R$可预测的需要加上离子项$\Pi _{ion} $。见Adv.Polym Sci. 109 :1(1993)Fig.5 实验结果Fig.6 NIPA是离子化的。}

\noindent 
{\bf 至少,对于原始$F-R$行为,体系在$\chi =0.5$附近很窄的范围内,将发生很大的体积变化。}


{\bf  Flory-Huggins相互作用参数本身的理论}

\noindent 
{\bf $x=\frac{{\rm \Delta }G}{k_{B} T} =\frac{{\rm \Delta }H-T{\rm \Delta }S}{k_{B} T} =\frac{1}{2} -A(1-\frac{\Theta }{T} )$ UCST和LCST看A}

\noindent 
{\bf $A=\frac{2{\rm \Delta }S+kB}{2k_{B} } $ $\Theta =\frac{2\Delta H}{2\Delta S+k_{B} } $A与熵贡献有关,$\Theta $是$\theta $温度}

\noindent 
{\bf 因此,按$F-R$,在很窄的范围内发生溶胀度的较大变化,但是仍是连续变化的。但在有电离的情况下则发生体积相变(温度)。}


{\bf  相平衡:溶胀平衡时:}

\noindent 
{\bf $\mu _{1}^{{\rm gel}} =\mu _{1}^{*} $ 又$\because {\rm \Delta }\mu _{1}^{gel} =\mu _{1}^{gel} -\mu _{1}^{*} $故${\rm \Delta }\mu _{1}^{gel} =0$与上述讨论直接衔接。}

\noindent 
{\bf 更多研究总结可见Phys.Rev,E75:011801(2007), Soft Matter 13:8271}


{\bf  聚合物网络---溶剂1---溶剂2体系}

\noindent 
{\bf 由$n_{1} =\frac{\phi _{1} V_{0} }{\phi _{3} v_{1} } $,$n_{2} =\frac{\phi _{2} V_{0} }{\phi _{3} v_{2} } $
\[\begin{array}{c} {\frac{{\rm \Delta }G}{RTV_{0} } =\frac{\phi _{1} }{\phi _{3} v_{1} } \ln \phi _{1} +\frac{\phi _{2} }{\phi _{3} v_{2} } \ln \phi _{2} +\chi _{12} \frac{\phi _{1} \phi _{2} }{\phi _{3} v_{1} } +\chi _{13} \frac{\phi _{3} }{v_{1} } +\chi _{23} \frac{\phi _{3} }{v_{2} } } \\ {+\frac{1}{2} \frac{n_{ch} }{V_{0} } \left(3\phi _{3}^{-\frac{2}{3} } -3+\ln \phi _{3} \right)} \end{array}\] 
\[\begin{array}{rcl} {d{\rm \Delta }G} & {=} & {-{\rm \Delta }SdT+{\rm \Delta }\mu _{1} dn_{1} +{\rm \Delta }\mu _{2} dn_{2} \quad dn_{1} =\nu _{1}^{-1} dV,dn_{2} =\nu _{2}^{-1} dV} \\ {} & {=} & {-{\rm \Delta }SdT+\left({\rm \Delta }\mu _{1} v_{1}^{-1} +\Delta \mu _{2} v_{2}^{-1} \right)dV=-{\rm \Delta }SdT-\Pi dV} \end{array}\] 
\[\Rightarrow \quad \Pi =-\left({\rm \Delta }\mu _{1} v_{1}^{-1} +{\rm \Delta }\mu _{2} v_{2}^{-1} \right)\] }

{\bf  当聚合物网络与溶剂浴达到平衡,应有:}

\noindent 
{\bf $\left\{\begin{array}{c} {{\rm \Delta }u_{1}^{g} ={\rm \Delta }u_{1}^{b} } \\ {{\rm \Delta }u_{2}^{g} ={\rm \Delta }u_{2}^{b} } \\ {{\rm \Delta }u_{1}^{b} =F({\rm \Delta }u_{2}^{b} )} \end{array}\right. $ $F(x)$的含义见上一个笔记$\Rightarrow \left\{\begin{array}{c} {{\rm \Delta }\mu _{1}^{gel} =F({\rm \Delta }\mu _{2}^{gel} )} \\ {{\rm \Delta }\mu _{2}^{gel} ={\rm \Delta }\mu _{2}^{bath} } \end{array}\right. $ }

\noindent 
{\bf $\Rightarrow \left\{\begin{array}{c} {{\rm \Delta }\mu _{1}^{g} =F({\rm \Delta }\mu _{2}^{g} )} \\ {{\rm \Delta }\mu _{2}^{g} ={\rm \Delta }\mu _{2}^{b} } \end{array}\right. $ 对于给定的溶剂浴组成$\phi _{2}^{b} $,可得到确定的一组$(\phi _{1}^{g} ,\phi _{2}^{g} )$可计算凝胶溶胀对溶剂的选择性。由$\phi _{3}^{g} =1-\phi _{1}^{g} -\phi _{2}^{g} $可得溶胀度。}

\noindent 
{\bf ---------------------------------------------------------------------------------------------------------------}


{\bf  文献调查}


{\bf  用Flory-Rechner形式,设定两相${\rm \Delta }\mu _{1}^{{'} } ={\rm \Delta }\mu _{2}^{{'} {'} } $找共存组分的方法,至少Dusek在1969年起用过J.Polym.Sci, Polym.Phys,Ed,6:1209(1968)(P+S二元)}

\noindent 
{\bf 且之后称为gel-gel demixing M28:1103(1995),31:2223(1998).}


{\bf  为了有更靠谱的非常数$\chi _{12} $关系,通过线形聚合物实验确定用于F-R预测网络总溶胀,有成功例子:}

\noindent 
{\bf Y.C.Bae等:}

\noindent 
{\bf Polymer 54:6776(2013)(P/S1/S2以下皆是)}

\noindent 
{\bf Okay:纯经验的argument:Macromol.Chem.Phys.202:304 (2001) Polymer47:561(2006)}

\noindent 
{\bf Maurer:UNIQUAC形式:Fluid Phase Equilibr.219:231(2004) 238:87(2005)}


{\bf  给$\Delta G$加新项}

\noindent 
{\bf 类似UNIQUAC的修改:Chem.Eng.Sci.65:3223(2010)}

\noindent 
{\bf 更多的考虑包括?compressible lattice敂free volume as 3${}^{rd}$ presudo component?}

\noindent 
{\bf ?H-bonding?(Lattice fluid hydrogen bond,LFHB)}

\noindent 
{\bf 还有?hydrated?M43:5103(2010)}

\noindent \eject 

\noindent \textbf{取特定末端距约束下的正则分配函数}


{\bf  自由单链在正则系综条件下按玻尔兹曼分布的概率涨落,$\vec{R}$也是涨落的,}


{\bf  如果约束一根链取某固定末端距$\vec{R}$,则与自由条件相比,体系只能取部分构象,即满足$\sum _{j=1}^{n}\vec{r}_{j} =\vec{R} $的构象。在正则系综平衡态下,配分函数
\[Z_{\vec{R}} =\int _{\Gamma _{\vec{R}} }d\left\{\overrightarrow{r_{j} },\overrightarrow{p_{j} }\right\}\exp \left(-\frac{\mathrm{{\mathcal H}}\left(\{ \overrightarrow{p_{j} },\overrightarrow{r_{j} }\} \right)}{{\rm k}_{B} {\rm T}} \right) \] }

{\bf 其中$\Gamma _{\vec{R}} $是所有满足${\rm \Sigma }_{j} \overrightarrow{r_{j} }=\vec{R}$的部分$\{ \overrightarrow{p_{j} },\overrightarrow{r_{j} }\} $}

\noindent 
{\bf 利用$\delta $函数上式
\[\Leftrightarrow Z_{\vec{R}} ={\mathop{\smallint }\nolimits_{\Gamma }} d\left\{\vec{r}_{j} ,\vec{p}_{j} \right\}\exp \left(-\frac{\mathrm{{\mathcal H}}\left(\left. \left\{\overrightarrow{p_{j} },\overrightarrow{r_{j} }\right\}\right)\right. }{k_{B} T} \right)\delta \left(\vec{R}-{\rm \Sigma }_{j} \vec{r}_{j} \right)\] }

{\bf 其中$\Gamma $是所有$\{ \overrightarrow{p_{j} },\overrightarrow{r_{j} }\} $,依照自由结合链正则分配函数的推导,可达到
\[Z_{\vec{R}} =Z_{0} {\mathop{\smallint }\nolimits_{\{ \vec{r}_{j} \} }} d\vec{r}_{1} \cdots d\vec{r}_{n} \exp \left[-\frac{1}{k_{B} T} \sum _{j}u(\parallel \vec{r}_{j} \parallel ) \right]\delta (\vec{R}-{\mathop{\sum }\limits_{k}} \vec{r}_{k} )\] }

{\bf 其中$Z_{0} =(2\pi mk_{B} T)^{3n/2} $。自由连结链的势函数$u(\parallel \vec{r}_{j} \parallel )$满足, }

\noindent 
{\bf $\exp \left[-\frac{1}{k_{B} T} u(\parallel \overrightarrow{r_{j} }\parallel )\right]=\sqrt{\frac{2\pi k_{B} T}{\kappa } } 4\pi b^{2} \psi (\parallel \overrightarrow{r_{j} }\parallel )$,其中$\psi (\parallel \vec{{\rm r}}\parallel )=\frac{2}{4\pi b^{2} } \delta (\parallel \vec{{\rm r}}\parallel -b)$}

\noindent 
{\bf 原式$\begin{array}{cc} {\Leftrightarrow } & {\begin{array}{rcl} {Z{}_{\vec{R}} } & {=} & {Z_{0} {\mathop{\smallint }\nolimits_{\mathrm{{\mathbb R}}^{3n} }} d\vec{r}_{1} \cdots d\vec{r}_{n} \quad (\frac{2\pi k_{B} T}{\kappa } )^{\frac{n}{2} } \delta (||\vec{r}_{1} ||-b)\cdots \delta (||\vec{r}_{n} ||-b)d^{5} (\vec{R}-\sum _{k}\vec{r}_{k}  )} \\ {} & {=} & {Z_{0} \left(\frac{2\pi /k_{B} T}{\kappa } \right)^{\frac{n}{2} } (2\pi )^{-3} {\mathop{\smallint }\nolimits_{\mathrm{{\mathbb R}}^{(3n+1)} }} d\vec{r}_{1} -d\vec{r}_{n} d\vec{k}\cdot \delta (\parallel \vec{r}_{1} \parallel -b)\cdots \delta (\parallel \vec{r}_{n} \parallel -b)\exp (i\vec{k}\cdot \vec{R})} \\ {} & {} & {\exp (-i\vec{k}\cdot \overrightarrow{r_{1} })\cdots \exp (-i\vec{k}\cdot \overrightarrow{r_{n} })} \\ {} & {=} & {Z_{0} \left(\frac{2\pi k_{B} T}{\kappa } \right)^{\frac{n}{2} } \left(2\pi \right)^{-3} {\mathop{\smallint }\nolimits_{\mathrm{{\mathbb R}}^{3} }} d\vec{k}\exp \left(-i\vec{k}\cdot \vec{R}\right)\left[{\mathop{\smallint }\nolimits_{R^{2n} }} d\vec{r}\delta \left(||\vec{r}||-b\right)\exp \left(-i\vec{k}\cdot \vec{r}\right)\right]^{n} } \end{array}} \end{array}$ }

\noindent 
{\bf 之前推导过,
\[{\mathop{\smallint }\nolimits_{\mathrm{{\mathbb R}}^{3} }} d\vec{r}\delta (||\vec{r}||-b)\exp (-i\vec{k}\cdot \vec{r})=4\pi b^{2} \frac{\sin (kb)}{kb} \approx 4\pi b^{2} \exp (-\frac{1}{6} k^{2} b^{2} ),kb\ll 1\] 
\[\begin{array}{cc} {\therefore } & {\begin{array}{rcl} {Z{}_{\vec{r}} } & {=} & {Z_{0} \left(\frac{2\pi k_{B} T}{\kappa } \right)^{\frac{n}{2} } (2\pi )^{-3} \left(4\pi b^{2} \right)^{n} {\mathop{\smallint }\nolimits_{\mathrm{{\mathbb R}}^{3} }} d\vec{k}\exp (-i\vec{k}\cdot \vec{R})\exp \left(-\frac{1}{6} k^{2} b^{2} \right),kb\ll 1} \\ {} & {=} & {Z_{0} \left(\frac{2\pi k_{B} T}{k_{5} } \right)^{\frac{n}{2} } (2\pi )^{-3} \left(4\pi b^{2} \right)^{n} \left(\frac{6\pi }{nb^{2} } \right)^{\frac{3}{2} } \exp \left(-\frac{2(nb)^{2} }{2nb^{2} } \right)\quad ,\quad kb\ll 1} \\ {} & {=} & {Z_{0} Z_{1} \left(\frac{3}{2\pi nb^{2} } \right)^{\frac{3}{2} } \exp \left(-\frac{3\parallel \vec{R}\parallel ^{2} }{2nb^{2} } \right),kb\ll 1} \end{array}} \end{array}\] }

{\bf 其中$Z_{1} =(4\pi b^{2} )^{n} (\frac{2\pi k_{B} T}{\kappa } )^{\frac{n}{2} } $,且之前推导过,自由连结链自由状态的正则配分函数$Z=Z_{0} Z_{1} $ }

\noindent 
{\bf 自由连结链末端据向量取$\vec{R}$的概率密度${\rm \Phi }(\vec{R})=(\frac{3}{2\pi nb^{2} } )^{\frac{3}{2} } \exp (-\frac{3||\vec{R}||^{2} }{2nb^{2} } )$,故
\[Z_{\vec{R}} =Z\phi (\vec{R})\] 
\[\ln Z_{\vec{R}} =\ln Z+\ln \Phi \left(\vec{R}\right)=\ln Z+\ln \left[(\frac{3}{2\pi nb^{2} } )^{\frac{3}{2} } \right]-\frac{3||\vec{R}||^{2} }{2nb^{2} } \] }
\eject 

\noindent \textbf{热机械行为}

\noindent 近年发现凝胶的热机械行为与橡胶相比有很多异常情况,我们把$\chi _{12} $推广至依赖体积分数$\varphi _{2} $,温度$T$与拉伸比$\lambda $的形式。此时$\chi _{12} =\chi _{12} (\varphi _{2} ,T,\lambda )$,此时,
\[\frac{{\rm \Delta }G}{RT} =n_{1} \ln \varphi _{1} +\chi _{12} \left(\varphi _{2} ,T,\lambda \right)n_{1} \varphi _{2} +\frac{1}{2} n_{ch} \left[\varphi _{2}^{-\frac{2}{3} } (\lambda ^{2} +2\lambda ^{-2} )+\ln \varphi _{2} -3\right]\] 
\[\begin{array}{rcl} {F} & {=} & {\frac{\partial {\rm \Delta G}}{\partial L} =\frac{\partial {\rm \Delta G}}{\partial \lambda } \varphi _{2}^{\frac{1}{3} } L_{0}^{-1} ,\quad L_{0} /rQÜ| } \\ {} & {=} & {RT\left[n_{2} \varphi _{2} \cdot \frac{\partial \chi _{12} }{\partial \lambda } +n_{ch} \cdot \varphi _{2}^{-\frac{2}{3} } \left(\lambda -\lambda ^{-2} \right)\right]\varphi _{2}^{\frac{2}{3} } L_{0}^{-1} } \\ {} & {=} & {RT\left[n_{1} \varphi _{2}^{\frac{4}{3} } L_{0}^{-1} \frac{\partial \chi _{12} }{\partial \lambda } +n_{ch} \varphi _{2}^{\frac{1}{3} } L_{0}^{-1} (\lambda -\lambda ^{-2} )\right]} \\ {} & {=} & {RTA_{0} \varphi _{2}^{\frac{1}{3} } \left[v^{-1} (1-\varphi _{2} )\frac{\partial \chi _{12} }{\partial \lambda } +\frac{n_{ch} }{V_{0} } \varphi _{2}^{-\frac{2}{3} } (\lambda -\lambda ^{2} )\right]} \end{array}\] 


\noindent $\chi _{12} $的温度依赖常写成?焓贡献?和?熵贡献?
\[\chi _{12} =\chi _{12}^{h} +\chi _{12}^{S} ,\] 
$\chi _{12} $现在定义为1对A-A B-B变成一堆A-B的自由能变的一半。即
\[\chi _{12} =\frac{{\rm \Delta }G_{A-B} }{RT} =\frac{{\rm \Delta }H_{A-B} -T\Delta S_{A-B} }{RT} =\frac{1}{2} -A\left(1-\frac{\Theta }{T} \right),\quad -\frac{{\rm \Delta }S_{A-B} }{R} =\frac{1}{2} -A,\quad \frac{{\rm \Delta }H_{A-B} }{RT} =\frac{A\Theta }{T} \] 
因此我们让参数$A$和$\Theta $都依赖$\phi _{2} $和$\lambda $,即$\chi _{12} (T,\varphi _{2} ,\lambda )=\frac{1}{2} -A(\varphi _{2} ,\lambda )(1-\frac{\Theta (\varphi _{2} ,\lambda )}{T} )$,则有
\[\frac{\partial \chi _{12} }{\partial \lambda } +T\frac{\partial ^{2} \chi _{12} }{\partial T\partial \lambda } =-\frac{\partial A}{\partial \lambda } \] 
$\begin{array}{rcl} {T\frac{\partial F}{\partial T} } & {=} & {RT[n_{1} \varphi _{2}^{\frac{4}{3} } L_{0}^{-1} (-\frac{\partial A}{\partial \lambda } )+n_{{\rm ch}} \varphi _{2}^{-\frac{1}{3} } L_{0}^{-1} (\lambda -\lambda ^{-2} )]} \\ {} & {=} & {RTA_{0} \varphi _{2}^{\frac{1}{3} } \left[\boxed{\frac{n_{ch} }{V_{0} } (\lambda -\lambda ^{-2} )} -\boxed{v_{1}^{-1} (2-\varphi _{2} )\frac{\partial A}{\partial \lambda } } \right]\; } \end{array}$ 其中$A_{0} $是干网络截面$A_{0} =V_{0} /L_{0} $ 

\noindent                  弹性熵贡献  拉伸依赖的混合熵贡献  $-{\partial A\mathord{\left/ {\vphantom {\partial A \partial }} \right. \kern-\nulldelimiterspace} \partial } \lambda =\frac{\partial x_{s} }{\partial \lambda } $ 
\[F-T\frac{\partial F}{\partial T} =RTA_{0} \varphi _{2} {}^{\frac{1}{3} } \left[v_{1} {}^{-1} (1-\varphi _{2} )T^{-1} \frac{\partial \boxed{\left[{\rm \Theta }(\varphi _{2} ,\lambda )A(\varphi _{2} ,\lambda )\right]} \equiv \frac{\partial x_{n} }{\partial \lambda } }{\partial \lambda } +\frac{n_{ch} }{V_{0} } (\lambda -\lambda ^{-2} )(\varphi _{2}^{-\frac{2}{3} } -1)\right]\] 
由$A_{0} =A_{\lambda } \lambda \varphi _{2}^{\frac{2}{3} } $可把系数$A_{0} $换成$A_{\lambda } $,得到真应力。$A_{\lambda } $是溶胀拉伸比$\lambda $时的截面积。

\noindent $A(\varphi _{2} ,\lambda )$是直接与$\chi _{12} $的熵贡献相关的。$\chi _{12} $的熵贡献分子机理有不少说法:

\noindent excess free energy : M26:5587(1993)24:\underbar{5076}(1991)24:5096(1991)24:5112(1991)

\noindent monomer structure,M34:1751(2001) 31:6681(1998) 33:3467(2000)

\noindent chain flexibility: J.Chem.Phys.114:5016 (2001)

\noindent chain-end effect! M34:1096(2001) 26:213(1993)

\noindent Karl Freed的Lattice cluster model是FH模型的主要修改,但后人作过各种简化。在Fluid Phase Equilibria 427:594(2016)中有简短回顾。

\noindent 

\noindent 上述推导考虑的是理想橡胶网络仅因容积效应而造成的热机械异常。在$\varphi _{2} =1$时退回至经典橡胶弹性。

\noindent \eject 

\noindent \textbf{溶剂热力学1:化学势的引入}

\noindent Ⅰ.化学势的引入


{\bf  多组分体系的内能,在孤立时,是$(S,V,\{ n_{i} \} )$的函数$U=U(S,V,\{ n_{i} \} )$
\[du=\left. \frac{\partial U}{\partial S} \right|_{V,\{ n_{i} \} } dS+\left. \frac{\partial U}{\partial V} \right|_{S,\{ n_{i} \} } dV+\sum _{i}\left. \frac{\partial U}{\partial n_{i} } \right|_{S,V,\{ n_{j\ne i} \} } dn_{i}  \] }

{\bf 与第一、二定量比较得(准静态过程)$dU=TdS-pdV+\mu _{i} dn_{i} $
\[T=\left. \frac{\partial U}{\partial S} \right|_{V,\{ ni\} } ,-p=\left. \frac{\partial U}{\partial V} \right|_{S,\{ ni\} } ,\quad \mu _{i} \equiv \left. \frac{\partial U}{\partial n_{i} } \right|_{S,V,\{ n_{j\ne i} \} } \] }

{\bf $U_{i} $是由于组分i分子数变化$dn_{i} $造成的内能变化,引入为化学势。}


{\bf  常考虑体系平衡态由$(T,p,\{ n_{i} \} )$ 决定的情况。此时热力学势是吉布斯自由能
\[\begin{array}{cc} {G{\mathop{=}\limits^{def}} U-TS+pV} & {\begin{array}{rcl} {dG} & {=} & {dU-TdS-SdT+pdV+VdP} \\ {} & {=} & {TdS-pdV+\mu _{i} dn_{i} -TdS-SdT+pdV+Vdp} \\ {} & {=} & {-SdT+Vdp+\mu _{i} dn_{i} } \\ {} & {=} & {\left. \frac{\partial G}{\partial T} \right|_{p,\{ n_{i} \} } dT+\left. \frac{\partial G}{\partial p} \right|_{T,\{ n_{i} \} } dp+\left. \frac{\partial G}{\partial n_{i} } \right|_{T,p,\{ n_{j\ne i} \} } dni} \end{array}} \end{array}\] 
\[\quad \Rightarrow \quad -S=\left. \frac{\partial G}{\partial T} \right|_{p,\{ n_{i} \} } ,\quad V=\left. \frac{\partial G}{\partial p} \right|_{T,\{ n_{i} \} } ,\quad \mu _{i} =\left. \frac{\partial G}{\partial n_{i} } \right|_{T,p,\{ n_{j\ne i} \} } \] }

{\bf Maxwell关系:
\[\left. \left. \left. \frac{\partial S}{\partial p} \right|_{T,\{ n_{i} \} } =\left. -\frac{\partial }{\partial p} {\rm (}\left. \frac{\partial G}{\partial T} \right|_{p,\{ n_{i} \} } \right)\right|_{T,\{ n_{i} \} } =\left. -\frac{\partial }{\partial T} {\rm (}\left. \frac{\partial G}{\partial p} \right|_{T,\{ n_{i} \} } \right)\right|_{p,\{ n_{i} \} } =\left. -\frac{\partial V}{\partial T} \right|_{p,\{ n_{i} \} } \] 
\[\frac{\partial S}{\partial n_{i} } \left|{}_{T,p,\left\{n_{j\ne i} \right\}} =-\frac{\partial }{\partial n_{i} } ,\left. {\rm (}\left. \frac{\partial G}{\partial T} \right|_{p,\left\{n_{i} \right\}} \right)\right|_{T,p,\left\{n_{j\ne i} \right\}} =-\left. \frac{\partial G}{\partial T} \left(\left. \frac{\partial G}{\partial n_{i} } \right|_{T,p,\left\{n_{j\ne i} \right\}} \right)\right|_{p,\left\{n_{i} \right\}} =-\left. \frac{\partial \mu _{i} }{\partial T} \right|_{p,\left\{n_{i} \right\}} \] 
\[\left. \left. \left. \frac{\partial V}{\partial n_{i} } \right|_{T,p,\{ n_{j\ne i} \} } =\frac{\partial }{\partial n_{i} } \left(\left. \frac{\partial G}{\partial p} \right|_{T,\{ n_{i} \} } \right)\right|_{T,p,\{ n_{i\ne j} \} } =\frac{\partial }{\partial p} \left(\left. \frac{\partial G}{\partial n_{i} } \right|_{T,p,\{ n_{j\ne i} \} } \right)\right|_{T,\{ n_{i} \} } =\left. \frac{\partial \mu _{i} }{\partial p} \right|_{T,\{ n_{i} \} } \] 
\[\left. \frac{\partial \mu _{i} }{\partial n_{j} } \right|_{T,p,\{ n_{k\ne j} \} } =\frac{\partial }{\partial n_{j} } \left. \left(\left. \frac{\partial G}{\partial n_{i} } \right|_{T,p,\{ n_{l\ne i} \} } \right)\right|_{T,p,\{ n_{k\ne j} \} } =\frac{\partial }{\partial n_{i} } \left. \left(\left. \frac{\partial G}{\partial n_{j} } \right|_{T,p,\{ n_{k\ne j} \} } \right)\right|_{T,p,\{ n_{l\ne i} \} } =\left. \frac{\partial \mu _{j} }{\partial n_{i} } \right|_{T,p,\{ n_{k\ne j} \} } \] }
Ⅱ.偏摩尔量

\noindent 上述Maxwell关系中的

\noindent $\left. \frac{\partial V}{\partial n_{i} } \right|_{T,p,\{ n_{j\ne i} \} } {\mathop{=}\limits^{def}} v_{i} $  偏摩尔体积

\noindent $\left. \frac{\partial S}{\partial n_{i} } \right|_{T,p,\{ n_{j\ne i} \} } {\mathop{=}\limits^{def}} S_{i} $  偏摩尔熵


{\bf  更多偏摩尔量可被类似地定义,例如}

\noindent 
{\bf $g_{i} {\mathop{=}\limits^{def}} \left. \frac{\partial G}{\partial ni} \right|_{p,T,n_{j\ne i} } $,偏摩尔Gibbs自由能}

\noindent 
{\bf 一般地,某688185024可逆688185024688185024129735893广度?状态量$Y=Y\left(T,p,\{ n_{i} \} \right)$的全微分是
\[\begin{array}{rcl} {dY} & {=} & {\left. \frac{\partial Y}{\partial T} \right|_{p,\{ n_{i} \} } dT+\left. \frac{\partial Y}{\partial P} \right|_{T,\{ n_{i} \} } dp+\sum _{i}\left. \frac{\partial Y}{\partial n_{i} } \right| _{T,p,\{ n_{j\ne i} \} } dn_{i} } \\ {} & {=} & {\left. \frac{\partial Y}{\partial T} \right|_{\begin{array}{c} {p,\{ n_{i} \} } \end{array}} dT+\left. \frac{\partial Y}{\partial p} \right|_{\begin{array}{c} {T,\{ n_{i} \} } \end{array}} dp+{\mathop{\sum }\limits_{i}} y_{i} dn_{i} } \end{array}\] }

{\bf 其中$Y$的偏摩尔量$y_{i} {\mathop{=}\limits^{def}} \left. \frac{\partial Y}{\partial n_{i} } \right|_{T,p,\{ n_{j\ne i} \} } $}

\noindent 
{\bf 注意到偏摩尔量$y_{i} $是强度量,在恒定$(T,p)$下,$y_{i} $是组分$\left\{n_{i} \right\}$的函数$y_{i} =y_{i} (\{ n_{i} \} )$}


{\bf  下面进一步推导:?$Y=\sum _{i}y_{i} n_{i}  $?记$x_{i} \equiv n_{i} /{\rm \Sigma }_{i} n_{i} $为组分$i$的摩尔份数,$n={\rm \Sigma }_{i} n_{i} $为总摩尔数。则在恒定$(T,p)$下$(dT=dp=0)$}

\noindent 
{\bf $dY=(\sum _{i}y_{i} x_{i}  )dn$,用到了$dn_{i} =x_{i} dn$}

\noindent 
{\bf 积分$\Rightarrow Y=({\mathop{\sum }\limits_{i}} y_{i} x_{i} )n+C$,$C$是积分常数,再由当$n=0$时$Y=0$的规定,得$C=0$,因此有$Y={\mathop{\sum }\limits_{i}} n_{i} y_{i} (RšT,p)$}

\noindent 
{\bf 即当已知$\{ n_{i} \} $时,$y_{i} =y_{i} (\{ n_{i} \} )$也都已知,$Y$与$y_{i} $满足上式。特别地,恒定$(T,p)$下,}

\noindent 
{\bf $G=\sum _{i}n_{i} q_{i}  =\sum _{i}n_{i} \mu _{i}  $(因此条件下恰有$\mu _{i} =g_{i} $)}

\noindent 
{\bf 由于对吉布斯自由能,偏摩尔量在恒定$(T,p)$下恰好是化学势,故进一步有
\[\begin{array}{c} {dG={\mathop{\sum }\limits_{i}} \mu _{i} dn_{i} \quad (G=G(T,p,\{ n_{i} \} ),dT=dp=0)} \\ {={\mathop{\sum }\limits_{i}} n_{i} d\mu _{i} +{\mathop{\sum }\limits_{i}} \mu _{i} dn_{i} \quad (OiÏ)} \end{array}\] }

{\bf $\begin{array}{l} {\Rightarrow {\mathop{\sum }\limits_{i}} n_{i} d\mu _{i} =0\Rightarrow {\mathop{\sum }\limits_{i}} x_{i} d\mu _{i} =0} \\ {{\rm \; \; \; \; \; \; \; \; \; \; \; \; \; \; \; \; \; \; \; \; }dn} \end{array}$ 即吉布斯------杜亥姆方程的混合物版。}


{\bf  实验上常只测得混合物体系的摩尔量随组分的变化,如何由这个数据求得各组分的偏摩尔量?}

\noindent 
{\bf 摩尔量$i$ 恒定$(T,p)$下,体系的广度量$Y$的摩尔量$Y_{m} {\mathop{=}\limits^{def}} Y/n$,其中$n\equiv {\rm \Sigma }_{i} n_{i} $}

\noindent 
{\bf 以双组份体系为例,恒定$(T,p)$下$(dT=dp=0){\rm \; \; \; }Y_{m} =Y_{m} (n_{1} ,n_{2} )$ 
\[\begin{array}{rcl} {{\rm d}Y{}_{m} } & {=} & {{\rm d}Y/n={\rm d}(n_{1} y_{1} +n_{2} y_{2} )/n={\rm d}(x_{1} y_{1} +x_{2} y_{2} )} \\ {} & {=} & {\boxed{x_{1} {\rm d}y_{2} +x_{2} {\rm d}y_{2} } +y_{1} {\rm d}x_{1} +y_{2} {\rm d}x_{2} } \\ {{\rm \; \; \; }} & {\Rightarrow } & {\equiv 0¨„{\rm Gibbs-Duhem}¹} \end{array}\] }

{\bf $\Leftrightarrow dY_{m} =y_{1} dx_{1} +y_{2} dx_{2} $,恒定$\left(p,T\right)$}

\noindent 
{\bf $\Rightarrow \left. \frac{\partial Y_{m} }{\partial x_{2} } \right|_{T,p} =-y_{1} +y_{2} $(用到了$dx_{1} \equiv -dx_{2} {\rm ,\; \; }\because x_{2} +x_{2} =1$
\[\Rightarrow x_{1} \left. \frac{\partial Y_{m} }{\partial x_{2} } \right|_{T,p} =-x_{1} y_{1} +x_{1} y_{2} \] }

{\bf \includegraphics*[width=2.40in, height=1.26in]{image8}又由$Y=n_{1} y_{1} +n_{2} y_{2} \Rightarrow Y_{m} =x_{1} y_{1} +x_{1} y_{2} $故上式$\Leftrightarrow $}

\noindent 
{\bf $x_{1} \left. \frac{\partial Y_{m} }{\partial x_{2} } \right|_{T,p} =-Y_{m} +y_{2} $或$\begin{array}{l} {y_{2} =Y_{m} +x_{1} \left. \frac{\partial Y_{m} }{\partial x_{2} } \right|_{p,T} } \\ {y_{2} =Y_{m} +(1-x_{2} )\left. \frac{\partial Y_{m} }{\partial x_{2} } \right|_{p,T} } \end{array}$}

\noindent 
{\bf 多组分体系的推荐,见Darken(1950)和Hilbert and co-workers(1998)书?
\[\begin{array}{l} {y_{i} =Y_{m} +(1-x_{i} )\frac{\partial Y_{m} }{\partial x_{i} } |_{T,p,\{ x_{j} /x_{k} ,\ldots x_{n} /x_{k} \} (j\ne k\ne i,n\ne k\ne i)} } \\ {y_{i} =Y_{m} +\left. \frac{\partial Y_{m} }{\partial x_{i} } \right|_{T,p,\{ x_{j\ne i} \} } -\sum _{j=i}^{n}\left. \frac{\partial Y_{m} }{\partial x_{j} } \right|_{p,T,x_{j\ne i} }  } \end{array}\] }
JACS 72:2909(1950)

\noindent Hilbert(2008),Phase Equilibria,Phase Diagrams and Phase Transformation,2${}^{nd}$ ed. Cambridge University Press.

\noindent \eject 

\noindent \textbf{受力作用下的平衡溶胀}

\noindent 假定凝胶是从高分子干网络无形度$\boldsymbol{\mathrm{F}}_{0} =\boldsymbol{\mathrm{I}}$状态(参考构型)先经过各向同性膨胀$\boldsymbol{\mathrm{F}}_{1} =\left(\frac{V}{V_{0} } \right)^{\frac{1}{3} } \boldsymbol{\mathrm{I}}$达到体积$V$,再发生等容形度$\boldsymbol{\mathrm{F}}_{2} $达到总形度\textbf{$\boldsymbol{\mathrm{F}}$},则$\boldsymbol{\mathrm{F}}=\boldsymbol{\mathrm{F}}_{1} \boldsymbol{\mathrm{F}}_{2} =(\frac{V}{V_{0} } )^{\frac{1}{3} } \boldsymbol{\mathrm{F}}_{2} $ ,$\det \boldsymbol{\mathrm{F}}=\det \boldsymbol{\mathrm{F}}_{1} \det \boldsymbol{\mathrm{F}}_{2} =\frac{V}{V_{0} } $ 

\noindent 应变张量$\boldsymbol{\mathrm{B}}=\boldsymbol{\mathrm{FF}}^{\mathrm{T}} =\mathrm{F}_{1} \mathrm{F}_{2} (\mathrm{F}_{1} \mathrm{F}_{2} )^{\mathrm{T}} =(\frac{V}{V_{0} } )^{\frac{2}{3} } \mathrm{F}_{2} \mathrm{F}_{2}^{\mathrm{T}} =(\frac{V}{V_{0} } )^{\frac{2}{3} } \mathrm{B}_{2} $ ,其中$\boldsymbol{\mathrm{B}}_{2} =\boldsymbol{\mathrm{F}}_{2} \boldsymbol{\mathrm{F}}_{2}^{\mathrm{T}} $。

\noindent 现在以单轴拉伸为例,$\left. \boldsymbol{\mathrm{F}}_{2} {\rm =}\left(\begin{array}{ccc} {\lambda } & {} & {} \\ {} & {\lambda ^{-\frac{1}{2} } } & {} \\ {} & {} & {\lambda ^{-\frac{1}{2} } } \end{array}\right. \right)$ $\left. \boldsymbol{\mathrm{B}}_{2} =\left(\begin{array}{ccc} {\lambda ^{2} } & {} & {} \\ {} & {\lambda ^{-2} } & {} \\ {} & {} & {\lambda ^{-1} } \end{array}\right. \right)$ 

\noindent 弹性自由能贡献:
\[\begin{array}{rcl} {{\rm \Delta }G_{el} } & {=} & {\frac{1}{2} n_{ch} RT(tr\boldsymbol{\mathrm{B}}-3-\ln (\det \boldsymbol{\mathrm{F}}))} \\ {} & {=} & {\frac{1}{2} n_{ch} RT\left[\varphi _{{\rm 2}}^{-\frac{{\rm 2}}{{\rm 3}} } \left(\lambda ^{{\rm 2}} {\rm +2}\lambda ^{-{\rm 1}} \right)-{\rm 3}+{\rm ln}\varphi _{{\rm 2}} \right]\; \; \; \; v-\varphi _{{\rm 2}} \equiv \frac{V_{0} }{V} } \\ {} & {=} & {\frac{1}{2} n_{ch} RT\left(\varphi _{2}^{-\frac{2}{3} } \lambda ^{2} +2\varphi _{2}^{-\frac{2}{3} } \lambda ^{-1} +\ln \varphi _{2} -3\right)} \end{array}\] 
$\Delta$总自由能
\[\frac{{\rm \Delta }G}{RT} =n_{1} \ln \varphi _{1} +\chi _{12} n_{1} \varphi _{2} +\frac{1}{2} n_{ch} \left(\varphi _{2}^{-\frac{2}{3} } \lambda ^{2} +2\varphi _{2}^{-\frac{2}{3} } \lambda ^{-1} +\ln \varphi _{2} -3\right)\] 
在这个讨论当中,$\Delta G$的独自变量是$n_{1} $和$\lambda $。注意到$\varphi _{2} =\frac{V_{0} }{V} =\frac{V_{0} }{V_{0} +n_{1} \nu _{1} } $。我们固定$\lambda $求满足$\frac{\partial {\rm \Delta }G}{RT\partial n_{1} } =0$的$n_{1} $,就可得到在拉伸比$\lambda $下平衡溶胀的$\varphi _{2} $,暂时假定$\chi _{12} $为常数。
\[\begin{array}{rcl} {\frac{\partial {\rm \Delta }G_{mix} }{RT\partial n_{1} } } & {=} & {n_{1} \varphi _{1}^{-1} \varphi _{2} \frac{\nu _{1} }{V} +\ln \varphi _{1} +\chi _{12} \varphi _{2} +\chi _{12} n_{1} (-\varphi _{2} )\frac{\nu _{1} }{V} } \\ {} & {=} & {\varphi _{2} +\ln (1-\varphi _{2} )+\chi _{12} \varphi _{2}^{2} } \end{array}\] 
\[\begin{array}{rcl} {\frac{\partial {\rm \Delta }G_{el} }{RT\partial n_{1} } } & {=} & {\frac{1}{2} n_{ch} \left[\left(\lambda ^{2} +2\lambda ^{-1} \right)\left(-\frac{2}{3} \right)\varphi _{2}^{-\frac{5}{3} } \left(-\varphi _{2} \right)\frac{\nu _{1} }{V} +\varphi _{2}^{-1} \left(-\varphi _{2} \right)\frac{\nu _{1} }{V} \right]} \\ {} & {=} & {\frac{n_{{\rm ch}} }{V_{0} } v_{1} \left[\frac{1}{3} \left(\lambda ^{2} +2\lambda ^{-1} \right)\ell \varphi _{2}^{-\frac{2}{3} } -\frac{1}{2} \varphi _{2} \right]} \end{array}\] 
\[\frac{{\rm \Delta }\mu _{1} }{RT} =\frac{\partial {\rm \Delta }G}{RT\partial n_{1} } =\frac{\partial {\rm \Delta }G_{mix} }{RT\partial n_{1} } +\frac{\partial {\rm \Delta }G_{el} }{RT\partial n_{1} } \] 
平衡溶胀状态下,${\rm \Delta }\mu _{1} (\varphi _{2} ,\lambda )=0$隐含地规定了函数关系$\varphi _{2}^{\circ } (\lambda )$,其中上标?$\circ $?表示平衡溶胀。当无额外拉伸($\lambda ={\rm 1}$)时退化为Flory-Rehner溶胀形式(Flory书p578 eq.\eqref{GrindEQ__38_})。

\noindent \includegraphics*[width=5.52in, height=4.44in]{image9}

\noindent 由隐函数求导可得:
\[\frac{d\varphi }{d\lambda } =\frac{12n_{{\rm ch}} \nu _{1} \left(\lambda ^{3} -1\right)\varphi _{2} \left(\varphi _{2} -1\right)}{\lambda \left(n_{{\rm ch}} \nu _{{\rm 1}} \left(\varphi _{2} -1\right)\left(8+4\lambda ^{3} +9\lambda \varphi _{2}^{{\rm 5}/3} \right)-18\lambda \varphi _{2}^{8/3} \left(1+2\left(\varphi _{2} -1\right)\chi _{{\rm 1}2} \right)\right)} \] 
$\lambda ={\rm 1}$时${d\varphi \mathord{\left/ {\vphantom {d\varphi  d\lambda }} \right. \kern-\nulldelimiterspace} d\lambda } $总为零。故$\lambda ={\rm 1}$处是$\varphi _{2}^{} (\lambda )$的极值点或拐点。

\noindent 应力-应变曲线。

\noindent 由热力学关系,其中$L$是终态长度

\noindent 拉伸比$\lambda =\frac{L}{L_{1} } $,$\frac{L_{1} }{L_{0} } =(\frac{V}{V_{0} } )^{\frac{1}{3} } $,其中$L_{0} $是干网络长度,$L_{1} $是各向同性膨胀后无形度时的长度。

\noindent 故有$\lambda =\frac{L}{L_{0} } \varphi _{2}^{\frac{1}{3} } $,$F=\frac{\partial {\rm \Delta }G_{el} }{\partial \lambda } \frac{\partial \lambda }{\partial L} =\varphi _{2}^{\frac{1}{3} } L_{0}^{-{\rm 1}} \frac{\partial {\rm \Delta }G_{el} }{\partial \lambda } $

\noindent 拉伸真应力$\sigma =\frac{FL}{V} =\frac{FL\varphi _{2} }{V_{0} } =\varphi _{2}^{\frac{1}{3} } L_{0}^{-1} \frac{\partial {\rm \Delta }G_{el} }{\partial \lambda } L\varphi _{2} V_{0}^{-1} =\frac{n_{ch} }{V_{0} } RT\varphi _{2}^{\frac{1}{3} } (\lambda ^{2} -\lambda ^{-1} )$

\noindent 在平衡态下,$\varphi _{{\rm 2}} $受${\rm \Delta }\mu _{1} =0$约束必须取$\varphi _{2}^{\circ } (\lambda )$,而非无应力平衡溶胀下的值$\varphi _{2}^{\circ } ({\rm 1})$,故$\sigma (\lambda )=\frac{n_{ch} }{V_{0} } RT\varphi _{2}^{\circ } (\lambda )^{\frac{1}{3} } (\lambda ^{2} -\lambda ^{-1} )$,$\lambda ={\rm 1}$时退回到neo-Hookian的无形度溶胀形式。

\noindent \includegraphics*[width=5.77in, height=2.46in]{image10}

\noindent \eject 

\noindent \textbf{输运性质}

输运性质(transport properties)一般指材料的传质和传热相关的性质。包括粘度、导热系数和扩散系数。狭义上这些都指线性不可逆热力学下的概念,分别对应于:Navier-Stokes方程、热方程和扩散方程。NS方程描述动量输运,扩散方程描述质量输运,热方程描述热输运。

此外,还有关于电荷的输运性质,或称电输运性质,包括电导率、介电常数、磁导率。

\noindent \includegraphics*[width=5.73in, height=1.62in]{image11}

输运性质在空间中的非均质性及其渝渗,与材料结构的渝渗是相对独立的问题,因为动力学性质与结构(静态性质)之间没有普适的确定关系。

仅在输运性质的渝渗层面上讨论,以导体/绝缘体复合材料为例,我们观察的是材料整体的电导率(表观的)随导体比例的变化规律。取决于导电区的形貌和尺寸分布规定,材料的表观电导率在某一导体组成比例$p_{c} $下发生突变。在$p_{c} $附近,$\sigma _{eff} (p)\sim {\rm const}\times (p-p_{0} )^{t} $,$t$是电导率的临界指数。

如果我们进一步对输运性质的临界指数与静态性质的临界指数之间的关系感兴趣,那就无法回避动力学性质与静态性质的关系问题。

已知导体在绝缘体中分布的结构信息(密度涨落、团簇尺寸分布?),如何推算材料的总体电导率?这本身就是材料科学中的经典难题。连续介质的几种建模方法,已在Hughe的章节中总结了,这里不详细介绍。

\noindent B. Hughes (2021), in: M. Sahimi and A. Hunt eds. Complex Media and Percolation Theory, Springer Meester and Roy (1996) Continuum Percolation, Cambridge University Press.

\noindent Ex. Poisson-centered conducting spheres of constant radius.

\noindent $\phi _{c} \approx 0.2895\pm 0.0005$ J. Phys. A 30: L585(1997)

\noindent Swiss cheese model

\noindent $\phi _{c} \approx 0.03$ J. Appl. Phus. 71: 2727(1992)

\noindent 对于格子(离散)模型,一个canonical example就是random resistor network。专著:

\noindent B. Hughes (1996) Random walks and random environments (vol. 1\&2), Clarendon.

\noindent 一般的随机电阻网络模型仍是复杂的。若电阻网络的键满足独立同分布$f\left(g\right)$
\[f\left(g\right)=(1-p)\delta _{+} (g)+ph\left(g\right),0<p<1\] 
$h\left(g\right)$是电导的某条件分布,$\delta _{+} (g)$是$\delta $函数的右半边,则为general percolation conduction problem。

\noindent 若$f(g)=(1-p)\delta _{+} (g)+p\delta (g-g_{0} )$ ,$0<p<1$ ,(即当$h\left(g\right)\equiv \delta (g-g_{0} )$),则为standard percolation conduction problem。这时模型才比较易于进行直接的数学处理,但仍需指明网络的网络结构。工作只能给出渝渗的上下界。

\noindent 电导率在临界点附近的标度律假设:

 $\sigma _{eff} \left(p\right)\approx \mu S\left(\left[p-p_{c} \right]\lambda ^{-1} ,a\mu ^{-1} \lambda ^{-A} ,\mu b^{-1} \lambda ^{-B} \right)$ Straley (1976) J. Phys. C 9: 783

\noindent $S\left(x,y,z\right)$是一个函数,$x,y,z$均小于1,$S(x,y,z)\to \infty $ 当$x$或$y$,或$z\to {\rm 0}$ 

\noindent 在$p=p_{c} $时再令$\lambda ^{t+s} =a/b,\mu =b\lambda ^{t} $,则

 $\sigma _{eff} (p)\approx a^{u} b^{1-u} S\left(0,1,1\right)$ ,$u=t/(s+t)$

\noindent \includegraphics*[width=5.73in, height=4.34in]{image12}

\noindent \includegraphics*[width=5.76in, height=4.67in]{image13}

\noindent 对于扩散系数/导热系数(以扩散为例)

\noindent 需要考虑的是溶质在非均质环境中的扩散问题。或更明确地说,要考虑溶质在材料中的扩散系数的空间涨落及其渝渗问题。

\noindent 这涉及了分形上的随机行走的大量理论工作。也在Hughes (2021)章节中总结了。其中比较为人所知的问题是Ant in the Labyrinth模型/问题,主要关注的是均方回转半径/位移$\propto $聚合度/步长的指数$dw$ 。

\noindent 但$\left\langle \overline{S_{n} }\right\rangle $如何联系到扩散系数,以至于扩散系数空间分布的渝渗转变如何,Hughes (2021)未有涉及。也许对于非均质体系,$\left\langle \overline{S_{n} }\right\rangle \ne 2D_{n} $,($dw\ne 1$),故只能以$\left\langle \overline{S_{n} }\right\rangle \propto n^{dw} $来谈。

\noindent $dw$的具体分析Hughes (2021)和我总结的Cates工作。包括fraction dimension概念。

\noindent \eject 

\noindent \includegraphics*[width=1.66in, height=0.65in]{image14}\textbf{橡胶弹性的统计理论}

\noindent ?高分子网络的结构参数

\noindent $\mathrm{\triangle}$考虑N条线形聚合物发生硫化型交联,全部结合成一个网络。交联点的单元总在链中,不在链端。

\noindent $\mathrm{\triangle}$设$\rho $是交联点单元占总重复单元$N_{0} $的数量分数,$\rho _{t} $是交联后链末端单元的数量分数,则交联前总链数N满足
\[N=\frac{1}{2} N_{0} \rho _{t} \] 
交联单元数
\[\nu =N_{0} \rho \] 
$\mathrm{\triangle}$设重复单元分子量为$M_{0} $,交联前聚合物分子量是$M$,则有
\[N=\frac{N_{0} M_{0} }{M} =\frac{V}{\overline{\nu }M} \] 
其中$V$是体系的体积,$\overline{\nu }$是聚合物单位质量的体积(可由密度得出)。此外还有
\[\nu =\frac{N_{0} M_{0} }{M_{c} } =\frac{V}{\overline{\nu }M_{c} } \] 
其中$M_{c} $是交联点间分子量。

\noindent $\mathrm{\triangle}$?理想网络?无孤立链端(定义),等效于交联前$M\to \infty $,此时所有网链均为有效网链,有效网链数就是$\nu $。

\noindent $\mathrm{\triangle}$?真实网络?含孤立链端。考虑:

\noindent \includegraphics*[width=3.66in, height=2.46in]{image15}

\noindent (注意:这里还未考虑不同链的交联单元如何结合而构成网络,即官能度$f$未定。)

\noindent 再考虑到交联前每根链都在交联后引入2个孤立链端,故总有效网链数就是$\nu +N-2N=\nu -N$,有效网链占总网链分数
\[S_{a} =\frac{\nu -N}{\nu +N} =1-\frac{2M_{c} }{M+M_{c} } \] 
(代入了之前$N$和$\nu $关于$M$和$M_{c} $的表达式)

\noindent 当$M\gg M_{c} $时
\[S_{a} \approx 1-\frac{2M_{c} }{M} \] 
(对于橡胶而言,$M$常比$M_{c} $大两个数量级)

\noindent $\mathrm{\triangle}$设交联点官能度是$f$($f>2$),则每个交联点由${f\mathord{\left/ {\vphantom {f 2}} \right. \kern-\nulldelimiterspace} 2} $个交联单元结合而成,故交联点个数等于${2\nu \mathord{\left/ {\vphantom {2\nu  f}} \right. \kern-\nulldelimiterspace} f} $。记有效网链数为$\nu _{e} $,则同理有效交联点个数等于${2\nu _{e} \mathord{\left/ {\vphantom {2\nu _{e}  f}} \right. \kern-\nulldelimiterspace} f} $。由上面推导的$S_{a} $,此处的有效网链数

 $\nu _{e} =\nu (1-\frac{2M_{c} }{M} )$ (应用了$M\gg M_{c} $的近似)

\noindent 故有效交联点个数等于${2\nu _{e} \mathord{\left/ {\vphantom {2\nu _{e}  f}} \right. \kern-\nulldelimiterspace} f} ={2\nu \mathord{\left/ {\vphantom {2\nu  f}} \right. \kern-\nulldelimiterspace} f} -{4N\mathord{\left/ {\vphantom {4N f}} \right. \kern-\nulldelimiterspace} f} $。

\noindent $\mathrm{\triangle}$经常默认$f=4$,此时(有效)交联点个数等于$\frac{1}{2} \nu $(或$\frac{1}{2} \nu _{e} $),$\nu _{e} =\frac{1}{2} \nu -N$

\noindent $\mathrm{\triangle}$以上考虑仍忽略了:1)缠结;2)回环链

\noindent ?橡胶弹性的统计理论

\noindent $\mathrm{\triangle}$一个聚合物网络的构象熵有两项,一项来自固定交联点构型下,所有网链的构象熵;另一项则是交联点的构象熵。

\noindent $\mathrm{\triangle}$假定交联点固定取某构型,任意两交联点间的位移向量规定了该两点间的网链末端距。总体而言,一个固定的交联点构型规定了一个分布末端距取$\overrightarrow{R_{i} }$的网链数密度$W\left(\overrightarrow{R_{i} }\right)$,使得取末端距在$\overrightarrow{R_{i} }\sim \overline{R_{i} }+d\overrightarrow{R_{i} }$范围的网链条数为
\[\nu W\left(\overrightarrow{R_{i} }\right)d\overrightarrow{R_{i} }\] 
$\mathrm{\triangle}$为方便我们选取直角坐标系$\left\{\widehat{e}_{x} ,\widehat{e}_{y} ,\widehat{e}_{z} \right\}$,$\overrightarrow{R_{i} }=X_{i} \widehat{e}_{x} +Y_{i} \widehat{e}_{y} +Z_{i} \widehat{e}_{z} $,而且网络的形度可由$x,y,z$方向的主拉伸$\alpha _{x} ,\alpha _{y} ,\alpha _{z} $描述。这不影响结论的物质客观性。

\noindent $\mathrm{\triangle}$再假设,交联反应足够随机,交联点数量足够少,形度足够小,使得交联链都是高斯链。

\noindent Note:

\begin{enumerate}
\item  高斯链的特点是,链的任一截也是高斯链,有自相似性。

\item  形度足够小(线性区)对应高斯链,可参考J. Frenkel (1940) Rubber Chem. Technol. 13:264的argument。
\end{enumerate}

\noindent 这使得$W\left(\overrightarrow{R_{i} }\right)$是高斯链形式 ${\rm \Phi }(\overrightarrow{R_{i} })=(\frac{3}{2\pi nb^{2} } )^{\frac{3}{2} } \exp (-\frac{3\left\| \overrightarrow{R_{i} }\right\| ^{2} }{2nb^{2} } )$ 

\noindent $\mathrm{\triangle}$假设宏观形度的发生(在选定坐标系下$\boldsymbol{\mathrm{F}}\mathrm{=}\left(\begin{array}{ccc} {\alpha _{x} } & {} & {} \\ {} & {\alpha _{y} } & {} \\ {} & {} & {\alpha _{z} } \end{array}\right)$ )是仿射的,即每个末端距向量也相应发生相同的形度,则原末端距为$\overrightarrow{R_{i} }$的链形度后末端距均为
\[\overrightarrow{r_{i} }=\left(\begin{array}{ccc} {\alpha _{x} } & {} & {} \\ {} & {\alpha _{y} } & {} \\ {} & {} & {\alpha _{z} } \end{array}\right)\overrightarrow{R_{i} }\] 

 或$x_{i} =\alpha _{x} X_{i} ,y_{i} =\alpha _{y} Y_{i} ,z_{i} =\alpha _{z} Z_{i} $ 

\noindent 则形度后末端距取$\overrightarrow{r_{i} }$的条数就是
\[\nu _{i} \left(\overrightarrow{r_{i} }\right)=\nu W\left(\left(\begin{array}{ccc} {\alpha _{x} } & {} & {} \\ {} & {\alpha _{y} } & {} \\ {} & {} & {\alpha _{z} } \end{array}\right)^{-1} \overrightarrow{r_{i} }\right)\left(\begin{array}{ccc} {\alpha _{x} } & {} & {} \\ {} & {\alpha _{y} } & {} \\ {} & {} & {\alpha _{z} } \end{array}\right)^{-1} d\overrightarrow{r_{i} }\] 
\[\nu _{i} \left(x_{i} ,y_{i} ,z_{i} \right)=\nu W\left(\frac{x_{i} }{\alpha _{x} } ,\frac{y_{i} }{\alpha _{y} } ,\frac{z_{i} }{\alpha _{z} } \right)\frac{dx}{\alpha _{x} } \frac{dy}{\alpha _{y} } \frac{dz}{\alpha _{z} } \] 
$\mathrm{\triangle}$现在我们开始数数儿。

\noindent 将$\nu $条高斯链摆出满足末端距分布$\nu _{i} $的构型,可以有多少种摆法?

\noindent 由高斯链假设,一条网链满足$\overrightarrow{r_{i} }$末端距的构型数(概率)就是$\Phi \left(\overrightarrow{r_{i} }\right)$高斯形式。简记$\omega _{i} =\Phi \left(\overrightarrow{r_{i} }\right)$,又知这样的链有$\nu _{i} $条。故依次把$\nu $条链摆好成某构象的方法,(等效于$\nu $条链依次?亮相?并恰好满足某符合要求构象的概率)是:
\[\prod _{i}\omega _{i} {}^{\nu _{i} }  \] 
但我们不在乎?依次?亮相的次序不同,只在乎结果不同,故上式要乘以$\nu $条链按$\nu _{1} ,\nu _{2} ,\cdots ,\nu _{i} ,\cdots $排序的个数,即
\[{\rm \Omega }_{1} =\prod _{i}\omega _{i} {}^{\nu _{i} }  \times \frac{\nu !}{\prod _{i}\nu _{i} ! } \] 
(其中$\frac{\nu !}{\prod _{i}\nu _{i} ! } $表示从$\nu $条中先取$\nu _{1} $条排序,再取$\nu _{2} $条排序,?的排法数。)

\noindent 因此,固定一交联点构型,网链的构象熵$S_{1} =\ln \Omega _{1} $ 

\noindent 应用斯特林公式,$\ln n!=n\ln n-n$ 
\[\ln {\rm \Omega }_{1} ={\mathop{\smallint }\limits_{V}} \nu _{i} \ln (\omega _{i} \nu /\nu _{i} ){\rm d}\boldsymbol{\mathrm{r}}\] 
代入上式中的$\nu _{i} $和$\omega _{i} $   (Flory书上有详细积分技巧)

\noindent \eject 

\noindent \textbf{渝渗理论}

\noindent \textbf{参考资料:Stauffer Phys. Rep. 54:1(1979)}

\noindent \includegraphics*[width=1.04in, height=1.02in]{image16}\textbf{}

\noindent \textbf{2021年的专著(贡献者包括D. Stauffer, A. Coniglio, R. Ziff等):}

\noindent \textbf{M. Sahimi and A. Hunt (eds.) (2021), Complex Media and Percolation Theory, Springer.}

\noindent 1.渝渗是连续相变

\noindent $\mathrm{\triangle}$考虑一套网格,每个格子要么是空的,要么被占据。记$p$为格子被占据的概率,并设每个格子的$p$都相等。(这是stauffer的引入方式,也可视$p$为被占据格子数与总格子数之比。)在某$p$值下,我们随机地选择哪些格子被占据,绘出一种构象。

\noindent $\mathrm{\triangle}$团簇:a group of occupied sites connected by nearest-neighbor distances.

\noindent $\mathrm{\triangle}$记$n_{s} $为\textit{s}-团簇(含$s$个格子的团簇)的个数与总格子数之比,$n_{s} $是$p$的函数。(在这里$n_{s} $好像是系综平均,即在$p$取某值下所有可能构象的平均?还是说只考虑偶然的构象?在此没有能量概念,只要满足$p$取值,一切构象都等概率?)$n_{s} $满足$\sum _{s}n_{s} =n_{c}  $,$n_{c} $是所有团簇占据格子数与总格子数之比。

\noindent $\mathrm{\triangle}$考虑无限大网格,随着$p$值增加($p$值是浓度,就算网络无穷大也良好定义),到某值时会有一个无限大团簇。已知这种?无限大团簇?总是只有一个的。(Shante and Kirkpatrick (1971)Adv. Phys. 20:325)这个无限大网络称渝渗网络(percolating network)

\noindent $\mathrm{\triangle}$无限大团簇出现时的$p$是临界点$p_{c} $,当$p<p_{c} $,无无限大团簇,当$p>p_{c} $,有无限大团簇,$p=p_{c} $处发生了一个相变,称渝渗转变。

\noindent $\mathrm{\triangle}$逾渗转变是一种连续相变。记$P_{\infty } $为属于无限团簇的格子数与总格子数之比(又可作为某已占格子属于无限大团簇的概率,这种把概率等同于浓度的做法默认了遍历性),则$P_{\infty } $是渝渗转变的序参量。因为$P_{\infty } \left\{\begin{array}{l} {=0p<p_{c} } \\ {>0p\ge p_{c} } \end{array}\right. $,$P_{\infty } \left(p\right)$是连续函数,故为连续相变。

\noindent 2.逾渗转变的临界指数

\noindent $\mathrm{\triangle}$设体系的某性质$f$作为$\varepsilon \equiv {(p-p_{c} )\mathord{\left/ {\vphantom {(p-p_{c} ) p_{c} }} \right. \kern-\nulldelimiterspace} p_{c} } $的函数$f=f\left(\varepsilon \right)$在$\varepsilon \to 0$($p\to p_{c} $)的渐进行为

 $f\left(\varepsilon \right)=A\varepsilon ^{k} (1+B\varepsilon ^{k_{1} } +\cdots )$ ,$k{\mathop{=}\limits^{\mathit{det}}} {\mathop{\lim }\limits_{\varepsilon \to 0}} \frac{\ln \left|f\left(\varepsilon \right)\right|}{\ln \left|\varepsilon \right|} $称$f$的临界指数(critical exponent)

\noindent (其中$(1+B\varepsilon ^{k_{1} } +\cdots )$其实是某一般函数$\widetilde{f}\left(\varepsilon \right)$ )

\noindent $\mathrm{\triangle}$一般地,我们不假定$\varepsilon \to 0^{-1} $和$\varepsilon \to 0^{-} $的临界指数(记为$k'$、$k$)相等,它们仅需满足一系列不等式(见相变基础知识),但标度律假说下,$\varepsilon \to 0^{-1} $和$\varepsilon \to 0^{-} $的临界指数相等。(关于这个不等式$\mathrm{\to}$等式的问题要看相变的书)

\noindent $\mathrm{\triangle}$$k$值可能是非有理实数(这是相变常见特点),$f$可能是非解析函数。例如

 $f\left(\varepsilon \right)=A_{0} +A_{1} \varepsilon +A_{2} \varepsilon ^{1.355} +A_{3} \varepsilon ^{1.855} +A_{4} \varepsilon ^{2} +\cdots $ ($f$非解析的$\varepsilon $是相边界)

\noindent 则$f\left(\varepsilon \right)$在$\varepsilon =0$邻近是光滑的但是非解析的,奇异项是$A_{2} \varepsilon ^{1.355} $和$A_{3} \varepsilon ^{1.855} $,$A_{0} +A_{1} \varepsilon +A_{4} \varepsilon ^{2} $则为解析?背景?。(这里的意思是说$f\left(\varepsilon \right)$其实不是完全纯幂得$f\left(\varepsilon \right)\sim \varepsilon ^{\lambda } $,而是一个调和函数$f\left(\varepsilon \right){\rm =}\varepsilon ^{\lambda } \widetilde{f}\left(\varepsilon \right)$。其中$\widetilde{f}\left(\varepsilon \right)$可展开成$(1+\cdots $但我们经常简写成?$f\left(\varepsilon \right)\sim \varepsilon ^{\lambda } ,\varepsilon \to 0$?)

\noindent $\mathrm{\triangle}$具体地:

\noindent 记$\left[\sum _{s}s^{k} n_{s}  \right]_{\mathit{sing}} $为团簇的\textit{k}-阶矩对$\varepsilon $展开的首个奇异项。$\left[\sum _{s}s^{k} n_{s}  \right]_{\mathit{sing}} \propto \left|\varepsilon \right|^{\mathit{exponent}} $

\begin{tabular}{|p{1.3in}|p{1.3in}|p{1.3in}|} \hline 
$k$ & 物理量 & 临界指数 \\ \hline 
0 & $n_{c} $ & $2-\alpha $ \\ \hline 
1 & $P_{\infty } $ & $\beta $ \\ \hline 
2 & 团簇重均质量 & $-\gamma $ \\ \hline 
\end{tabular}

$n_{c} $在$\varepsilon =0$处是连续的,只有其3次导数发展,即$n_{c} \propto A+B\varepsilon +C\varepsilon ^{2} +D\varepsilon ^{2-\alpha } $ 

\noindent (这里跟液体或自旋的对比不太直接。)

\noindent 另,记$\xi $为相关长度,

 $\xi ^{2} \equiv \frac{\sum _{s}s^{2} n_{s} R_{s}^{2}  }{\sum _{s}s^{2} n_{s}  } $ 其中$R_{s} $是$s$团簇的半径。常假设满足分形关系$s\sim R_{s}^{d_{f} } $ 

\noindent 其临界指数是$\nu $,即$\xi \propto \varepsilon ^{-\nu } ,\varepsilon \to 0$ (这里的$\xi $就是团簇的$z$均半径)

\noindent 记?典型团簇质量?$s_{\xi } $为半径等于相关长度的团簇大小,或

 $s_{\xi } \equiv \frac{\sum _{s}s^{3} n_{s}  }{\sum _{s}s^{2} n_{s}  } $ 其临界指数是${-1\mathord{\left/ {\vphantom {-1 \sigma }} \right. \kern-\nulldelimiterspace} \sigma } $,即$s_{\xi } \propto \varepsilon ^{{-1\mathord{\left/ {\vphantom {-1 \sigma }} \right. \kern-\nulldelimiterspace} \sigma } } ,\varepsilon \to 0$ 

\noindent 总之我们引入了$\alpha ,\beta ,\gamma ,\nu ,\sigma $。

\noindent 3.体系在临界点处的性质(分形假设)

\noindent $\mathrm{\triangle}$如前面提到,我们常假设团簇取统一一个分形维数,$R_{s}^{d_{f} } \propto s$,我们可以考虑$d_{f} $随$p$变化,在这种情况下我们以下只考虑$p=p_{c} $时的团簇分形维数,仍记为$d_{f} $。(所以这部分应该放在$S\left(q\right)$的部分)

\noindent $\mathrm{\triangle}$网格所在空间的维数记为$d$。

\noindent $\mathrm{\triangle}$$d_{f} $与二点相关函数$g\left(r\right)$直接关联。$g\left(r\right)$是相隔$r$的两已占格子同属一个团簇的概率,它随$r$的衰减(在$\varepsilon =0$时)满足(由于分形假设)
\[g\left(r\right)\propto r^{-(d-d_{f} )} \] 
常引入指数$\eta $,满足$g\left(r\right)\propto r^{2-d-\eta } $,故$\eta \equiv 2-d_{f} $ 

\noindent ($g\left(r,\varepsilon \right)=\frac{1}{r^{d-2+\eta } } F_{G} \left(r\varepsilon ^{\nu } \right)$ 其中$\xi \propto \varepsilon ^{-\nu } $,即$F_{G} \left(r\xi ^{-1} \right)$ 。$d-2+\eta $这个是相关函数问题,$\eta $是Fisher在OZ的基础上加的。)

\noindent (这里的知识包括:$g\left(r\right)$到$S\left(q\right)$到散射光强$I\left(q\right)$的关联(最后这个只有依赖假设才具体化)。比如Ornstein-Zernike理论,绘出$G\left(r\right)\sim R^{-2(e^{{-\kappa r\mathord{\left/ {\vphantom {-\kappa r \gamma }} \right. \kern-\nulldelimiterspace} \gamma } } )} ,\kappa \equiv \xi ^{-1} $,correlation length。)

\noindent $\mathrm{\triangle}$在后续讨论还常考虑在团簇上的无规行走轨迹的分形维度$d_{w} $。以及这种无规行走中,经历的不同格子数与步数$N$的标度律指数$d_{s} $(经历的不同格子数$\propto N^{d_{s} } $),称为谱维数(spectral dimension)。

\noindent 4.标度律假设

\noindent $\mathrm{\triangle}$在相变基础中,标度律假设下临界指数之间将有特定的等式关系,在教科书中是基于Ising模型的自由能的广义调和假设推导的。在渝渗问题中标度律假设:

\noindent 在渝渗临界点邻近,体系的性质主要由典型团簇:$s_{\xi } $-团簇所决定(因为这些团簇在$p_{c} $附近处是很大的,否则它们的尺寸,即$\xi $,不会发散。)

\noindent 典型团簇的概念在Stauffer原文是定义松散的,使得该作者还要假设不同具体定义的$s_{\xi } $均以$s_{\xi } \propto \varepsilon ^{{-1\mathord{\left/ {\vphantom {-1 \sigma }} \right. \kern-\nulldelimiterspace} \sigma } } $发散。作者的意图是,$s_{\xi } $是$\sum _{s}s^{k} n_{s}  $的奇异项的主要贡献而非$\sum _{s}s^{k} n_{s}  $的主要贡献,矩阶$k$不重要,因为假设不同矩阶上述定义的$s_{\xi } $均按$s_{\xi } \propto \varepsilon ^{{-1\mathord{\left/ {\vphantom {-1 \sigma }} \right. \kern-\nulldelimiterspace} \sigma } } $发展。

\noindent $\mathrm{\triangle}$按此假设,体系性质均应只依赖比值${s\mathord{\left/ {\vphantom {s s_{\xi } }} \right. \kern-\nulldelimiterspace} s_{\xi } } $。然而直接写?$n_{s} \left(p\right)=f\left({s\mathord{\left/ {\vphantom {s s_{\xi } }} \right. \kern-\nulldelimiterspace} s_{\xi } } \right)$?过于简单,这时当$p=p_{c} $ $s_{\xi } \to \infty $,$n_{s} $就成了常数,这不符合渝渗过程。我们更倾向于令比值${n\left(s\right)\mathord{\left/ {\vphantom {n\left(s\right) n\left(s_{\xi } \right)}} \right. \kern-\nulldelimiterspace} n\left(s_{\xi } \right)} $仅依赖${s\mathord{\left/ {\vphantom {s s_{\xi } }} \right. \kern-\nulldelimiterspace} s_{\xi } } $,得到下式
\[{\raise0.7ex\hbox{$ n_{s}  $}\!\mathord{\left/ {\vphantom {n_{s}  n_{s_{\xi } } }} \right. \kern-\nulldelimiterspace}\!\lower0.7ex\hbox{$ n_{s_{\xi } }  $}} =\widetilde{f}\left({\raise0.7ex\hbox{$ s $}\!\mathord{\left/ {\vphantom {s s_{\xi } }} \right. \kern-\nulldelimiterspace}\!\lower0.7ex\hbox{$ s_{\xi }  $}} \right)\] 
$\mathrm{\triangle}$为了实用性改考虑比值${n_{s} \left(p\right)\mathord{\left/ {\vphantom {n_{s} \left(p\right) n_{s} \left(p_{c} \right)}} \right. \kern-\nulldelimiterspace} n_{s} \left(p_{c} \right)} $,并把上述假设改写为:

\noindent 假设比值$V\left(p\right)\equiv {\raise0.7ex\hbox{$ n_{s} \left(p\right) $}\!\mathord{\left/ {\vphantom {n_{s} \left(p\right) n_{s} \left(p_{c} \right)}} \right. \kern-\nulldelimiterspace}\!\lower0.7ex\hbox{$ n_{s} \left(p_{c} \right) $}} $ 以及其他团簇性质的类似比值都只依赖比值${s\mathord{\left/ {\vphantom {s s_{\xi } }} \right. \kern-\nulldelimiterspace} s_{\xi } } $:
\[V\left(p\right)=\widetilde{f}\left({\raise0.7ex\hbox{$ s $}\!\mathord{\left/ {\vphantom {s s_{\xi } }} \right. \kern-\nulldelimiterspace}\!\lower0.7ex\hbox{$ s_{\xi }  $}} \right)\] 
这就是渝渗理论中的标度律假设。按此假设,不同$\varepsilon $的$n_{s} $分布曲线可通过重整化横坐标为${s\mathord{\left/ {\vphantom {s s_{\xi } }} \right. \kern-\nulldelimiterspace} s_{\xi } } $,纵坐标为而得到一条主曲线$\widetilde{f}\left({s\mathord{\left/ {\vphantom {s s_{\xi } }} \right. \kern-\nulldelimiterspace} s_{\xi } } \right)$

\noindent $\mathrm{\triangle}$进一步,$n_{s} \left(p_{c} \right)$满足$n_{s} \propto s^{-\tau } $,$p=p_{c} $。这有两个支持,一是蒙特卡罗模拟结果,二是考虑到任何更复杂的分布$n_{s} \propto s^{-\tau } \exp \left(-{s\mathord{\left/ {\vphantom {s s_{0} }} \right. \kern-\nulldelimiterspace} s_{0} } \right)$都会给出截断团簇$s_{0} $,作为典型团簇,不满足典型团簇发散的模型设定。因此有

 $n_{s} \left(p\right)\propto s^{-\tau } \widetilde{f}\left({s\mathord{\left/ {\vphantom {s s_{\xi } }} \right. \kern-\nulldelimiterspace} s_{\xi } } \right)$ (即$n_{s} \left(p\right)\propto n_{s} \left(p_{c} \right)\widetilde{f}\left({s\mathord{\left/ {\vphantom {s s_{\xi } }} \right. \kern-\nulldelimiterspace} s_{\xi } } \right)$)

\noindent $\mathrm{\triangle}$由$s_{\xi } \propto \varepsilon ^{{-1\mathord{\left/ {\vphantom {-1 \sigma }} \right. \kern-\nulldelimiterspace} \sigma } } $,${s\mathord{\left/ {\vphantom {s s_{\xi } }} \right. \kern-\nulldelimiterspace} s_{\xi } } \propto s\varepsilon ^{{1\mathord{\left/ {\vphantom {1 \sigma }} \right. \kern-\nulldelimiterspace} \sigma } } \propto \left(s^{\sigma } \varepsilon \right)^{{1\mathord{\left/ {\vphantom {1 \sigma }} \right. \kern-\nulldelimiterspace} \sigma } } $,我们可将$\widetilde{f}$换成另一个函数$f$使得

 $n_{s} \left(p\right)\propto s^{-\tau } f\left(\varepsilon s^{\sigma } \right)$ ,其中$f\left(x\right)$满足$f\left(0\right)=1$ 

\noindent 5.临界指数之间的关系

\noindent $\mathrm{\triangle}$基于标度律假设,指数$\tau $和$\sigma $与$\alpha ,\beta ,\gamma \cdots $的关系还需更多推导。$\alpha ,\beta ,\gamma \cdots $定义中的$\sum _{s}s^{k} n_{s}  $要联系到$\sum _{s_{\xi } }s_{\xi }^{k} n_{s_{\xi } }  $,才能完成这个任务。这时可用$n_{s_{\xi } } \propto s^{-\tau } f\left(const\right)\propto s^{-\tau } $ 
\[\begin{array}{rcl} {\sum _{s}s^{k} n_{s}  } & {\to } & {\int _{0}^{\infty }s^{k} n\left(s\right) ds=q_{0} \int _{0}^{\infty }s^{k-\tau }  f\left(\varepsilon s^{\sigma } \right)ds} \\ {} & {=} & {\frac{q_{0} }{\sigma } \int _{0}^{\pm \infty }s^{1+k-\tau }  z^{-1} f\left(z\right)dz} \end{array}\] 
上式是令$z=\varepsilon s^{\sigma } $,$dz=\varepsilon \sigma s^{\sigma -1} ds$,$ds=\varepsilon ^{-1} \sigma ^{-1} s^{1-\sigma } dz=sz^{-1} \sigma ^{-1} dz$得到的。当$p<p_{c} $时$z$积分从$0\sim -\infty $,$p>p_{c} $时积分从$0\sim +\infty $。

\noindent 这时我们用
\[\left|\varepsilon s^{\sigma } \right|^{{(1+k-\tau )\mathord{\left/ {\vphantom {(1+k-\tau ) \sigma }} \right. \kern-\nulldelimiterspace} \sigma } } =\left|\varepsilon \right|^{{(1+k-\tau )\mathord{\left/ {\vphantom {(1+k-\tau ) \sigma }} \right. \kern-\nulldelimiterspace} \sigma } } s^{1+k-\tau } \] 
\[\begin{array}{rcl} {Ÿ} & {=} & {\frac{q_{0} }{\sigma } \left|\varepsilon \right|^{-{(1+k-\tau )\mathord{\left/ {\vphantom {(1+k-\tau ) \sigma }} \right. \kern-\nulldelimiterspace} \sigma } } \int \left|z\right| ^{{(1+k-\tau )\mathord{\left/ {\vphantom {(1+k-\tau ) \sigma }} \right. \kern-\nulldelimiterspace} \sigma } } z^{-1} f\left(z\right)dz} \\ {} & {\propto } & {\left|\varepsilon \right|^{{(\tau -1-k)\mathord{\left/ {\vphantom {(\tau -1-k) \sigma }} \right. \kern-\nulldelimiterspace} \sigma } } } \end{array}\] 
(见D. Stauffer原文有更多讨论)

\noindent $\mathrm{\triangle}$$k=1,2,3,\cdots $对应$\alpha ,\beta ,\gamma ,\cdots $,故有

 $\left\{\begin{array}{rcl} {2-\alpha } & {=} & {{(\tau -1)\mathord{\left/ {\vphantom {(\tau -1) \sigma }} \right. \kern-\nulldelimiterspace} \sigma } } \\ {\beta } & {=} & {{(\tau -2)\mathord{\left/ {\vphantom {(\tau -2) \sigma }} \right. \kern-\nulldelimiterspace} \sigma } } \\ {-\gamma } & {=} & {{(\tau -3)\mathord{\left/ {\vphantom {(\tau -3) \sigma }} \right. \kern-\nulldelimiterspace} \sigma } } \end{array}\right. $  ,$p=p_{c} $ 

\noindent $\mathrm{\triangle}$关于渝渗模型中的指数$\delta $,在渝渗图像中没有直观意义。在伊辛模型中除了温度$T$还有外场$H$,在渝渗模型中假想某外场$h$,则

 $\sum _{s}sn_{s}  \left(p_{c} \right)e^{-hs} \propto h^{{1\mathord{\left/ {\vphantom {1 \delta }} \right. \kern-\nulldelimiterspace} \delta } } $ ,$p=p_{c} $ $h\to 0$,在标度律假定下${1\mathord{\left/ {\vphantom {1 \delta }} \right. \kern-\nulldelimiterspace} \delta } =\tau -2$,$p=p_{c} $

\noindent (更多讨论和参考文献见Stauffer原文。)

\noindent $\mathrm{\triangle}$$\alpha ,\beta ,\gamma ,\delta $的式子可两两消去$\tau $,或两两消去$\sigma $得到更多等式。部分形式如下:

 $2-\alpha =2\beta +\gamma =\beta (\delta +1)=\gamma \frac{\delta +1}{\delta -1} $ ,$\beta \delta ={1\mathord{\left/ {\vphantom {1 \sigma }} \right. \kern-\nulldelimiterspace} \sigma } =\frac{1}{\beta +\gamma } $ 

\noindent 6.强标度律假设

\noindent $\mathrm{\triangle}$由标度律假设的精神,团簇半径与相关长度的比值${R_{s} \mathord{\left/ {\vphantom {R_{s}  \xi }} \right. \kern-\nulldelimiterspace} \xi } $在$p_{c} $临近应只依赖比值${s\mathord{\left/ {\vphantom {s s_{\xi } }} \right. \kern-\nulldelimiterspace} s_{\xi } } $。故可写

 $R_{s} =\xi \widetilde{R}\left(z\right)$ ,$z=\varepsilon s^{\sigma } $,这仍是标度度律假设。

\noindent 由$\xi \propto \varepsilon ^{-\nu } \propto s_{\xi }^{\sigma \nu } $ ,上式改写为
\[R_{s} =s_{\xi }^{\sigma \nu } \widetilde{R}_{1} \left(z\right)\] 
$\mathrm{\triangle}$为了考虑$\widetilde{R}\left(z\right)$或$\widetilde{R}_{1} \left(z\right)$的形式,需要一个新的假设:

\noindent 在$p>p_{c} $但接近$p_{c} $时,一个很大但非无限大的团簇密度与无限大团簇很接近,因此这个很大的有限团簇密度可以用$P_{\infty } $来计算,即$pP_{\infty } $(视$s$为质量)

\noindent $\mathrm{\triangle}$在此假设下,记该团簇体积为$V_{s} $,$s=V_{s} pP_{\infty } $,当$p\to p_{c} $,$P_{\infty } \propto \left|\varepsilon \right|^{\beta } $,故

 $V_{s} \propto s\left|\varepsilon \right|^{-\beta } $ (跟体积相关,就跟维数$d$相关)

\noindent 又由体积几何意义$V_{s} \propto R_{s}^{d} $,故必需要有$R_{s} \propto \varepsilon ^{-{\beta \mathord{\left/ {\vphantom {\beta  d}} \right. \kern-\nulldelimiterspace} d} } s^{{1\mathord{\left/ {\vphantom {1 d}} \right. \kern-\nulldelimiterspace} d} } $,$p>p_{c} $,$s$是很大的有限值。

\noindent $\mathrm{\triangle}$我们首先注意到,$R_{s} $关于$s$的依赖关系是$R_{s} \propto s^{{1\mathord{\left/ {\vphantom {1 d}} \right. \kern-\nulldelimiterspace} d} } $,若$R_{s} \propto \xi \widetilde{R}\left(z\right)$,则这一依赖关系全靠$\widetilde{R}\left(z\right)$,因为$z$含$s$。具体地,一个出现$s^{{1\mathord{\left/ {\vphantom {1 d}} \right. \kern-\nulldelimiterspace} d} } $的$z$指数是$z^{{1\mathord{\left/ {\vphantom {1 (\sigma d)}} \right. \kern-\nulldelimiterspace} (\sigma d)} } $(由$z=\varepsilon s^{\sigma } $),故$\widetilde{R}\left(z\right)$在$p\to p_{c}^{+} $时的渐近应有$\widetilde{R}\left(z\right)\sim z^{{1\mathord{\left/ {\vphantom {1 (\sigma d)}} \right. \kern-\nulldelimiterspace} (\sigma d)} } $

\noindent $\mathrm{\triangle}$进而,$R_{s} \propto \xi z^{{1\mathord{\left/ {\vphantom {1 (\sigma d)}} \right. \kern-\nulldelimiterspace} (\sigma d)} } $再由$\xi \propto \varepsilon ^{-\nu } $可得$R_{s} \propto \varepsilon ^{-\nu } \varepsilon ^{{1\mathord{\left/ {\vphantom {1 (\sigma d)}} \right. \kern-\nulldelimiterspace} (\sigma d)} } s^{{1\mathord{\left/ {\vphantom {1 d}} \right. \kern-\nulldelimiterspace} d} } \propto \varepsilon ^{-\nu +{1\mathord{\left/ {\vphantom {1 (\sigma d)}} \right. \kern-\nulldelimiterspace} (\sigma d)} } s^{{1\mathord{\left/ {\vphantom {1 d}} \right. \kern-\nulldelimiterspace} d} } $,该式与上面得到的$R_{s} \propto \varepsilon ^{-{\beta \mathord{\left/ {\vphantom {\beta  d}} \right. \kern-\nulldelimiterspace} d} } s^{{1\mathord{\left/ {\vphantom {1 d}} \right. \kern-\nulldelimiterspace} d} } $比较得:${\beta \mathord{\left/ {\vphantom {\beta  d}} \right. \kern-\nulldelimiterspace} d} =\nu -{1\mathord{\left/ {\vphantom {1 (\sigma d)}} \right. \kern-\nulldelimiterspace} (\sigma d)} $,或
\[d\nu =\beta +{1\mathord{\left/ {\vphantom {1 \sigma }} \right. \kern-\nulldelimiterspace} \sigma } =\beta (\delta +1)=2-\alpha \] 
第一个等号是本节新增假设下才有的,称为强标度(hyperscaling)假设。(强标度假设:定义为涉及维数$d$的标度假设。)

\noindent $\mathrm{\triangle}$根据相变理论的更多研究(见Stauffer原文参考文献),强标度在$d>6$失效,平均场(朗道)在$d>6$成立,在$d=6$两者等价。

\noindent $\mathrm{\triangle}$在$p=p_{c} $,$R_{s} $应是有限值,故$\widetilde{R}_{1} \left(z\to 0\right)\to const$,$R_{s} \left(p_{c} \right)\propto s^{\sigma \nu } $,故$d_{f} ={1\mathord{\left/ {\vphantom {1 (\sigma \nu )}} \right. \kern-\nulldelimiterspace} (\sigma \nu )} $ 

\noindent \eject 

\noindent \textbf{正则溶液}

\noindent \includegraphics*[width=5.69in, height=4.61in]{image17}

\noindent $\mathrm{\triangle}$考虑双组份,小分子,摩尔数与体积的关系:

\noindent $n_{i} $:摩尔数,$n=\sum _{i}n_{i}  $ 

\noindent $N_{i} $:个数,$N_{i} =n_{i} N_{A} $  $N=\sum _{i}N_{i}  $ 

\noindent $x_{i} $:摩尔分数,$x_{i} ={n_{i} \mathord{\left/ {\vphantom {n_{i}  n}} \right. \kern-\nulldelimiterspace} n} ={N_{i} \mathord{\left/ {\vphantom {N_{i}  N}} \right. \kern-\nulldelimiterspace} N} $ 

\noindent $v_{i} $:纯组份$i$摩尔体积
\[V=\sum _{i}n_{i}  v_{i} +nv_{12}^{E} \] 
其中$v^{E} $由上式定义,是$x_{1} $的函数。

\noindent $\mathrm{\triangle}$由上述体积关系,把体积$V$分为$n$摩尔个格子,每个体积大小视所放入的分子种类取$v_{i} $,$i=1,2$。把两种组份的分子放入格子中的方法是组合数

 $\Omega =\frac{N!}{N_{1} !N_{2} !} $ (给定组成$x_{1} $)

\noindent 每种放法就是一个构象。分子可以通过热运动遍历所有给定宏观约束下允许的构象。设体系是正则系统,
\[\begin{array}{rcl} {Z} & {=} & {\int _{\Gamma }d\left\{\vec{p}_{i} ,\vec{r}_{i} \right\} \exp \left(-{\raise0.7ex\hbox{$ {\rm H}  $}\!\mathord{\left/ {\vphantom {{\rm H}  k_{B} {\rm T} }} \right. \kern-\nulldelimiterspace}\!\lower0.7ex\hbox{$ k_{B} {\rm T}  $}} \right)} \\ {} & {=} & {\int _{\Gamma }d\left\{\vec{p}_{i} ,\vec{r}_{i} \right\}\exp \left(-\sum _{i=1}^{N}\left[\frac{\vec{p}_{i} {}^{2} }{2m} +u\left(\left\{\vec{r}_{i} \right\}\right)\right] \right) } \\ {} & {=} & {\int _{\Gamma }d\left\{\vec{p}_{i} \right\} \exp \left(-\sum _{i=1}^{N}\frac{\vec{p}_{i} {}^{2} }{2k_{B} {\rm T} m}  \right)\int _{\Gamma }d\left\{\vec{r}_{i} \right\} \exp \left(-\frac{u\left(\left\{\vec{r}_{i} \right\}\right)}{k_{B} {\rm T} } \right)} \\ {} & {=} & {\left(2\pi mk_{B} {\rm T} \right)^{\frac{3N}{2} } \int _{\Gamma }d\left\{\vec{r}_{i} \right\} \exp \left(-\frac{u\left(\left\{\vec{r}_{i} \right\}\right)}{k_{B} {\rm T} } \right)} \end{array}\] 
其中$\left\{\vec{r}_{i} \right\}$就代表某种格子构象,给定格子构象下的总势能,引入格子维数$d\equiv 3$,配位数$z$,分子间作用能$\varepsilon _{11} ,\varepsilon _{22} ,\varepsilon _{12} $ 

\noindent 在这里,格子简化相当于对$u_{ij} \left(\vec{r}_{ij} \right)$进行了简化。

\noindent 给定一个格子,其相邻$z$个格子中组分1分子的比例等于平均组成(平均场):$\frac{N_{1} }{N-1} \approx \frac{N_{1} }{N} =x_{1} $,故一个组份2周围平均有$zx_{1} $个组分1分子和$z\left(1-x_{1} \right)$个组份2分子。1-2相邻的情况数量是$zx_{1} N_{2} $ 

\noindent 体系的总势能可通过分层(全为1-1与1-2接触)时的值加上交换混合的差值得到。一个1-1与一个2-2变成两个1-2的势能差$2\Delta \varepsilon =2\varepsilon _{12} -\varepsilon _{11} -\varepsilon _{22} $或$\Delta \varepsilon =\varepsilon _{12} -\frac{1}{2} \left(\varepsilon _{11} -\varepsilon _{22} \right)$,我们视$\Delta \varepsilon $为该混合物组份组合的特征参数。

 $\begin{array}{rcl} {u} & {=} & {\frac{1}{2} N_{1} \varepsilon _{11} +\frac{1}{2} N_{2} \varepsilon _{22} +zx_{1} N_{2} \Delta \varepsilon } \\ {} & {=} & {u_{0} +zx_{1} N_{2} \Delta \varepsilon } \end{array}$  不同构象的$u$是相同的

\noindent (其中$\frac{1}{2} N_{1} \varepsilon _{11} +\frac{1}{2} N_{2} \varepsilon _{22} $:分层势能$\equiv u_{0} $ ,$zx_{1} N_{2} \Delta \varepsilon $:变为当前构象的势能增量)

\noindent 令$\chi _{12} \equiv {\Delta z\varepsilon \mathord{\left/ {\vphantom {\Delta z\varepsilon  (k_{B} {\rm T} }} \right. \kern-\nulldelimiterspace} (k_{B} {\rm T} } )$ 则$u\left(\left\{\vec{r}_{i} \right\}\right)=u_{0} +k_{B} {\rm T} \chi _{12} x_{1} N_{2} $ 
\[Z=\left(2\pi mk_{B} {\rm T} \right)^{\frac{3N}{2} } \int _{\Gamma }d\left\{\vec{r}_{i} \right\} \exp \left(-\frac{u\left(\left\{\vec{r}_{i} \right\}\right)}{k_{B} {\rm T} } \right)=\left(2\pi mk_{B} {\rm T} \right)^{\frac{3N}{2} } \exp \left(-\frac{u_{0} +zx_{1} N_{2} \Delta \varepsilon }{k_{B} {\rm T} } \right)\Omega \] 
其中$\int _{\Gamma }d\left\{\vec{r}_{i} \right\} \equiv \Omega $ 
\[\ln Z=-\ln Z_{0} -\frac{1}{k_{B} {\rm T} } (u_{0} +zx_{1} N_{2} {\rm \Delta }\varepsilon )+\ln \Omega \] 
\[F=-k_{B} {\rm T} \ln Z=-k_{B} {\rm T} \ln Z_{0} +u_{0} +zx_{1} N_{2} \Delta \varepsilon -k_{B} {\rm T} \ln \Omega \] 
若初态是分层则$F_{0} =-k_{B} T\ln Z_{0} +u_{0} $ ,${\rm \Delta }F_{{\rm mix}} =F-F_{0} =zx_{1} N_{2} {\rm \Delta }\varepsilon -k_{B} {\rm T} \ln {\rm \Omega }$ 

\noindent $\ln \Omega $按一般教材推导,用斯特林公式
\[\ln {\rm \Omega }\approx -\left(N_{1} \ln x_{1} +N_{2} \ln x_{2} \right)\] 
\[\begin{array}{rcl} {{\rm \Delta }F_{\mathrm{mix}} } & {=} & {k_{B} {\rm T} \left(N_{2} x_{1} \frac{z{\rm \Delta }\varepsilon }{k_{B} {\rm T} } +N_{1} \ln x_{1} +N_{2} \ln x_{2} \right)} \\ {} & {=} & {Nk_{B} {\rm T} \left(\frac{z{\rm \Delta }\varepsilon }{k_{B} {\rm T} } x_{1} x_{2} +x_{1} \ln x_{1} +x_{2} \ln x_{2} \right)} \\ {} & {=} & {Nk_{B} {\rm T} \left(\chi _{12} x_{1} x_{2} +x_{1} \ln x_{1} +x_{2} \ln x_{2} \right)} \\ {} & {=} & {k_{B} {\rm T} \left(\chi _{12} N_{1} x_{2} +N_{1} \ln x_{1} +N_{2} \ln x_{2} \right)} \end{array}\] 
${\rm \Delta }G_{mix} ={\rm \Delta }F_{mix} +p{\rm \Delta }V$,其中${\rm \Delta }V$是混合造成体积变化,故$\Delta V=nv_{12}^{E} $,若该项忽略则${\rm \Delta }G_{mix} \approx {\rm \Delta }F_{mix} $

 $x_{i} $与$\phi _{i} $差别:$\phi _{i} =\frac{n_{i} \nu _{i} }{V} $ ,$x_{i} =\frac{n_{i} }{n} $ 

 $\therefore n_{i} =nx_{i} $ ,$\phi _{i} =\frac{nx_{i} v_{i} }{V} $ ,$x_{i} =\frac{\phi _{i} V}{nv_{i} } =\phi _{i} \frac{\bar{v}}{v_{i} } $ ,$\bar{v}\equiv \frac{V}{n} $是平均摩尔体积

 ${\rm \Delta }F_{mix} =Nk_{B} {\rm T} \left(\chi _{12} \phi _{1} \phi _{2} \frac{\bar{v}^{2} }{v_{1} v_{2} } +\phi _{1} \frac{\bar{v}}{v_{1} } \ln \phi _{1} +\phi _{2} \frac{\bar{v}}{v_{2} } \ln \phi _{2} +\ln \frac{\bar{v}^{2} }{v_{1} v_{2} } \right)$ 是不可化简成的。

\noindent 当$v_{1} =v_{2} =\bar{v}$时,$x_{i} =\phi _{i} $,${\rm \Delta }F_{mix} $才可写成纯$\phi _{i} $表示。 

\noindent $\mathrm{\triangle}$对于摩尔体积差别不可忽略的体系,也可视格子体积恒定,$v_{o} \equiv \frac{V}{R} $,$R$为格子数。$i$组份分子总共占据$r_{i} $个格子,$\sum _{i}r_{i}  =R$,则$i$组份的摩尔体积$v_{i} ={r_{i} V_{0} \mathord{\left/ {\vphantom {r_{i} V_{0}  n_{i} }} \right. \kern-\nulldelimiterspace} n_{i} } $,体积分数$\phi _{i} {\rm =}{r_{i} \mathord{\left/ {\vphantom {r_{i}  R}} \right. \kern-\nulldelimiterspace} R} $。

\noindent 这种考虑如何继续待补充。

\noindent 对于$i$分子,原网格

\noindent \eject 

\noindent 

\noindent \textbf{自由结合链的末端距概率密度}

\noindent $\mathrm{\triangle}$球面上的均匀分布:三维欧几里德空间${\mathbb R}^{{\rm 3}} $的位置向量$\vec{r}$满足概率密度$\psi \left(\vec{r}\right)$,
\[\psi \left(\vec{r}\right)=\frac{1}{4\pi b^{2} } \delta \left(\left\| \vec{r}\right\| -b\right),\psi :\mathrm{{\mathbb R}}^{3} \to \mathrm{{\mathbb R}}\] 
是半径为$b$的球面上的均匀分布。其中Dirac delta函数

 $\delta \left(\vec{r}\right)=\left(2\pi \right)^{-d} \int _{{\mathbb R}^{d} }\exp \left(i\vec{k}\cdot \vec{r}\right) d\vec{k}$ ,$d$是维数

\noindent $\mathrm{\triangle}$球面上的均匀分布与正态分布的关系:若$\vec{r}$满足半径为$b$的球面分布,则${r_{i} \mathord{\left/ {\vphantom {r_{i}  b}} \right. \kern-\nulldelimiterspace} b} \sim N\left(0,1\right)$,$i=1,2,3$ (Muller (1959) Commun. ACM 2:19)(这使得许多书以$\vec{r}$在直角坐标轴上的投影的一维随机行走来推算。)

\noindent 球坐标系:
\[\left\{\begin{array}{l} {x=\rho \sin \theta \cos \varphi \quad \rho >0,\quad 0<\theta \le \pi ,\quad 0<\varphi <2\pi } \\ {y=\rho \sin \theta \sin \varphi } \\ {z=\rho \cos \theta } \end{array}\right. \] 
\[J=\left(\begin{array}{ccc} {\sin \theta \cos \varphi } & {\rho \cos \theta \cos \varphi } & {-\rho \sin \theta \sin \varphi } \\ {\sin \theta \sin \varphi } & {\rho \cos \theta \sin \varphi } & {\rho \sin \theta \cos \varphi } \\ {\cos \theta } & {-\rho \sin \theta } & {0} \end{array}\right)\] 
\[\begin{array}{l} {\vec{C}_{1} =\left(\sin \theta \cos \varphi ,\sin \theta \sin \varphi ,\cos \theta \right)^{{\rm T}} } \\ {\vec{C}_{2} =\left(\rho \cos \theta \cos \varphi ,\rho \cos \theta \sin \varphi ,-\rho \sin \theta \right)^{{\rm T}} } \\ {\vec{C}_{3} =\left(-\rho \sin \theta \sin \varphi ,\rho \sin \theta \cos \varphi ,0\right)^{{\rm T}} } \\ {h_{1} =1,h_{2} =\rho ,h_{3} =\rho \sin \theta } \end{array}\] 
\[\left\{\begin{array}{l} {x^{c} =x\sin \theta \cos \varphi +y\sin \theta \sin \varphi +z\cos \theta =\sqrt{x^{2} +y^{2} +z^{2} } } \\ {y^{c} =x\cos \theta \cos \varphi +y\cos \theta \sin \varphi -z\sin \theta =0} \\ {z^{c} =-x\sin \varphi +y\cos \varphi =0} \end{array}\right. \] 
注意一般地,点$\vec{r}=r_{\rho } \hat{\rho }+r_{\theta } \hat{\theta }+r_{\varphi } \hat{\varphi }$的基$\hat{\rho },\hat{\theta },\hat{\varphi }$依赖$\vec{r}$,故
\[\vec{k}=k_{\rho } \hat{\rho }(\vec{k})\vec{r}=r_{\rho } \hat{\rho }(\vec{r})\] 
\[\begin{array}{rcl} {\vec{k}\cdot \vec{r}} & {=} & {k_{\rho } r_{\rho } \hat{\rho }(\vec{k})\cdot \hat{\rho }(\vec{r})} \\ {} & {=} & {k_{\rho } r_{\rho } \left(\sin \theta _{k} \cos \varphi _{k} \sin \theta _{r} \cos \varphi _{r} +\sin \theta _{k} \sin \varphi _{k} \sin \theta _{r} \sin \varphi _{r} +\cos \theta _{k} \cos \theta _{r} \right)} \\ {} & {=} & {k_{\rho } r_{\rho } \left[\sin \theta _{k} \sin \theta _{r} \cos \left(\varphi _{k} -\varphi _{r} \right)+\cos \theta _{k} \cos \theta _{r} \right]} \\ {} & {=} & {k_{\rho } r_{\rho } \cos \omega } \end{array}\] 
($\omega $是$\vec{k}$与$\vec{r}$的夹角)

\noindent 如果在整个${\mathbb R}^{{\rm 3}} $上的关于$\vec{r}$的积分含$\vec{k}\cdot \vec{r}$,在球坐标下$\vec{k}\cdot \vec{r}=\left\| \vec{k}\right\| \left\| \vec{r}\right\| \cos \omega $,其中当$\theta $遍历$\left(0,\left. \pi \right]\right. $,$\varphi $遍历$\left(0,\left. 2\pi \right]\right. $,则$\omega $遍历了$\left(0,\left. \pi \right]\right. $,积分可使$\omega =\theta $而保持原值,有些书的表述是:(在积分时)不妨选$\vec{k}$与$z$轴方向相同。

\noindent $\mathrm{\triangle}$$\psi \left(\vec{r}\right)$满足归一化条件:
\[\begin{array}{rcl} {\int _{{\mathbb R}^{3} }\psi \left(\vec{r}\right)d \vec{r}} & {=} & {\frac{1}{4\pi b^{2} } \int _{{\mathbb R}^{3} }\delta \left(\left\| \vec{r}\right\| -b\right) d\vec{r}=\frac{1}{4\pi b^{2} } \int _{-\infty }^{\infty }dr_{1}  \int _{-\infty }^{\infty }dr_{2}  \int _{-\infty }^{\infty }dr_{3}  \delta \left(\sqrt{r_{1}^{2} +r_{2}^{2} +r_{3}^{2} } -b\right)} \\ {} & {=} & {\frac{1}{4\pi b^{2} } \int _{0}^{\infty }\rho ^{2}  d\rho \int _{0}^{\pi }\sin \theta  d\theta \int _{0}^{2\pi }d\varphi \delta \left(\rho -b\right) } \\ {} & {=} & {\frac{1}{4\pi b^{2} } \cdot b^{2} \cdot 2\cdot 2\pi } \\ {} & {=} & {1} \end{array}\] 
$\mathrm{\triangle}$考虑$n$个首尾相连、独立满足式(1)的分布的向量$\vec{r}_{1} ,\cdots ,\vec{r}_{n} $,记为$\left\{\vec{r}_{i} \right\}$,如此组成的链称为自由结合链(freely jointed chain)。$\vec{r}_{i} ,i=1,\cdots ,n$称为链段向量,一组特定的链段向量$\left\{\vec{r}_{i} \right\}$称为自由结合链的一个构象。

\noindent $\mathrm{\triangle}$自由结合链取构象$\left\{\vec{r}_{i} \right\}$的概率密度是$\Psi \left(\left\{\vec{r}_{i} \right\}\right)$,它是各个链节向量的联分布
\[\Psi \left(\left\{\vec{r}_{i} \right\}\right)=\prod _{i=1}^{n}\psi \left(\vec{r}_{i} \right) \] 
\[\int _{{\mathbb R}^{3n} }\Psi \left(\left\{\vec{r}_{i} \right\}\right) d\left\{\vec{r}_{i} \right\}=\prod _{i=1}^{n}\int _{{\mathbb R}^{3} }\psi \left(\vec{r}_{i} \right)d  \vec{r}_{i} =1\] 
$\mathrm{\triangle}$自由结合链在某构象$\left\{\vec{r}_{i} \right\}$下的末端位移向量$\vec{R}\left(\left\{\vec{r}_{i} \right\}\right)=\sum _{i=1}^{n}\vec{r}_{i}  $,自由结合链的末端位移向量为$\vec{R}$的概率密度是$\Phi \left(\vec{R}\right)$,它是所有$\left\{\left. \left\{\vec{r}_{i} \right\}\right|\sum _{i=1}^{n}\vec{r}_{i}  =\vec{R}\right\}$的概率之和,即
\[\begin{array}{rcl} {\Phi \left(\vec{R}\right)} & {=} & {\int _{{\mathbb R}^{3} }d\vec{r}_{1}  \cdots \int _{{\mathbb R}^{3} }d\vec{r}_{n}  \delta \left(\vec{R}-\sum _{i=1}^{n}\vec{r}_{i}  \right)\Psi \left(\left\{\vec{r}_{i} \right\}\right)} \\ {} & {=} & {\int _{{\mathbb R}^{3} }d\vec{r}_{1}  \cdots \int _{{\mathbb R}^{3} }d\vec{r}_{n}  \frac{1}{\left(2\pi \right)^{3} } \int _{{\mathbb R}^{3} }d\vec{k} \exp \left[i\vec{k}\cdot \left(\vec{R}-\sum _{i=1}^{n}\vec{r}_{i}  \right)\right]\psi \left(\vec{r}_{1} \right)\cdots \psi \left(\vec{r}_{n} \right)} \\ {} & {=} & {\frac{1}{\left(2\pi \right)^{3} } \int _{{\mathbb R}^{3} }d\vec{r}_{1}  \cdots \int _{{\mathbb R}^{3} }d\vec{r}_{n}  \int _{{\mathbb R}^{3} }e^{i\vec{k}\cdot \vec{R}} \prod _{i=1}^{n}e^{-i\vec{k}\cdot \vec{r}_{i} } \psi \left(\vec{r}_{i} \right) d\vec{k} } \\ {} & {=} & {\frac{1}{\left(2\pi \right)^{3} } \int _{{\mathbb R}^{3} }d\vec{k} e^{i\vec{k}\cdot \vec{R}} \prod _{i=1}^{n}\int _{{\mathbb R}^{3} }d\vec{r}_{i}   e^{-i\vec{k}\cdot \vec{r}_{i} } \psi \left(\vec{r}_{i} \right)} \\ {} & {=} & {\frac{1}{\left(2\pi \right)^{3} } \int _{{\mathbb R}^{3} }d\vec{k} e^{i\vec{k}\cdot \vec{R}} \left[\int _{{\mathbb R}^{3} }d\vec{r} e^{-i\vec{k}\cdot \vec{r}} \psi \left(\vec{r}\right)\right]^{n} } \end{array}\] 
其中$\begin{array}{rcl} {\int _{{\mathbb R}^{3} }d\vec{r} e^{-i\vec{k}\cdot \vec{r}} \psi \left(\vec{r}\right)} & {=} & {\frac{1}{4\pi b^{2} } \int _{0}^{\infty }\rho ^{2}  d\rho \int _{0}^{\pi }\sin \theta  d\theta \int _{0}^{2\pi }d\varphi e^{-i\left\| \vec{k}\right\| \rho \cos \theta } \delta \left(\rho -b\right) } \\ {} & {=} & {\frac{1}{4\pi b^{2} } \cdot \frac{4\pi b\sin \left(kb\right)}{k} } \\ {} & {=} & {\frac{\sin \left(kb\right)}{kb} } \end{array}$ 
\[\therefore \Phi \left(\vec{R}\right){\rm =}\frac{{\rm 1}}{\left({\rm 2}\pi \right)^{{\rm 3}} } \int _{{\mathbb R}^{3} }d\vec{k} e^{i\vec{k}\cdot \vec{R}} \left(\frac{\sin \left(kb\right)}{kb} \right)^{n} \] 
$\mathrm{\triangle}$$kb\ll 1$的近似

\noindent 由${\raise0.7ex\hbox{$ \sin x $}\!\mathord{\left/ {\vphantom {\sin x x}} \right. \kern-\nulldelimiterspace}\!\lower0.7ex\hbox{$ x $}} =1-\frac{x^{2} }{6} +o\left(x^{3} \right),x\to 0$ ,$\left[{\raise0.7ex\hbox{$ \sin \left(kb\right) $}\!\mathord{\left/ {\vphantom {\sin \left(kb\right) \left(kb\right)}} \right. \kern-\nulldelimiterspace}\!\lower0.7ex\hbox{$ \left(kb\right) $}} \right]^{n} \approx \left(1-\frac{k^{2} b^{2} }{6} \right)^{n} ,kb\ll 1$ 

\noindent 又由$\exp \left(-x\right)=1-x+o\left(x^{2} \right),x\to 0$ ,$\exp \left(-\frac{1}{6} k^{2} b^{2} \right)\approx 1-\frac{1}{6} k^{2} b^{2} ,kb\ll 1$ 
\[\therefore \left[{\sin \left(kb\right)\mathord{\left/ {\vphantom {\sin \left(kb\right) \left(kb\right)}} \right. \kern-\nulldelimiterspace} \left(kb\right)} \right]^{n} \approx \exp \left(-\frac{1}{6} nk^{2} b^{2} \right)\] 
\[\begin{array}{l} {\Phi \left(\vec{R}\right)\approx \frac{{\rm 1}}{\left({\rm 2}\pi \right)^{{\rm 3}} } \int _{{\mathbb R}^{3} }d\vec{k} e^{i\vec{k}\cdot \vec{R}} e^{-{nk^{2} b^{2} \mathord{\left/ {\vphantom {nk^{2} b^{2}  6}} \right. \kern-\nulldelimiterspace} 6} } } \\ {=\frac{{\rm 1}}{\left({\rm 2}\pi \right)^{{\rm 3}} } \prod _{i=1}^{3}\int _{-\infty }^{\infty }dk_{i}   e^{ik_{i} \cdot R_{i} -{nk_{i}^{2} b^{2} \mathord{\left/ {\vphantom {nk_{i}^{2} b^{2}  6}} \right. \kern-\nulldelimiterspace} 6} } } \\ {=\frac{{\rm 1}}{\left({\rm 2}\pi \right)^{{\rm 3}} } \prod _{i=1}^{3}\sqrt{\frac{6\pi }{nb^{2} } }  \exp \left(-\frac{3R_{i}^{2} }{2nb^{2} } \right)} \\ {=\left(\frac{3}{2\pi nb^{2} } \right)^{\frac{3}{2} } \exp \left(-\frac{3\left\| \vec{R}\right\| ^{2} }{2nb^{2} } \right)} \end{array}\] 
近似了的$\Phi \left(\vec{R}\right)$满足归一化条件。

\noindent $\mathrm{\triangle}$特别地,由近似式,$\left\langle \left\| \vec{R}\right\| \right\rangle =nb^{2} ,kb\ll 1$ 。

\noindent $\mathrm{\triangle}$$kb\ll 1$可翻译成$n\gg 1$,即当我们关心的相关尺度远大于链段时,即链长很大。

\noindent $\mathrm{\triangle}$事实上,由中心极限定理,无穷多个任意分布的随机量之和的分布是正态分布。 

\noindent \eject 

\noindent \textbf{自由结合链的正则配分函数}

\noindent $\mathrm{\triangle}$我们考虑一根由$n+1$个质量为$m$的质点连结而成的链,$\vec{R}_{i} ,\vec{p}_{i} ,i=0,\cdots ,n$分别是这些质点的坐标和动量。

\noindent $\mathrm{\triangle}$这$n+1$个质点的哈密顿量。
\[{\rm H} \left(\left\{\vec{p}_{i} ,\vec{R}_{i} \right\}\right)=\sum _{i=0}^{n}\frac{\vec{p}_{i}^{2} }{2m} +V\left(\left\{\vec{R}_{i} \right\}\right) \] 
Eduards and Goodyear (1972) J. Phys. A 5:965, 5:1188

\noindent Fokker-Planck equation of a polymer chain (FLC)

\noindent $\mathrm{\triangle}$ $V\left(\left\{\vec{R}_{i} \right\}\right)$是这一系统的势能,包括以下几种贡献

\noindent 1)高分子链的共价连接,第$i-1$与第$i$个质点间的势$u_{i} \left(\vec{R}_{i} -\vec{R}_{i-1} \right)$。假设这一势能对每对相邻质点都是相同的,且只依赖距离大小(各向同性),则$u_{i} =u\left(\left\| \vec{R}_{i} -\vec{R}_{i-1} \right\| \right),i=1,\cdots ,n$。

\noindent 2)质点之间的非共价相互作用,例如排除体积斥力(自回避链)、范德华、静电力等。

\noindent 3)外场作用。

\noindent Glatting et al. (1995) Colloid Polym. Sci. 273: 32

\noindent Canonical ensemble average of FLC

\noindent M26:6085 (1993)

\noindent Macromol. Theor. Simul. 3:575 (1993)

\noindent $\mathrm{\triangle}$第1)种是高分子链问题固有的,设2)、3)类贡献为零,$V\left(\left\{\vec{R}_{i} \right\}\right)=\sum _{i=0}^{n}u\left(\left\| \vec{R}_{i} -\vec{R}_{i-1} \right\| \right) $

\noindent $\mathrm{\triangle}$考虑该链与一个温度为${\rm T} $的大热源交换(恒温),正则配分函数
\[Z\left({\rm T} \right)=\int _{\Gamma '}d\left\{\vec{p}_{i} ,\vec{R}_{i} \right\} \exp \left(-{\raise0.7ex\hbox{$ {\rm H}  $}\!\mathord{\left/ {\vphantom {{\rm H}  \left(k_{B} {\rm T} \right)}} \right. \kern-\nulldelimiterspace}\!\lower0.7ex\hbox{$ \left(k_{B} {\rm T} \right) $}} \right)\delta \left(\vec{R}_{0} \right)\delta \left(\vec{p}_{0} \right)\] 
Yamakawa的书是overdamped limit的情况

\noindent $\Gamma '$是所有可取的状态$\left\{\vec{p}_{i} ,\vec{R}_{i} \right\}$,$\delta \left(\vec{R}_{0} \right)$和$\delta \left(\vec{p}_{0} \right)$把第一个质点设为固定在坐标原点。能这么做是因为在没有外场的情况下,${\rm H} $是空间平移不变的,记$\vec{r}_{i} \equiv \vec{R}_{i} -\vec{R}_{i-1} ,i=1,\cdots ,n$,则
\[\begin{array}{rcl} {Z\left({\rm T} \right)} & {=} & {\int _{\Gamma }d\left\{\vec{p}_{i} ,\vec{r}_{i} \right\} \exp \left(-{\raise0.7ex\hbox{$ {\rm H}  $}\!\mathord{\left/ {\vphantom {{\rm H}  \left(k_{B} {\rm T} \right)}} \right. \kern-\nulldelimiterspace}\!\lower0.7ex\hbox{$ \left(k_{B} {\rm T} \right) $}} \right)} \\ {} & {=} & {\int _{\Gamma }d\left\{\vec{p}_{i} ,\vec{r}_{i} \right\} \exp \left[-\frac{1}{k_{B} {\rm T} } \sum _{i=1}^{n}{\vec{p}_{i}^{2} \mathord{\left/ {\vphantom {\vec{p}_{i}^{2}  \left(2m\right)}} \right. \kern-\nulldelimiterspace} \left(2m\right)}  \right]\exp \left[-\frac{1}{k_{B} {\rm T} } V\left(\left\{\vec{r}_{i} \right\}\right)\right]} \\ {} & {=} & {Z_{0} \int d\vec{r}_{1} \cdots d\vec{r}_{n}  \exp \left[-\frac{1}{k_{B} {\rm T} } \sum _{i=1}^{n}u\left(\left\| \vec{r}_{i} \right\| \right) \right]} \\ {} & {=} & {Z_{0} \left(\int d\vec{r}\exp \left[-\frac{1}{k_{B} {\rm T} } u\left(\left\| \vec{r}\right\| \right)\right] \right)^{n} } \end{array}\] 
(此处$\Gamma $是所有可取的状态$\left\{\vec{p}_{i} ,\vec{r}_{i} \right\}$的集合。)(此处假设$u\left(\left\| \vec{r}_{i} \right\| \right)=u\left(\left\| \vec{r}\right\| \right)\forall i$)
\[\begin{array}{rcl} {v-Z{}_{0} } & {=} & {\int d\vec{p}_{1} \cdots d\vec{p}_{n}  \exp \left(-\frac{1}{k_{B} {\rm T} } \sum _{i=1}^{n}{\vec{p}_{i}^{2} \mathord{\left/ {\vphantom {\vec{p}_{i}^{2}  \left(2m\right)}} \right. \kern-\nulldelimiterspace} \left(2m\right)}  \right)} \\ {} & {=} & {\left(\int d\vec{p} \exp \left(-\frac{1}{k_{B} {\rm T} } {\vec{p}_{}^{2} \mathord{\left/ {\vphantom {\vec{p}_{}^{2}  \left(2m\right)}} \right. \kern-\nulldelimiterspace} \left(2m\right)} \right)\right)^{n} } \\ {} & {=} & {\left(2\pi mk_{B} {\rm T} \right)^{{3n\mathord{\left/ {\vphantom {3n 2}} \right. \kern-\nulldelimiterspace} 2} } } \end{array}\] 
$\mathrm{\triangle}$系统取任一状态$\left(\vec{p}_{i} ,\vec{r}_{i} \right)$的概率密度(Boltzmann分布)
\[\begin{array}{rcl} {p\left(\vec{p}_{i} ,\vec{r}_{i} \right)} & {=} & {Z^{-1} \exp \left(-\frac{{\rm H} \left(\vec{p}_{i} ,\vec{r}_{i} \right)}{k_{B} {\rm T} } \right)} \\ {} & {=} & {Z^{-1} \exp \left[-\frac{1}{k_{B} {\rm T} } \sum _{j}\left(\frac{\vec{p}_{j} }{2m} +u\left(\left\| \vec{r}_{j} \right\| \right)\right) \right]} \end{array}\] 
假设$u\left(\left\| \vec{r}\right\| \right)$满足简写振子势$u\left(\left\| \vec{r}\right\| \right)=\frac{1}{2} \kappa \left(\left\| \vec{r}\right\| -b\right)^{2} $,其中$\kappa $是弹簧系数,$b$是弹簧的平衡长度。该式可与$\delta $函数相联系。由狄拉克$\delta $函数的极限表示:
\[\delta \left(x-a\right)={\mathop{\lim }\limits_{\varepsilon \to 0^{+} }} \frac{1}{\sqrt{2\pi \varepsilon } } \exp \left[-\frac{1}{2\varepsilon } \left(x-a\right)^{2} \right]\] 
当$\kappa \to \infty $时,有
\[\begin{array}{rcl} {\exp \left[-\frac{1}{k_{B} {\rm T} } u\left(\left\| \vec{r}\right\| \right)\right]} & {=} & {\exp \left[-\frac{\kappa }{2k_{B} {\rm T} } \left(\left\| \vec{r}\right\| -b\right)^{2} \right]} \\ {} & {\approx } & {\sqrt{\frac{2\pi k_{B} {\rm T} }{\kappa } } \delta \left(\left\| \vec{r}\right\| -b\right)=\sqrt{\frac{2\pi k_{B} {\rm T} }{\kappa } } 4\pi b^{2} \psi \left(\vec{r}\right)} \end{array}\] 
其中$\psi \left(\vec{r}\right)=\frac{1}{4\pi b^{2} } \delta \left(\left\| \vec{r}\right\| -b\right)$是半径为$b$的球面上的均匀分布。
\[\begin{array}{l} {\therefore p\left(\vec{r}_{i} ,\vec{p}_{i} \right)=Z^{-1} \exp \left(-\frac{1}{k_{B} {\rm T} } \sum _{j}\left(\frac{\vec{p}_{j} }{2m} \right) \right)\cdot \exp \left(-\frac{1}{k_{B} {\rm T} } \sum _{k}\sqrt{\frac{2\pi k_{B} {\rm T} }{\kappa } } \delta \left(\left\| \vec{r}_{k} \right\| -b\right) \right)} \\ {=Z^{-1} \left(\frac{2\pi k_{B} {\rm T} }{\kappa } \right)^{\frac{n}{2} } \exp \left(\left(-\frac{1}{k_{B} {\rm T} } \sum _{j}\left(\frac{\vec{p}_{j} }{2m} \right) \right)\right)\exp \left(-\frac{1}{k_{B} {\rm T} } \sum _{k}\delta \left(\left\| \vec{r}_{k} \right\| -b\right) \right)} \end{array}\] 
末端距$\vec{R}=\sum _{j}\vec{r}_{j}  $的均方值
\[\begin{array}{rcl} {\left\langle \vec{R}^{2} \right\rangle } & {=} & {\int _{\Gamma }\vec{R}^{2} p\left(\vec{r}_{j} ,\vec{p}_{j} \right) d\left\{\vec{r}_{j} ,\vec{p}_{j} \right\}=\int _{\Gamma }\left(\sum _{k}\vec{r}_{k}  \right) \cdot \left(\sum _{l}\vec{r}_{l}  \right)p\left(\vec{r}_{j} ,\vec{p}_{j} \right)d\left\{\vec{r}_{j} ,\vec{p}_{j} \right\}} \\ {} & {=} & {} \end{array}\] 

\[\begin{array}{rcl} {Z\left({\rm T} \right)} & {=} & {Z_{0} \left[\int d\vec{r} \left(\frac{2\pi k_{B} {\rm T} }{\kappa } \right)^{\frac{1}{2} } 4\pi b^{2} \psi \left(\vec{r}\right)\right]^{n} } \\ {} & {=} & {Z_{0} \left(\frac{2\pi k_{B} {\rm T} }{\kappa } \right)^{\frac{n}{2} } \left(4\pi b^{2} \right)^{n} } \end{array}\] 


\noindent 

\noindent 


\end{document}

