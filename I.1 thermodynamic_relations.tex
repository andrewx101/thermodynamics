\documentclass[main.tex]{subfiles}
\begin{document}
\subsection{化学势的引入}
热力学能$U$作为特性函数\footnote{请回顾《物理化学》\S 3.13、\S 4.4。},是以$\left(S, V,\left\{n_i\right\}\right)$为特征变量的函数,$U=U\left(S,V,\left\{n_i\right\}\right)$,其全微分是
\[dU=\left.\frac{\partial U}{\partial S}\right|_{V,\left\{n_i\right\}}dS+\left.\frac{\partial U}{\partial V}\right|_{S,\left\{n_i\right\}}dV+\sum_i\left.\frac{\partial U}{\partial n_i}\right|_{S,V,\left\{n_{j\neq i}\right\}}dn_i\]
其中$\left\{n_i\right\}$表示$n_1,n_2,\cdots$。与由第一、二定律得到的式子$dU=TdS-pdV+\sum_i\mu_idn_i$(可逆过程)比较得
\[T=\left.\frac{\partial U}{\partial S}\right|_{V,\left\{n_i\right\}},\quad-p=\left.\frac{\partial U}{\partial V}\right|_{S,\left\{n_i\right\}},\quad\mu_i\equiv\left.\frac{\partial U}{\partial n_i}\right|_{S,V,\left\{n_{j\neq i}\right\}}\]
其中$\mu_i$是此处引入的,表示由于组份$i$的分子数变化$dn_i$所造成的混合物体系热力学能的变化,称为组份$i$在混合物中的\CJKunderdot{化学势}(chemical potential)。

化学势恰好同时是其他几个特性函数对组份的偏导数:
\[\mu_i=\left.\frac{\partial U}{\partial n_i}\right|_{S,V,\left\{n_{j\neq i}\right\}}=\left.\frac{\partial H}{\partial n_i}\right|_{S,p,\left\{n_{j\neq i}\right\}}=\left.\frac{\partial A}{\partial n_i}\right|_{T,V,\left\{n_{j\neq i}\right\}}=\left.\frac{\partial G}{\partial n_i}\right|_{T,p,\left\{n_{j\neq i}\right\}}\]






\section{恒温恒压开放体系的一般热力学关系}
由条件$\left\{T,p,\left\{n_i\right\}\right\}$约束的体系(恒温恒压开放体系)的热力学势是吉布斯自由能。由定义:
\[G\equiv U-TS+pV\]
其全微分
\begin{align*}
dG&=dU-TdS-SdT+pdV+Vdp&\text{(}&\text{由定义式)}\\
&=TdS-pdV+\sum_i\mu_idn_i-TdS-SdT+pdV+Vdp&\text{(}&\text{代入$dU$)}\\
&=-SdT+Vdp+\sum_i\mu_idn_i\\
&=\left.\frac{\partial G}{\partial T}\right|_{p,\left\{n_i\right\}}dT+\left.\frac{\partial G}{\partial p}\right|_{T,\left\{n_i\right\}}dp+\sum_i\left.\frac{\partial G}{\partial n_i}\right|{T,p,\left\{n_{j\neq i}\right\}}dn_i&\text{(}&\text{由$G=G\left(T,p,\left\{n_i\right\}\right)$)}\\
\Rightarrow -S&=\left.\frac{\partial G}{\partial T}\right|_{p,\left\{n_i\right\}},\quad V=\left.\frac{\partial G}{\partial p}\right|_{T,\left\{n_i\right\}},\quad\mu_i=\left.\frac{\partial G}{\partial n_i}\right|_{T,p,\left\{n_{j\neq i}\right\}}
\end{align*}
进一步可给出以下Maxwell关系:
\begin{align*}
\left.\frac{\partial S}{\partial p}\right|_{T,\left\{n_i\right\}}&=-\left.\frac{\partial V}{\partial T}\right|_{p,\left\{n_i\right\}},&\left.\frac{\partial S}{\partial n_i}\right|_{T,p,\left\{n_{j\neq i}\right\}}&=-\left.\frac{\partial\mu_i}{\partial T}\right|_{p,\left\{n_i\right\}}\\
\left.\frac{\partial V}{\partial n_i}\right|_{T,p,\left\{n_{j\neq i}\right\}}&=\left.\frac{\partial \mu_i}{\partial p}\right|_{T,\left\{n_i\right\}},&\left.\frac{\partial\mu_i}{\partial n_j}\right|_{T,p,\left\{n_{k\neq j}\right\}}&=\left.\frac{\partial\mu_j}{\partial n_i}\right|_{T,p,\left\{n_{k\neq i}\right\}}
\end{align*}
若定义
\[V_i\equiv\left.\frac{\partial V}{\partial n_i}\right|_{T,p,\left\{n_{j\neq i}\right\}},\quad S_i\equiv\left.\frac{\partial S}{\partial n_i}\right|_{T,p,\left\{n_{j\neq i}\right\}}\]
分别为体系中组分$i$在约束条件$\left(T,p,\left\{n_i\right\}\right)$平衡态下的\CJKunderdot{偏摩尔体积}(partial molar volume)和\CJKunderdot{偏摩尔熵}(partial molar entropy),则相关Maxwell关系可表示为
\[V_i=\left.\frac{\partial\mu_i}{\partial p}\right|_{T,\left\{n_i\right\}},\quad S_i=\left.\frac{\partial\mu_i}{\partial T}\right|_{p,\left\{n_i\right\}}\]
有了上述两个偏摩尔量的定义,可以进一步考虑体系在平衡态下组份$i$的化学势全微分
\begin{align*}
d\mu_i&=\left.\frac{\partial\mu_i}{\partial T}\right|_{p,\left\{n_i\right\}}dT+\left.\frac{\partial\mu_i}{\partial p}\right|_{T,\left\{n_i\right\}}+\sum_j\left.\frac{\partial\mu_i}{\partial n_j}\right|_{T,p,\left\{n_{k\neq j}\right\}}dn_j\\
&=-S_idT+V_idp+\sum_j\left.\frac{\partial\mu_i}{\partial n_j}\right|_{T,p,\left\{n_{k\neq j}\right\}}dn_j
\end{align*}
\end{document}