\documentclass[main.tex]{subfiles}
\begin{document}
\subsection{偏离函数}
在《物理化学》课本中我们已经接受,凡状态函数$M$,决定其的变量就是决定体系状态的变量\footnote{p.81。}。对于所谓的“简单的系统”\footnote{这里“简单的系统”宜参考\cite[p.9]{Callen1985}:“systems that are macroscopically homogeneous, isotropic, and uncharged, that are large enough so that surface effects can be neglected, and that are not acted on by electric, magnetic, or gravitational fields.”或参考\cite[p.17]{王竹溪1960}中的概念体系,定义为仅用几何、力学和化学变数描写的单相系,且只有单相系才有物态方程。},在$p$,$V$,$T$中任选两个独立变量,再加上组成(比如用各组份物质的量表示$\left\{n_i\right\}$),就可以决定系统的状态。能说只需“$p$、$V$、$T$任选两个”,是因为简单体系的状态方程形式总能写成$f=f\left(T,p,V,\left\{n_i\right\}\right)$的形式。除组份$\left\{n_i\right\}$外,$T$,$p$,$V$定了任意两个的值,第三个量理论上就作为状态方程的解而被确定,体系的状态亦被确定。

由于热力学能和熵是广度性质,热力学基本理论中由它们经数学关系衍生出来的状态函数也都是广度性质。按广度性质的定义,状态函数又必是一次齐函数\cite[p.75]{傅献彩2022I}。若记体系的(总)摩尔数$n\equiv\sum_i n_i$,其中$n_i$表示体系中组份$i$的摩尔数,则任一状态函数$M$都有其平均摩尔量(简称摩尔量),$M_\text{m}\equiv M/n$,其中,正体下标“m”加在表示原广度性质状态函数的字母$M$上,就表示相应广度性质状态函数的摩尔量。由定义,摩尔量都是强度性质。以下我们谈到一种性质$M$时,常直接讨论其摩尔量$M_\text{m}$。

在热力学理论中,我们往往无法知道状态函数的表达式。这很大程度是来自,体系的状态方程是事物的特殊性,不同类型的体系,状态方程的形式不同,更何况所谓状态方程只是对真实事物的数学抽象。真实事物的性质只能通过实验测量来获知,而许多状态函数又不能直接测量。为了使问题可以被仔细讨论,我们常常先举出一种模型体系作为参照,把真实体系的性质表示为对模型体系的偏离。具体地,选定某模型(model),为所研究的实际(actual)体系的任一状态函数$M_\text{m}$,皆可定义相应的偏离函数(deviation function)
\[M^\text{dev}_\text{m}\equiv M^\text{act}_\text{m}-M^\text{mod}_\text{m}\]
其中$M^\text{act}_\text{m}$和$M^\text{mod}_\text{m}$分别是相同状态下真实体系与模型体系的性质。

这个模型体系的选择原则上是任意的,但为了使理论便于应用,应作如下考虑。首先,所选择模型体系应具有清晰的微观机理背景,以便我们可以把实际体系对其的偏离作出一种微观物理意义层面的解读。其次,所选择的模型体系应是真实体系在某种明确定义概念下的“极限”。也就是说,要能说得出,到底在哪些条件、如何逐渐变化的过程中,任何真实体系就趋于这一模型体系。如果一个模型体系是真实体系无论如何都达不到的,那它的意义也不大。最后,模型体系本身的状态方程应该具有明确的且尽可能简单的数学形式。我们将会看到,理想气体是其中一种比较好的选择。

偏离函数应拿同一状态下真实体系与模型体系的性质比较得到。这里的“同一状态”,就是指确定体系状态的变量取值相等。我们经常讨论的两组同样都能确定体系状态的变量是:$\left\{T,V_\text{m},\left\{n_i\right\}\right\}$和$\left\{T,p,\left\{n_i\right\}\right\}$。我们具体记
\begin{equation}\label{eq:def_deviation_function_D}
    M^\text{D}_\text{m}\left(T,p,\left\{n_i\right\}\right)\equiv M_\text{m}\left(T,p,\left\{n_i\right\}\right)-M^\text{mod}_\text{m}\left(T,p,\left\{n_i\right\}\right)
\end{equation}
和
\begin{equation}\label{eq:def_deviation_function_d}
    M^\text{d}_\text{m}\left(T,V_\text{m},\left\{n_i\right\}\right)\equiv M_\text{m}\left(T,V_\text{m},\left\{n_i\right\}\right)-M^\text{mod}_\text{m}\left(T,V_\text{m},\left\{n_i\right\}\right)
\end{equation}
以区别这两组变量确定状态的情况。有时我们把$V_\text{m}$换成摩尔密度$\rho\equiv V^{-1}_\text{m}$,仍属于第二种情况。

这两种偏离函数之间的关系是不难推出的。假设在$\left(T,V_\text{m},\left\{n_i\right\}\right)$下,我们所关心的实际体系压强是$p$,则可试求
\[M^\text{d}_\text{m}\left(T,V_\text{m},\left\{n_i\right\}\right)-M^\text{D}_\text{m}\left(T,p,\left\{n_i\right\}\right)=M^\text{mod}_\text{m}\left(T,p,\left\{n_i\right\}\right)-M^\text{mod}_\text{m}\left(T,V_\text{m},\left\{n_i\right\}\right)\]
由于真实体系和模型体系一般不同,既然在$\left(T,V_\text{m},\left\{n_i\right\}\right)$和$\left(T,p,\left\{n_i\right\}\right)$下这一真实体系处于相同的状态,则对模型体系未必是,故上式等号右边不为零。由微积分原理,它可写成
\[M^\text{mod}_\text{m}\left(T,p,\left\{n_i\right\}\right)-M^\text{mod}_\text{m}\left(T,V_\text{m},\left\{n_i\right\}\right)=\int_{p^*}^p\left.\frac{\partial M^\text{mod}_\text{m}\left(T,p^\prime,\left\{n_i\right\}\right)}{\partial p^\prime}\right|_{T,\left\{n_i\right\}}\mathrm{d}p^\prime\]
其中$p^*$是使模型体系在条件$T$、$\left\{n_i\right\}$下摩尔体积为$V_\text{m}$的压强。因此有
\begin{equation}\label{eq:rel_M~D_M~d}
    M^\text{d}_\text{m}=M^\text{D}_\text{m}+\int_{p^*}^p\left.\frac{\partial M^\text{mod}_\text{m}}{\partial p^\prime}\right|_{T,\left\{n_i\right\}}\mathrm{d}p^\prime
\end{equation}

\subsection{残余函数}
选用理想气体为模型体系(mod=ig)构建的偏离函数特称为残余函数(residual functions),它们用上标R或r表示。
\begin{equation}\label{eq:def_residual_function_R}
    M^\text{R}_\text{m}\left(T,p,\left\{n_i\right\}\right)\equiv M_\text{m}\left(T,p,\left\{n_i\right\}\right)-M^\text{ig}\left(T,p,\left\{n_i\right\}\right)
\end{equation}
\begin{equation}\label{eq:def_residual_function_r}
    M^\text{R}_\text{m}\left(T,p,\left\{n_i\right\}\right)\equiv M_\text{m}\left(T,p,\left\{n_i\right\}\right)-M^\text{ig}\left(T,p,\left\{n_i\right\}\right)
\end{equation}
利用式\eqref{eq:rel_M~D_M~d},对于处于$\left(p,V_\text{m},T\right)$下的某真实体系,有
\begin{equation}
    \label{eq:rel_M~R_M~r}
    M^\text{r}_\text{m}=M^\text{R}+\int_{RT/V_\text{m}}^p\left.\frac{\partial M^\text{ig}_\text{m}}{\partial p^\prime}\right|_{T,\left\{n_i\right\}}\mathrm{d}p^\prime
\end{equation}

理想气体的内能、焓、等容热容和等压热容均不依赖压强,因此这些量的摩尔量的相应两种残余函数之间是相等的(即式\eqref{eq:rel_M~R_M~r}的积分为零)。但是理想气体的熵、亥姆霍兹自由能和吉布斯自由能是压强的函数,具体地
\begin{align*}
    \left.\frac{\partial S^\text{ig}_\text{m}}{\partial p}\right|_{T,\left\{n_i\right\}} & =-\frac{R}{p}                                                                                      \\
    \left.\frac{\partial A^\text{ig}_\text{m}}{\partial p}\right|_{T,\left\{n_i\right\}} & =\left.\frac{\partial G^\text{ig}_\text{m}}{\partial p}\right|_{T,\left\{n_i\right\}}=\frac{RT}{p}
\end{align*}
把这些代入式\eqref{eq:rel_M~R_M~r}得
\begin{align}
    S^\text{r}_\text{m} & =S^\text{R}_\text{m}-R\ln Z  \\
    A^\text{r}_\text{m} & =A^\text{R}_\text{m}+RT\ln Z \\
    G^\text{r}_\text{m} & =G^\text{R}_\text{m}+RT\ln Z
\end{align}
其中$Z\equiv pV_\text{m}/\left(RT\right)$是体系的压缩因子。
\end{document}