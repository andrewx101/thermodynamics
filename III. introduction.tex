\documentclass[main.tex]{subfiles}
\begin{document}
本章虽冠以“非电解质溶液的热力学”,但此称谓主要出于习惯。热力学理论只能一般地规定热力学函数是哪些状态变量的函数,而无法给出具体的函数表达形式。后者随系统本身的物理化学特性而变化,需要从微观模型出发来建模完成,而这是统计力学的任务。

对于液体混合物(或称溶液)体系,最核心的热力学函数是混合Gibbs自由能$\Delta_\text{mix} G$(由式\eqref{eq:II.4_def_mixing_function}定义,详见\S\ref{sec:II.4.6 mixing_function})。一旦获得$\Delta_\text{mix} G$,诸多应用便可顺理成章地推出,例如渗透压、化学势与相平衡判据等;相关内容已在\S\ref{sec:II.5 typical_problems}介绍。

值得说明的是,\S\ref{sec:II.5 typical_problems}貌似给出了各类问题的一些具体表达式,但是它们都是在特定的限制条件下的。例如,在\S\ref{sec:II.5.2 vapor-liquid_equilibrium}“气液共存”中,最一般地我们只能得到逸度关系\eqref{eq:II.5_vapor-liquid_equilibrium_fugacity},而由逸度的定义,它无非是化学势的另一种表达方式。后面最终得到“可以用来做题”的拉乌尔定律,包括了“液相是不可压缩的理想液体混合物”以及“气相是理想气体混合物”的假定。实际上,这些理想混合物,无非是抹去了体系的化学特异性的理想体系。一旦化学特异性被引入,实际行为不可能是普适的,而必然依赖化学变量。基本热力学关系是不负责给出化学特异性的,这需要从引入化学特异性后的微观模型出发,依赖统计力学来完成。在\S\ref{sec:II.5.3 non_volatile_solute}和\S\ref{sec:II.5.4_semipermeable_membrane}中,依数性和渗透压的可用表达式,都是极稀溶液的情况,也就是溶质-溶质相互作用可以忽略的情况,这也等于抹去了溶质分子的化学特异性。总之,热力学理论是负责给出普适关系的。因此,它要么无法解释具体体系化学特异性造成的行为差异,要么就是在化学特异性不明显的限定条件或理想情况下给出仍然普适但具体明显的表达式。

用于构建凝聚态物质理论的统计力学大致可以分为两种出发点:格子模型和连续空间模型。所谓连续空间的模型,就是假定粒子可以在3维空间内任意分布。而格子模型则规定粒子只能放入某个网格(lattice)的格子或位点(cells/sites)中。这个网格可能是在$d$维空间上构建的,并且规定了每个位点周边直接相邻的位点数——称为\emph{配位数(coordination number)}$z$。例如,一个3维空间中的立方结构网格(参照氯化钠晶体的结构)的配位数就是6。

所谓“非电解质”的设定,实际意思是排除了分子或原子间的静电相互作用势。这是一种作用距离比较长的相互作用。我们将会看到,本章只近似考虑最邻近两分子间的相互作用势,如果有静电相互作用,这是不够的。原则上,我们还应加上“非极性”的限定。具有极性的分子间的相互作用势有分子方向的依赖性(或称各向异性),这也是本章的介绍中没有考虑的。最终本章的模型讨论最适用于非极性分子间的色散力相互作用势;它的作用距离越不过一个分子的大小,而且没有分子方向依赖性。

本章将仅基于格子模型来介绍。以下列出一些连续空间模型的参考书。这类统计力学理论用于高分子体系又分为两类。一类是基于粒子的统计力学,不管是小分子,还是高分子链段,都明确视为可数的质点。另一类是基于场论的统计力学,视高分子链为连续曲线。前者的代表性著作有(仅限于溶液的平衡态统计)——
\begin{itemize}
    \item 孙民华, 牛丽. 液态物理概论[M]. 北京:科学出版社, 2013
    \item S. Rice, P. Gray (1965), \emph{The Statistical Mechanics of Simple Liquids}, Interscience
    \item J.-P. Hansen, I. McDonlad (2013), \emph{Theory of Simple Liquids, 4th ed.}, Academic Press
    \item 山川裕巳/Hiromi Yamakawa (1971), \emph{Modern Theory of Polymer Solutions}, Harper \& Row
\end{itemize}
后者可参考——
\begin{itemize}
    \item H. Kleinert (1995), \emph{Path Integrals in Quantum Mechanics, Statistics, and Polymer Physics, 2nd ed.}, World Scientific
    \item 藤田博/Hiroshi Fujita (1990), \emph{Polymer Solutions}, Elsevier
\end{itemize}
\end{document}