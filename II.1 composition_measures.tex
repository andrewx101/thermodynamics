\documentclass[main.tex]{subfiles}
\begin{document}
本节介绍三种刻划混合物组成的方法。考虑一个均一的混合物体系,记$m_i$、$n_i$分别是组份$i$在体系中的质量和物质的量,由可定义以下三种强度性质:

质量分数(mass fraction):$w_i\eqdef\frac{m_i}{\sum_jm_j}$

摩尔分数(mole fraction):$x_i\eqdef\frac{n_i}{\sum_jn_j}$

浓度(concentration):$c_i\eqdef\frac{n_i}{V}$

一般地,完整确定体系的组成需要所有组份的质量或物质的量,记为$\left\{m_i\right\}=\left\{m_1,m_2,\cdots\right\}$和$\left\{n_i\right\}=\left\{n_1,n_2,\cdots\right\}$。

如果采用$\left\{w_i\right\}$或$\left\{x_i\right\}$来表示组成,虽然利用恒等关系
\[\sum_ix_i=\sum_iw_i=1\]
似乎可比使用$\left\{n_i\right\}$或$\left\{m_i\right\}$少用一个变量来确定体系的组成,但计算$x_i$或$w_i$所需要的分母$\sum_in_i$和$\sum_im_i$本身就要求所有$\left\{n_i\right\}$和$\left\{m_i\right\}$,因此独立、完整地确定混合物体系组成的变量个数是一定的。

浓度定义中的$V$是混合物体系的体积。它本身又依赖体系的状态,由该体系的状态方程来主导其变化规律。因此浓度也是依赖体系状态的,不是独立反映体系组成的量,因此在热力学理论叙述中不常采用。但是对于远离临界点的凝聚态,体积随温度、压强的变化一般不大,所以在实验上使用广泛。

若记组份$i$的摩尔质量(molar mass)\footnote{即平时说的“分子量”,但我们考虑的组份的微观最小单元未必是分子。}为
\[M_{\text{w},i}\eqdef\frac{n_i}{m_i}\]
则易知$x_i$与$w_i$之间有如下关系
\[w_i=x_i\frac{M_i}{\overline{M_\text{w}}}\]
其中
\[\overline{M_\text{w}}\equiv\frac{\sum_jn_jM_{\text{w},j}}{\sum_jn_j}=\frac{\sum_jm_j}{\sum_jn_j}\]
是混合物的平均摩尔质量。

在上述讨论中,我们并不明确体系所含组份种类的个数。这是考虑到开放体系不仅各组份的量可能会变化,就连组份种类数量也可能会变化。
\end{document}