\documentclass[main.tex]{subfiles}
\begin{document}
本节介绍三种刻划混合物的\emph{组成(composition)}的方法。考虑一个均一的混合物系统,记$m_i$、$n_i$分别是组份$i$在系统中的质量和物质的量,由可定义以下三种强度性质:

\emph{质量分数(mass fraction)}:
\[w_i\eqdef\frac{m_i}{\sum_jm_j}\]

\emph{摩尔分数(mole fraction)}:
\[x_i\eqdef\frac{n_i}{\sum_jn_j}\]

\emph{浓度(concentration)}:
\[c_i\eqdef\frac{n_i}{V}\]

一般地,完整确定系统的组成需要所有组份的质量或物质的量,记为$\left\{m_i\right\}=\left\{m_1,m_2,\cdots\right\}$和$\left\{n_i\right\}=\left\{n_1,n_2,\cdots\right\}$。

如果采用$\left\{w_i\right\}$或$\left\{x_i\right\}$来表示组成,虽然利用恒等关系
\[\sum_ix_i=\sum_iw_i=1\]
似乎可比使用$\left\{n_i\right\}$或$\left\{m_i\right\}$少用一个变量来确定系统的组成,但计算$x_i$或$w_i$所需要的分母$\sum_in_i$和$\sum_im_i$本身就要求所有$\left\{n_i\right\}$和$\left\{m_i\right\}$,因此独立、完整地确定混合物系统组成的变量个数仍是一定的。

注意到,对于给定的一个混合物系统,其各组份的摩尔数$\left\{n_i\right\}$可视为摩尔份数$\left\{x_i\right\}$恒定下的广度性质,因为它们在这种条件下正比于系统的(总)摩尔数$n$,即$n_i=x_in$。因此$x_i$就是与$n_i$对应的强度性质。

给定一个混合物系统的状态函数$M$,我们经常写成$M=M\left(X,Y,\left\{n_i\right\}\right)$,其中$X$、$Y$是除组成外的其他状态参数。但是在讨论中,我们又经常考虑某组份$i$的摩尔分数$x_i$在保持总摩尔数$n$恒定(封闭系统)时,取值的连续变化过程,而不对其他组份的摩尔数$\left\{n_{j\neq i}\right\}$的变化方式作出规定。诚然,满足这一要求的组成变化路径是有无数条的。在理论探讨中,我们仅需要明确,状态函数的取值不依赖历史路径的选择,且总有一条可逆的路径使得式\eqref{eq:I.1_integral_of_function}得以适用即可。因此,在本讲义和很多其他资料中,都不会仔细区分$M\left(T,p,\left\{n_i\right\}\right)$和$M\left(T,p,\left\{n;x_i\right\}\right)$这两种写法,例如会出现类似以下的表达式:
\[\int_a^bM\left(T,p,\left\{n_i\right\}\right)\mathrm{d}x_i\]
上式表示,找到组份$i$的摩尔分数$x_i$由$a$连续变化至$b$的某条可逆路径来求这个积分。这不是一个简单的定积分,而是一个在组成空间中某条规定路径上的曲线积分。所幸的是我们几乎不可能被要求具体计算这样一个积分。不限制组份数,且坚持使用数学语言,是为了保证理论构建的严格性和一般性。

浓度定义中的$V$是混合物系统的体积。它本身又依赖系统的状态,由该系统的$pVTn_i$状态方程来主导其变化规律。因此给定组成的混合物系统的浓度仍然依赖该系统的状态,不是独立反映系统组成的量,所以在热力学理论叙述中不常采用。但是对于远离临界点的凝聚态,体积随温度、压强的变化一般不大,所以在实验中广泛使用。

若记组份$i$的\emph{摩尔质量(molar mass)}\footnote{即平时说的“分子量”,但我们考虑的组份的微观最小单元未必是分子。}为
\[M_{\text{w},i}\eqdef\frac{m_i}{n_i}\]
则易知$x_i$与$w_i$之间有如下关系
\[w_i=x_i\frac{M_i}{\overline{M_\text{w}}}\]
其中
\[\overline{M_\text{w}}\equiv\frac{\sum_jn_jM_{\text{w},j}}{\sum_jn_j}=\frac{\sum_jm_j}{\sum_jn_j}\]
是混合物的平均摩尔质量。混合物组成与组分的摩尔质量的关系是利用稀溶液依数性测量分子量的基础。

在上述讨论中,我们并不明确系统所含组份种类的个数。这是一般地考虑到开放系统或者有化学反应发生的系统中不仅各组份的量可能会变化,就连组份种类数量也可能会变化。
\end{document}