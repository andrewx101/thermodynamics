\documentclass[main.tex]{subfiles}
\begin{document}
\section{热力学的学习建议}
仅靠在本科阶段的物理化学课堂学习是很难一次性把热力学学透彻的。

首先,想要顺利地学进去,很依赖对热力学最基本的概念和思想——尤其是导出热力学第一定律之前的准备性概念——的准确把握。而准确理解这些概念和思想是需要反复地反思和质疑,最好能批判地回顾热力学的历史,以便理解今天所呈现的抽象版本它从表象到本质的变化过程。这是一个时间不短的学习过程。

准确理解热力学看待问题的一般方法之后,第二个难关是物理上的体系(body)与过程(path)与数学的多元函数微积分语言的紧密结合。大部分同学在本科阶段往往只从头到尾学习了一次高等数学。对多元函数的微积分的概念的理解可能都还不准确、不熟练,面对热力学关系中大量出现的微分关系和偏导数关系来不及掌握,哪怕为“记公式”,也不知道何时使用、如何使用。

第三个难关是,灵活应用热力学去处理科学研究或工业生产中的问题。

我建议同学们分清这三个阶段,认识到热力学学习之路的悠远,不必急于一蹴而就。热力学是统领大半个物理学的基本理论,它值得任何一名未来的科技工作者花一生的时间去掌握,因此不必强求学习进度的快慢。以下我将分几个方面介绍一些写得好、易于自学的参考资料。

一、热力学思想和概念(到热力学第二定律为止)

以下章节可供读者在有限时间内,不太发散地集中学习这方面内容:
\begin{itemize}
    \item 王竹溪. 热力学[M]. 第二版. 高等教育出版社, 1960. 绪论、第一至三章
    \item 韩德刚 高执棣 高盘良. 物理化学[M]. 第二版. 高等教育出版社, 2008. 第一章
\end{itemize}
以下资料可供本科层次读者细致地了解热力学理论的历史脉络:
\begin{itemize}
    \item Fowler, Michael. Physics 152: Heat and Thermodynamics[EB/OL]. [2024/8/12]. https://galileo.phys.virginia.edu/classes/152.mf1i.spring02/HeatIndex.htm.
    \item Müller, I. A History of Thermodynamics[M]. Springer, 2007.
\end{itemize}
以下资料可供读者在初步完整地学习完热力学之后温故而知新:
\begin{itemize}
    \item 王竹溪. 热力学[M]. 第二版. 高等教育出版社, 1960. 全书
    \item Callen, Herbert B. Thermodynamics and an introduction to thermostatistics[M]. 2nd ed. Wiley, 1985. Part I
    \item Reichl, E L. A Modern Course in Statistical Physics[M]. 2nd ed. University of Texas Press, 1980. Ch.1$\sim$4
\end{itemize}

二、复习高等数学中函数微积分部分的内容,并用这些知识去理解物理化学或热力学课本当中出现的数学式子和推导过程。这部分资料很多,不一一罗列了。在数学过关的基础上,亲自推导物理化学或热力学课本上的全部微分和偏导数关系式,最好重复若干次。

三、想要感受和学习热力学的实践应用,可以参考以“化工热力学”为标题的教材,并重点留意处理真实体系的内容。
\begin{itemize}
    \item Smith, J. M.; Van Ness, H. C.; Abbott, M. M. Introduction to Chemical Engineering Thermodynamics[M]. 9th ed. McGraw Hill, 2022.
\end{itemize}
除了一般的化工热力学教材之外,还可以从一些热力学在你关心领域的应用专著上感受,基础课本上的热力学如何推广到更加广泛的问题中去的。不同体系的差别,不在热力学基本理论的差别,而在于状态方程的差别。而状态方程要么是经验公式,要么来自微观统计理论,所以免不了要额外学习一些统计力学。但到了应用层面,我们处理的还是宏观体系,只要体系的状态方程在手了就行。以下介绍各领域热力学特色显著的书。溶液热力学:
\begin{itemize}
    \item Van Ness, H. C. Classical Thermodynamics of Non-Electrolyte Solutions[M]. Pergamon Press, 1964.
    \item Van Dijk, M.; Wakker, A. Concepts of Polymer Thermodynamics[M]. ChemTec Publishing, 1997
    \item Klenin, V. J. Thermodynamics of Systems Containing Flexible-Chain Polymers[M]. Elsevier, 1999
    \item Flory, P. J. Principles of Polymer Chemistry[M]. Cornell University Press, 1953
\end{itemize}

最后,我列出几条一般的学习建议,能够解决学习过程中常见的困惑。
\begin{itemize}
    \item 区分体系与环境。在讨论过程中要注意保持讨论对象不变。在有些问题当中我们不止讨论一个体系,或者会变换所讨论的体系(比如相平衡问题)。在这种情况下就更要敏感和清醒,每句陈述是视什么为体系,视什么为环境。
    \item 认清体系与环境之间的关系,即:有无物质交换和能量交换。
    \item 认真理解各类“过程”(可逆、不可逆、准静态、自发……)。任何时候都能立刻回答:当前式子用到热力学第二律了没有?用到第二定律的式子,为什么写成等式?什么时候关心变化量$\Delta M$,什么时候关心微分$\mathrm{d}M$?它们之间有什么关系?
    \item 为物理化学课本上的微分关系式推导过程补全至高等数学课所要求的仔细程度,并重视在这件事上碰到的疑问,必须努力得到解答。高等数学本身比较弱的同学,不要急着开始学习热力学,先复习和加强高等数学。我们最终要习惯不依赖文字,只通过数学式子听故事,只写数学式子讲故事。
\end{itemize}
这些学习建议只涉及到热力学头几章的内容。如果在学习这些基本内容的过程当中,还没达到上述建议的要求,就不要急着往后学。

\section{热力学是怎么应用的}
一般地,某体系的性质$M$若是状态函数,那它就是独立、完整地确定体系状态的一组变量$\left(X,Y,\cdots\right)$的函数$M=M\left(X,Y,\cdots\right)$。如果$\mathrm{d}M$是函数$M$的全微分,则应有
\[\mathrm{d}M=\left.\frac{\partial M}{\partial X}\right|_{Y,\cdots}\mathrm{d}X+\left.\frac{\partial M}{\partial Y}\right|_{X,\cdots}\mathrm{d}Y+\cdots\]
反之,如果告诉你$\mathrm{d}M$在最一般的情况下能写成上式,意思就是说$\left(X,Y,\cdots\right)$独立、完整地确定函数$M$的值,如果$M$还是状态函数且是体系的平衡态某性质,那就是说这个体系的状态独立地由$\left(X,Y,\cdots\right)$所确定,那么这个体系的其他状态函数性质也以$\left(X,Y,\cdots\right)$为自变量。独立、完整确定体系状态的变量可以不止一组。例如,说一个体系的状态可独立、完整地由温度$T$和压强$p$确定,那它也可以独立、完整地由温度$T$和体积$V$确定,因为这个体系的压强、温度和体积由状态方程所联系,所以不是三个量都互相独立,确定了两个就同时确定了第三个。例如,确定了$\left(T,p\right)$就同时确定了$V$,即$V=V\left(T,p\right)$,是体系状态方程的一种表达形式。若说$M=M\left(T,V\right)$,则理论上总能通过把$V=V\left(T,p\right)$代进去,写出$M=M\left(T,V\left(T,p\right)\right)=M\left(T,p\right)$的形式。但若说$M=M\left(T\right)$,那就叫“不完整”;若说$M=M\left(T,p,V\right)$,那就叫“不独立”。有的书用状态参数(state parameters)来称呼这样的一组变量。

经验表明,独立、完整地确定一个多组份混合物体系平衡状态的变量,除在相同前提下的单组份体系所需要的那些外,还需要增加各组份的摩尔数$n_1,n_2,\cdots$,简记为$\left\{n_i\right\}$。例如,一个不受外场(重力、电、磁等)作用,且接触力作用只有各向同性静压(即所谓的“只做体积功”)的单组分体系,摩尔数$n$一定时,它的状态可用温度$T$、压强$p$、体积$V$三个变量中的两个所确定。而对于总摩尔数$n\equiv\sum_in_i$一定的多组分体系,则还要在单组份情况的基础上加上$\left\{n_i\right\}$才能确定其状态。

一个体系最基本的状态函数性质就是内能$U$和熵$S$。它们的引入可详见其他热力学教材。对于单组份体系,由热力学第一定律
\[\mathrm{d}U=\dbar Q+\dbar W\]
和第二定律
\[\mathrm{d} S=\frac{\dbar Q_\text{rev}}{T}\]
给出,在可逆过程中
\[\mathrm{d}U=T\mathrm{d}S+\mathrm{d}W_\text{rev}\]
在只做体积功的情况下,
\[\mathrm{d}U=T\mathrm{d}S-p\mathrm{d}V\]
由于内能$U$是状态量,上列式子又是普适定律,故上列式子可视为$U$的完整全微分,暗示$U=U\left(S,V\right)$,即$\left(S,V\right)$独立、完整地确定这种体系的状态,具体有
\begin{align*}
    \mathrm{d}U                                                   & =\left.\frac{\partial U}{\partial S}\right|_V\mathrm{d}S+\left.\frac{\partial U}{\partial V}\right|_{S}\mathrm{d}V \\
                                                                  & =T\mathrm{d}S-p\mathrm{d}V                                                                                         \\
    \Leftrightarrow\left.\frac{\partial U}{\partial S}\right|_{V} & =T,\quad\left.\frac{\partial U}{\partial V}\right|_{S}=-p
\end{align*}

对于多组份体系,确定体系状态的变量新增$\left\{n_i\right\}$,即$U=U\left(S,V,\left\{n_i\right\}\right)$,故
\begin{equation}\label{eq:I.1_first_law_multicomp}
    \mathrm{d}U=\left.\frac{\partial U}{\partial S}\right|_{V,\left\{n_i\right\}}\mathrm{d}S+\left.\frac{\partial U}{\partial V}\right|_{S,\left\{n_i\right\}}\mathrm{d}V+\sum_i\left.\frac{\partial U}{\partial n_i}\right|_{S,V,\left\{n_{j\neq i}\right\}}\mathrm{d}n_i
\end{equation}
若保持该体系组成恒定,又可视体系相当于一个不区分组份种类的单组份体系,故仍有
\[\left.\frac{\partial U}{\partial S}\right|_{V,\left\{n_i\right\}}=T,\quad\left.\frac{\partial U}{\partial V}\right|_{S,\left\{n_i\right\}}=-p\]
而新引入的偏导数$\left(\partial U/\partial n_i\right)_{S,V,\left\{n_{j\neq i}\right\}}$则定义为组份$i$在混合物中的化学势(chemical potential),记为
\[\mu_i\eqdef\left.\frac{\partial U}{\partial n_i}\right|_{S,V,\left\{n_{j\neq i}\right\}},\]

对焓、亥姆霍兹自由能和吉布斯自由能的定义式(仍假定体系只做体积功)
\[H\eqdef U+pV,\quad A\eqdef U-TS,\quad G\eqdef U+pV-TS\]
作微分,可得出这些热力学函数作为特性函数(characteristic functions)\footnote{见《物理化学》\S 3.13。}的微分式\footnote{在处理开放系统时,还会用到巨热力学势$J\eqdef U-TS-n\mu$,这里不介绍了。}
\begin{align}
    \mathrm{d}H & = T\mathrm{d}S+V\mathrm{d}p+\sum_i\mu_i\mathrm{d}n_i\label{eq:I.1_dH} \\
    \mathrm{d}A & =-S\mathrm{d}T-p\mathrm{d}V+\sum_i\mu_i\mathrm{d}n_i\label{eq:I.1_dA} \\
    \mathrm{d}G & =-S\mathrm{d}T+V\mathrm{d}p+\sum_i\mu_i\mathrm{d}n_i\label{eq:I.1_dG}
\end{align}
例如,对吉布斯自由能定义式作全微分,就有
\[\mathrm{d}G=\mathrm{d}U+p\mathrm{d}V+V\mathrm{d}p-T\mathrm{d}S-S\mathrm{d}T\]
把式\eqref{eq:I.1_first_law_multicomp}代入上式就能得到相应的微分式。

由于特性函数的用处,体系在两个我们关心的状态之间变化的可逆性和方向性就可以判断了。例如体系状态由$\left(X,Y,\left\{n_i\right\}\right)$独立、完整地确定,且对照特性函数与特征变量的对应关系应选用$M=M\left(X,Y,\left\{n_i\right\}\right)$,则两个状态——$\left(X_1,Y_1,\left\{n_{i,1}\right\}\right)$到$\left(X_2,Y_2,\left\{n_{i,2}\right\}\right)$——的$M$值变化不依赖路径(因$M$是状态函数),故总可找到一条可逆路径,将$M$的微分式$\mathrm{d}M=X\mathrm{d}Y+Y\mathrm{d}X+\sum_i\mu_i\mathrm{d}n_i$用于以下积分
\begin{equation}
    \begin{aligned}\label{eq:I.1_integral_of_function}
        \Delta M & =M\left(X_2,Y_2,\left\{n_{i,2}\right\}\right)-M\left(X_1,Y_1,\left\{n_{i,1}\right\}\right)                                      \\
                 & =\int_{\left(X_1,Y_1,\left\{n_{i,1}\right\}\right)\rightarrow_\text{rev}\left(X_2,Y_2,\left\{n_{i,2}\right\}\right)}\mathrm{d}M \\
                 & =\int_{Y_1}^{Y_2}X\mathrm{d}Y+\int_{X_1}^{X_2}Y\mathrm{d}X+\sum_i\int_{n_{i,1}}^{n_{i,2}}\mu_i\mathrm{d}n_i
    \end{aligned}
\end{equation}
知道$\Delta M$还有许多其他应用,在《物理化学》课的学习中我们已有深刻体会。而$\Delta M$全是靠$dM$在两个状态之间的积分算出来。所以在热力学中我们主要的篇幅都是在讨论$\mathrm{d}M$的表达式和它们之间的关系。

式\eqref{eq:I.1_dH}至\eqref{eq:I.1_dG}与它们的全微分式比较可得
\begin{equation}
    \mu_i=\left.\frac{\partial U}{\partial n_i}\right|_{S,V,\left\{n_{j\neq i}\right\}}
    =\left.\frac{\partial H}{\partial n_i}\right|_{S,p,\left\{n_{j\neq i}\right\}}
    =\left.\frac{\partial A}{\partial n_i}\right|_{T,V,\left\{n_{j\neq i}\right\}}
    =\left.\frac{\partial G}{\partial n_i}\right|_{T,p,\left\{n_{j\neq i}\right\}}\label{eq:I.1_first_order_partial_mu}
\end{equation}
以及
\begin{align}
    T & =\left.\frac{\partial U}{\partial S}\right|_{V,\left\{n_i\right\}}=\left.\frac{\partial H}{\partial S}\right|_{V,\left\{n_i\right\}},\label{eq:I.1_first_order_partial_T}   \\
    p & =-\left.\frac{\partial U}{\partial V}\right|_{S,\left\{n_i\right\}}=-\left.\frac{\partial A}{\partial V}\right|_{T,\left\{n_i\right\}},\label{eq:I.1_first_order_partial_p} \\
    V & =\left.\frac{\partial H}{\partial p}\right|_{S,\left\{n_i\right\}}=\left.\frac{\partial G}{\partial p}\right|_{T,\left\{n_i\right\}},\label{eq:I.1_first_order_partial_V}   \\
    S & =-\left.\frac{\partial A}{\partial T}\right|_{V,\left\{n_i\right\}}=-\left.\frac{\partial G}{\partial T}\right|_{p,\left\{n_i\right\}}\label{eq:I.1_first_order_partial_S}
\end{align}

$U$、$S$、$H$、$A$、$G$等热力学函数(及其偏导数)是无法直接测量的。我们能直接测量的是体系的状态参数——即独立、完整确定体系状态的那些量,及其关系——即状态方程。此外我们还能通过量热手段测量体系的热容。因此,我们需要把热力学函数的微分式中的那些偏导数努力地表示成我们能直接测量的量。对于物质的量恒定,只做体积功的单组份体系,这些可直接测量的量包括:

两种可逆过程热容:定容热容
\[C_V\eqdef\mathrm{d} Q_\text{可逆等容变温}/\mathrm{d}T\]
和定压热容
\[C_p\eqdef\mathrm{d} Q_\text{可逆等压变温}/\mathrm{d}T\]
它们常直接重新定义为
\[C_V\eqdef\left.\frac{\partial U}{\partial T}\right|_{V,\left\{n_i\right\}},\quad C_p\eqdef\left.\frac{\partial H}{\partial T}\right|_{p,\left\{n_i\right\}}\]
但更有用的是由可逆过程熵变的热温商式得到的
\begin{align}
    C_p & =T\left.\frac{\partial S}{\partial T}\right|_{p,\left\{n_i\right\}}\label{eq:I.1_heat_capacity_entropy_p} \\
    C_V & =T\left.\frac{\partial S}{\partial T}\right|_{V,\left\{n_i\right\}}\label{eq:I.1_heat_capacity_entropy_V}
\end{align}

体系状态参数之间的偏导数,可称之为$pVT$响应函数。它们之所以能有相互的偏导数是因为它们数学上由体系的状态方程所联系。组份不变时,有——

等压热膨胀系数:$\alpha_p\eqdef V^{-1}\left.\frac{\partial V}{\partial T}\right|_{p,\left\{n_i\right\}}$

等温压缩系数:$\kappa_T\eqdef-V^{-1}\left.\frac{\partial V}{\partial p}\right|_{T,\left\{n_i\right\}}$

等容压强系数:$\beta_V\eqdef\left.\frac{\partial p}{\partial T}\right|_{V,\left\{n_i\right\}}$

由于$p$、$V$、$T$之间是相关联的,这三个偏导数之间也是相关联的:$\alpha_p=-\beta_V\kappa_T$\footnote{见《物理化学》附录I.3式(2)。}。

上列$pVT$响应函数是适用于等温实验的。还有另一系列等熵(isentropic)实验的对应参数\footnote{等熵过程就是可逆绝热过程。在多数实验中,我们一般是控制一定的温度。如果体系在定温下等到平衡态再记录测量数据,得到的就是等温响应函数;而如果是在条件突然变化的瞬间测得数据,则由于热来不及传导而近似绝热条件的响应,这样的实验结果常近似作为等熵参数来报道。这是等熵参数的实际意义。}——

等熵热膨胀系数:$\alpha_S\eqdef V^{-1}\left.\frac{\partial V}{\partial T}\right|_{S,\left\{n_i\right\}}$

等熵压缩系数:$\kappa_S\eqdef -V^{-1}\left.\frac{\partial V}{\partial p}\right|_{S,\left\{n_i\right\}}$

等熵压强系数:$\beta_S\eqdef\left.\frac{\partial p}{\partial T}\right|_{S,\left\{n_i\right\}}$

等到后面面介绍Maxwell关系时我们将会发现,等温系列和等熵系列参数之间可通过热力学基本定律和状态方程相互关联。

还有一些表征体系特性的系数也是热力学函数的偏导数——

焦汤系数:$\mu_\text{JT}\eqdef\left.\frac{\partial T}{\partial p}\right|_{H,\left\{n_i\right\}}$

热容比:$\gamma\eqdef\frac{C_p}{C_V}$

它们也是可以通过热力学定律和状态方程与上述的响应函数相关联的。特别地,
\begin{equation}
    \gamma=\frac{\kappa_T}{\kappa_S}
\end{equation}

多组份体系的各组份物质的量$\left\{n_i\right\}$也是状态参数,所以状态参数之间的偏导数还应包括:
\[\left.\frac{\partial p}{\partial n_i}\right|_{T,V,\left\{n_{j\neq i}\right\}},\quad\left.\frac{\partial V}{\partial n_i}\right|_{T,p,\left\{n_{j\neq i}\right\}},\quad\left.\frac{\partial T}{\partial n_i}\right|_{p,V,\left\{n_{j\neq i}\right\}}\]
其中第二个就是偏摩尔体积。实际上只需要测量偏摩尔体积即可。因为
\begin{align*}
    \left.\frac{\partial p}{\partial n_i}\right|_{T,V,\left\{n_{j\neq i}\right\}} & =\left.\frac{\partial p}{\partial V}\right|_{T,\left\{n_i\right\}}\left.\frac{\partial V}{\partial n_i}\right|_{T,p,\left\{n_{j\neq i}\right\}}=-\frac{1}{\kappa_TV}\left.\frac{\partial V}{\partial n_i}\right|_{T,p,\left\{n_{j\neq i}\right\}}, \\
    \left.\frac{\partial T}{\partial n_i}\right|_{p,V,\left\{n_{j\neq i}\right\}} & =\left.\frac{\partial T}{\partial V}\right|_{p,\left\{n_i\right\}}\left.\frac{\partial V}{\partial n_i}\right|_{T,p,\left\{n_{j\neq i}\right\}}=\frac{1}{\alpha_p V}\left.\frac{\partial V}{\partial n_i}\right|_{T,p,\left\{n_{j\neq i}\right\}}
\end{align*}

类似地,$p$、$V$关于$n_i$的等温偏导数也有另一套对应的等熵的版本,不再列出了。我们把混合物体系的这些等温或等熵性质叫做$pVTn_i$响应函数。

运用Tobolsky方法\cite{Tobolsky1942},可以把任意热力学函数偏导数表示成仅含上列可测量的形式,从而打通热力学理论和实验应用的道路。这个方法需要使用Maxwell关系\footnote{参见《物理化学》\S 3.13。}。具体地,对式\eqref{eq:I.1_first_order_partial_mu}至式\eqref{eq:I.1_first_order_partial_S}中的偏导数再作不同变量的交叉二阶导数,并由于这些状态函数都假定连续可微而可交换偏导数顺序,可以得到一系列Maxwell关系式。其中将会出现很多$pVTn_i$响应函数。由内能的交叉偏导数得到:
\begin{align}
    \left.\frac{\partial^2 U}{\partial S\partial V}\right|_{\left\{n_i\right\}}             & =\left.\frac{\partial T}{\partial V}\right|_{S,\left\{n_i\right\}}=\left(\alpha_S V\right)^{-1}=-\left.\frac{\partial p}{\partial S}\right|_{V,\left\{n_i\right\}}\label{eq:I.1_Maxwell_USV} \\
    \left.\frac{\partial ^2U}{\partial n_i\partial S}\right|_{V,\left\{n_{j\neq i}\right\}} & =\left.\frac{\partial \mu_i}{\partial S}\right|_{V,\left\{n_i\right\}}=\left.\frac{\partial T}{\partial n_i}\right|_{S,V,\left\{n_{j\neq i}\right\}}\label{eq:I.1_Maxwell_UnS}               \\
    \left.\frac{\partial ^2U}{\partial n_i\partial V}\right|_{S,\left\{n_{j\neq i}\right\}} & =\left.\frac{\partial \mu_i}{\partial V}\right|_{S,\left\{n_i\right\}}=-\left.\frac{\partial p}{\partial n_i}\right|_{S,V,\left\{n_{j\neq i}\right\}}\label{eq:I.1_Maxwell_UnV}
\end{align}
由焓的交叉偏导数得到
\begin{align}
    \left.\frac{\partial^2 H}{\partial S\partial p}\right|_{\left\{n_i\right\}}             & =\left.\frac{\partial T}{\partial p}\right|_{S,\left\{n_i\right\}}=\beta_S^{-1}=\left.\frac{\partial V}{\partial S}\right|_{p,\left\{n_i\right\}}\label{eq:I.1_Maxwell_HSp}    \\
    \left.\frac{\partial ^2H}{\partial n_i\partial S}\right|_{p,\left\{n_{j\neq i}\right\}} & =\left.\frac{\partial \mu_i}{\partial S}\right|_{p,\left\{n_i\right\}}=\left.\frac{\partial T}{\partial n_i}\right|_{S,p,\left\{n_{j\neq i}\right\}}\label{eq:I.1_Maxwell_HnS} \\
    \left.\frac{\partial ^2H}{\partial n_i\partial p}\right|_{S,\left\{n_{j\neq i}\right\}} & =\left.\frac{\partial \mu_i}{\partial p}\right|_{S,\left\{n_i\right\}}=\left.\frac{\partial V}{\partial n_i}\right|_{S,p,\left\{n_{j\neq i}\right\}}\label{eq:I.1_Maxwell_Hnp}
\end{align}
由亥姆霍兹自由能的交叉偏导数得到:
\begin{align}
    \left.\frac{\partial^2 A}{\partial T\partial V}\right|_{\left\{n_i\right\}}             & =\left.\frac{\partial S}{\partial V}\right|_{T,\left\{n_i\right\}}=\left.\frac{\partial p}{\partial T}\right|_{V,\left\{n_i\right\}}=\beta_V^{-1}\label{eq:I.1_Maxwell_ATV}     \\
    \left.\frac{\partial ^2A}{\partial n_i\partial T}\right|_{V,\left\{n_{j\neq i}\right\}} & =\left.\frac{\partial \mu_i}{\partial T}\right|_{V,\left\{n_i\right\}}=-\left.\frac{\partial S}{\partial n_i}\right|_{T,V,\left\{n_{j\neq i}\right\}}\label{eq:I.1_Maxwell_AnT} \\
    \left.\frac{\partial ^2A}{\partial n_i\partial V}\right|_{T,\left\{n_{j\neq i}\right\}} & =\left.\frac{\partial \mu_i}{\partial V}\right|_{T,\left\{n_i\right\}}=-\left.\frac{\partial p}{\partial n_i}\right|_{T,V,\left\{n_{j\neq i}\right\}}\label{eq:I.1_Maxwell_AnV}
\end{align}
由吉布斯自由能的交叉偏导数得到:
\begin{align}
    \left.\frac{\partial^2 G}{\partial T\partial p}\right|_{\left\{n_i\right\}}             & =\left.\frac{\partial S}{\partial p}\right|_{T,\left\{n_i\right\}}=-\left.\frac{\partial V}{\partial T}\right|_{p,\left\{n_i\right\}}=-\alpha_p V\label{eq:I.1_Maxwell_GTp}     \\
    \left.\frac{\partial ^2G}{\partial n_i\partial T}\right|_{p,\left\{n_{j\neq i}\right\}} & =\left.\frac{\partial \mu_i}{\partial T}\right|_{p,\left\{n_i\right\}}=-\left.\frac{\partial S}{\partial n_i}\right|_{T,p,\left\{n_{j\neq i}\right\}}\label{eq:I.1_Maxwell_GnT} \\
    \left.\frac{\partial ^2G}{\partial n_i\partial p}\right|_{T,\left\{n_{j\neq i}\right\}} & =\left.\frac{\partial \mu_i}{\partial p}\right|_{T,\left\{n_i\right\}}=\left.\frac{\partial V}{\partial n_i}\right|_{T,p,\left\{n_{j\neq i}\right\}}\label{eq:I.1_Maxwell_GnV}
\end{align}

最后,从上述Maxwell关系可归纳出,除了$C_p$、$C_V$、各$pVTn_i$(等温或等熵)响应函数,还有两种必须知道的量我们还没讨论,那就是等压或等容偏摩尔熵(式\eqref{eq:I.1_Maxwell_AnT}和\eqref{eq:I.1_Maxwell_GnT})。具体地,它们分别是含在视熵为$\left(T,V,\left\{n_i\right\}\right)$或$\left(T,p,\left\{n_i\right\}\right)$的函数的全微分中的:
\begin{equation}
    \begin{aligned}
        \mathrm{d}S & =\left.\frac{\partial S}{\partial T}\right|_{V,\left\{n_i\right\}}\mathrm{d}T+\left.\frac{\partial S}{\partial V}\right|_{T,\left\{n_i\right\}}\mathrm{d}V+\sum_i\left.\frac{\partial S}{\partial n_i}\right|_{T,V,\left\{n_{j\neq i}\right\}} \\
                    & =\frac{C_V}{T}\mathrm{d}T+\beta_V^{-1}\mathrm{d}V+\sum_i\left.\frac{\partial S}{\partial n_i}\right|_{T,V,\left\{n_{j\neq i}\right\}} \label{eq:I.1_dS_T_V}
    \end{aligned}
\end{equation}
\begin{equation}
    \begin{aligned}
        \mathrm{d}S & =\left.\frac{\partial S}{\partial T}\right|_{p,\left\{n_i\right\}}\mathrm{d}T+\left.\frac{\partial S}{\partial p}\right|_{T,\left\{n_i\right\}}\mathrm{d}p+\sum_i\left.\frac{\partial S}{\partial n_i}\right|_{T,p,\left\{n_{j\neq i}\right\}} \\
                    & =\frac{C_p}{T}\mathrm{d}T-\alpha_pV\mathrm{d}p+\sum_i\left.\frac{\partial S}{\partial n_i}\right|_{T,p,\left\{n_{j\neq i}\right\}}\label{eq:I.1_dS_T_p}
    \end{aligned}
\end{equation}
其中用到了式\eqref{eq:I.1_heat_capacity_entropy_p}和\eqref{eq:I.1_heat_capacity_entropy_V}。这些偏摩尔熵的实验测量,跟熵本身一样,最终是落实到相应的偏摩尔热容
\begin{align}
    C_{p,i} & \eqdef\left.\frac{\partial C_p}{\partial n_i}\right|_{T,p\left\{n_{j\neq i}\right\}} \\
    C_{V,i} & \eqdef\left.\frac{\partial C_V}{\partial n_i}\right|_{T,V\left\{n_{j\neq i}\right\}}
\end{align}
和偏摩尔$pVT$等温响应函数
\begin{align}
    \alpha_{p,i} & \eqdef\left.\frac{\partial \alpha_p}{\partial n_i}\right|_{T,p,\left\{n_{j\neq i}\right\}} \\
    \kappa_{T,i} & \eqdef\left.\frac{\partial \kappa_T}{\partial n_i}\right|_{T,p,\left\{n_{j\neq i}\right\}} \\
    \beta_{V,i}  & \eqdef\left.\frac{\partial \beta_V}{\partial n_i}\right|_{T,p,\left\{n_{j\neq i}\right\}}
\end{align}
的测量上。而偏摩尔$pVT$等温响应函数又可由$pVTn_i$响应函数得到,所以真正需要额外测量的就是偏摩尔热容,这可以用一系列不同组成的试样在同温同压下的热容数据得到\footnote{关于偏摩尔量,还有很多重要的热力学关系,将在后续章节专门介绍。}。总而言之,想要完整确定体系的热力学性质,需要测定体系的热容和$pVT$响应函数;若是混合物体系,还需测定偏摩尔热容和偏摩尔体积。

有了这些关系,就总是能从已知体系的状态方程出发(无论是来自实验测量或者理论模型),把式\eqref{eq:I.1_first_law_multicomp}至\eqref{eq:I.1_dG}表示成仅含可测量的形式,再由式\eqref{eq:I.1_integral_of_function}得到体系的任何平衡态热力学行为,实现“想算什么就算什么”的自由。

例如,我们随便要求算一个古怪的偏导数:$\left.\frac{\partial H}{\partial A}\right|_{S}$。从这个偏导数形式上看,它来自由$\left(A,S\right)$所独立而完整确定的形式$H=H\left(A,S\right)$,是单组分体系。故令$X\equiv\left.\frac{\partial H}{\partial F}\right|_{S}$、$Y\equiv\left.\frac{\partial H}{\partial S}\right|_{A}$,则$H$的全微分可表示成
\[\mathrm{d}H=X\mathrm{d}A+Y\mathrm{d}S\]
但是$H$自己有作为特性函数的微分式\eqref{eq:I.1_dH}(单组分体系$\mathrm{d}n_i=0$),故可联系而得到以下式子:
\[X\mathrm{d}A+Y\mathrm{d}S=T\mathrm{d}S+V\mathrm{d}p\]
然后,我们需要确定,我们的实验是在什么特性参数条件下做的。例如,我们的实验是恒温恒压下做的,那么我们就需要把上列的微分式中的$\mathrm{d}A$、$\mathrm{d}S$换成$\mathrm{d}T$和$\mathrm{d}p$。这需要恰当选用相应的式子。比如,如果我们用式\eqref{eq:I.1_dA}把$\mathrm{d}A$换掉,就会新增我们所不需要的一个$\mathrm{d}V$项。这时只需再通过状态方程按$V=V\left(T,p\right)$,可以把$\mathrm{d}V=\alpha_pV\mathrm{d}T-\kappa_TV\mathrm{d}p$再代进去,就得到只含$\mathrm{d}T$和$\mathrm{d}p$的项了。类似地$\mathrm{d}S$用式\eqref{eq:I.1_dS_T_p}代入,最终可得到:
\[\left(-SX-pX\alpha_pV+\frac{YC_p}{T}-C_p\right)\mathrm{d}T+\left(pX\kappa_TV-Y\alpha_pV+T\alpha_pV-V\right)\mathrm{d}p=0\]
由于上式是热力学关系推出来的,总成立,故有
\begin{align*}
    -SX-pX\alpha_pV+\frac{YC_p}{T}-C_p  & =0 \\
    pX\kappa_TV-Y\alpha_pV+T\alpha_pV-V & =0
\end{align*}
解得
\begin{align*}
    X & =\frac{C_p}{C_p p \kappa_T-T\alpha_p\left(S+pV\alpha_p\right)}                          \\
    Y & =T\left(1-\frac{S+p V \alpha_p}{T\alpha_p\left(S+pV\alpha_p\right)-C_pp\kappa_T}\right)
\end{align*}
其中$X$是我们想要的。我们发现,这些表达式中除了含有之前说到的各种可测量响应函数之外,还含有熵值$S$。这也是不用担心的。将来在最后应用式\eqref{eq:I.1_integral_of_function}时,可转化成同状态间的熵变$\Delta S$,又可通过式\eqref{eq:I.1_dS_T_p}由积分\eqref{eq:I.1_integral_of_function}得到。此例说明,我们总是能用我们方便实验的条件(恒温恒压)测量的结果,去计算任意一个也许在特定理论分析中碰到的,又很难实验直接测量的偏导数(“恒熵下含随亥姆霍兹自由能的变化量”)。因此,热力学在实操层面上的重点在于体系的热容和状态方程的确定(多组份情况下还包括必要的偏摩尔热容和偏摩尔体积)。
\end{document}