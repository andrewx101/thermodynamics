\documentclass[main.tex]{subfiles}
\begin{document}
在《物理化学》课本当中,真实体系的性质是直接与另行举出的理想气体“比较”而定义的,二者之间的热力学过程联系并不清楚,因而会造成误解。

经典条件下,我们认为任一处于凝聚态的真实体系在压强$p$小到一定程度时都会是气态,且在$p=0$邻近接近理想气体。因此我们常对真实体系取$p\rightarrow 0$极限,以便为同一个(真实)体系构造出“从理想气体($p=0$)变成当前状态”的过程。这样才能利用热力学基本关系,严格地把真实体系的性质写成对理想气体的偏离形式。当然,如果所讨论的(假想的)体系的气态就是理想气体,那在它处于气态的压强范围内的任一压强下都有理想气体的性质,而不必限制在$p=0$邻近。

由于上述观念是“我们认为”的,它是热力学理论中的规定。这种规定当然是来自经验总结的,旨在使所建立的理论与经验相符,而非相悖。皆因它除了来自经验总结之外,无法再由更基本的原则推出(在不借助统计力学的情况下),才只好在热力学理论中作为公设“规定”而提出。本节我们将介绍若干用于描述真实体系的概念和相应的物理量。它们的定义都包括在理想气体极限下趋于理想气体值的要求。

\subsection{气体混合物的分压}
气体的压强$p$是容器壁单位面积受到的法向分力。气体混合物中,某组份$i$的分压(partial pressure)是
\[p_i\equiv x_i^\text{(g)}p\]
其中$x_i^\text{(g)}$表示组份$i$在气相中的摩尔分数。对于理想气体混合物,组份$i$的分压是同样摩尔数的纯组份$i$在相同体积下和其余条件下的压强。但对于真实气体混合物,$x_i^\text{(g)}p$没有如此直接的物理意义,因此我们以下不写“$p_i$”而写“$x_i^\text{(g)}p$
”,以表示这个量只是个乘积;对于真实气体,我们只是拿实验可知量$x_i$和$p$相乘得出一个被标为$p_i$的量。但这种分压的定义保证了,无论是理想还是真实气体混合物,都有“分压定律”$p=\sum_i p_i$。

\subsection{气体的压缩因子}
气体混合物的压缩因子(compressibility factor)由下式定义
\[pV_\text{m}=ZRT\]
且
\[\lim_{p\to 0}Z\left(p;T,\left\{n_i\right\}\right)\equiv 1\]
其中$R$是气体常数。$V_m$是气体混合物的摩尔体积。自此,字母$Z$有了特定的物理意义(不同于上节)。由定义可知,压缩因子表示的是,给定温度$T$和压强$p$下,气体的摩尔体积与理想气体值之比。压缩因子是一个摩尔量,但按照《物理化学》的符号惯例,我们没有用$Z_\text{m}$表示。与压缩因子相对应的偏摩尔量应由下式给出:
\[z_i\equiv\left.\frac{\partial\left(n Z\right)}{\partial n_i}\right|_{T,p,\left\{n_{j\neq i}\right\}}=z_i\left(T,p,\left\{n_i\right\}\right)\]
称为组份$i$在气体混合物中的偏摩尔压缩因子。由偏摩尔量的加和性有
\[Z=\sum_i x_i^\text{(g)}z_i\]
注意,压缩因子是仅针对气态物质而提出的概念。

$\lim_{p\to 0}Z=1$的规定是来自经验总结的,可参考《物理化学》图1.21。

压缩因子定义式中含摩尔体积,我们进一步推导偏摩尔压缩因子与偏摩尔体积的关系:
\begin{align*}
    v_i^\text{(g)} & =\left.\frac{\partial V}{\partial n_i}\right|_{T,p,\left\{n_{j\neq i}\right\}}                                                                                               \\
                   & =\left.\frac{\partial\left(n V_\text{m}\right)}{\partial n_i}\right|_{T,p,\left\{n_{j\neq i}\right\}}                                                                        \\
                   & =\left.\frac{\partial n}{\partial n_i}\right|_{\left\{n_{j\neq i}\right\}}V_\text{m}+n\left.\frac{\partial V_\text{m}}{\partial n_i}\right|_{T,p,\left\{n_{j\neq i}\right\}} \\
                   & =\frac{ZRT}{p}+\frac{nRT}{p}\left.\frac{\partial Z}{\partial n_i}\right|_{T,p,\left\{n_{j\neq i}\right\}}                                                                    \\
                   & =\frac{RT}{p}\left.\frac{\partial\left(nZ\right)}{\partial n_i}\right|_{T,p,\left\{n_{j\neq i}\right\}}=\frac{RTz_i}{p}
\end{align*}
即对气体混合物的任一组份$i$,有
\[p v_i^\text{(g)}=z_iRT\]

\subsection{气体的残余体积}
气体混合物的残余体积(residual volume)是给定温度和压强下,实际气体的体积与理想气体值之差,具体定义为
\[\alpha\equiv \frac{RT}{p}-V_\text{m}=\alpha\left(T,p,\left\{n_i\right\}\right)\]
可见,$\alpha$是一个摩尔量。与其相应的偏摩尔量(称偏摩尔残余体积)应是
\[\alpha_i\equiv\left.\frac{\partial\left(n\alpha\right)}{\partial n_i}\right|_{T,p,\left\{n_{j\neq i}\right\}}=\alpha\left(T,p,\left\{n_i\right\}\right)\]
从而有
\[\alpha=\sum_i x_i^\text{(g)}\alpha_i\]
与气体的压缩因子类似,气体的残余体积是仅针对气态体系而提出的概念。由定义易得$\alpha$与$Z$的关系
\[\alpha=\frac{RT}{p}\left(1-Z\right)\]

在理想气体极限,气体的残余体积不为零。具体地
\begin{align*}
    \lim_{p\to 0}\alpha & =RT\lim_{p\to 0}\frac{1-Z}{p}                                                                             \\
                        & =-RT\left.\frac{\partial Z}{\partial p}\right|_{T,p\to 0,\left\{n_i\right\}} & \text{(} & \text{利用洛比达法则)}
\end{align*}
也就是说,在理想气体极限下,气体的残余体积是压缩因子在$p\to 0$的渐近斜率,一般不为零。

\subsection{逸度}
无论气态还是液态混合物,我们定义混合物体系的平均逸度(简称逸度)$f$和组份$i$在混合物中的逸度$f_i$。混合物体系的逸度$f$满足
\[\mathrm{d}G_\text{m}=RT\mathrm{d}\ln f,\quad\text{恒定$T$}\]
且
\[\lim_{p^\prime\to 0}\frac{f\left(p^\prime;T,\left\{n_i\right\}\right)}{p^\prime}\equiv 1\]
组份$i$在混合物中的逸度$f_i$满足
\[\mathrm{d}\mu_i=RT\mathrm{d}\ln f_i,\quad\text{恒定$T$}\]
且
\[\lim_{p^\prime\to 0}\frac{f_i\left(p^\prime;T,\left\{n_i\right\}\right)}{p^\prime}\equiv 1\]

这两个逸度定义的第一个微分式是恒定$T$下的微变化,所以不是全微分式。在所给定的温度$T$下,$f$或$f_i$是压强和组成的函数。

逸度定义的第二个含有极限的条件,其实是要求如此定义的逸度在理想气体极限($p\rightarrow 0$)附近等于压强$p$。这是基于大量经验的总结。严格地,该极限表达式说,$f_i^*$是$p$的同阶无穷小。在给定温度$T$下,$f_i$关于$p$的函数形式可写成展开式:$f_i^*\left(p;T,n_i\right)=p+o\left(p^2\right)$。所以,虽然当$p=0$时$f_i^*=0$,但当$p$足够接近零时,$f_i^*\approx p$,且$p$越接近零$f_i^*$越接近$p$。

组份$i$在混合物中的逸度定义的第二个含有极限的条件,是想说,在压强极小时,组份$i$的逸度接近$x_i p$,亦即气体接近遵守道尔顿分压定律。这是缺乏实验基础的,因此这一规定甚至不能说是来自经验的总结,而是纯粹为了使理论自洽而而构造出来的。因为,道尔顿分压定律是理想气体的性质,如果认为$p\rightarrow 0$时纯物质气体接近理想气体,则必须同时认为$p\rightarrow 0$时气体混合物接近遵循道尔顿分压定律。

由于这种基于微分关系的定义方式,想要显定地写下其定义表达式,就需要求定积分,需要选定一个状态作为参考初态。一个自然的选择就是理想气体极限($p\rightarrow 0$)。由Gibbs--Duhem方程,在恒定$T$下,有
\[\sum_i x_i\mathrm{d}\mu_i=V_m\mathrm{d}p\]
用等号左边代入逸度定义,等号右边用压缩因子表示(即仅考虑体系呈气态压强$p$范围),上式变为
\[\sum_i x_i\mathrm{d}\ln f_i=RTZ\mathrm{d}\ln p\]
我们考虑从$p\rightarrow 0$极限到当前压强$p$的等温、恒定组成的变化过程,即对上式进行如下广义积分
\begin{align*}
    \lim_{p^\prime\to 0}\sum_ix_i\ln\frac{f\left(p;T,\left\{n_i\right\}\right)}{f\left(p^\prime;T,\left\{n_i\right\}\right)} & =\lim_{p^\prime\to 0}\sum_ix_i\ln\frac{f\left(p;T,\left\{n_i\right\}\right)}{p^\prime} & \text{(} & \text{用到了逸度定义的极限规定)} \\
                                                                                                                             & =\int_0^p Z\left(p^\prime;T,\left\{n_i\right\}\right)\mathrm{d}\ln p^\prime            &          &
\end{align*}
若所考虑的体系是一个气态总是理想气体的假想体系,即$Z\equiv 1$,则上式变为
\begin{align*}
                    & \lim_{p^\prime\to 0}\left[\sum_i x_i\ln\frac{f\left(p;T,\left\{n_i\right\}\right)}{p^\prime}-\ln\frac{p}{p^\prime}\right]=0 \\
    \Leftrightarrow & \lim_{p^\prime\to 0}\left[\sum_ix_i\ln f\left(p;T,\left\{n_i\right\}\right)-\ln p\right]=0                                  \\
    \Leftrightarrow & f\left(p\right)=p,\quad\text{理想气体}
\end{align*}
——正如其所应当的那样。

我们可以继续证明,组份$i$在混合物中的逸度$f_i$,在假想理想气体的情况中就等于$x_i p$,即满足道尔顿分压定律。但在这么做之前,先要认识清楚,$f_i$不是与$nf$对应的偏摩尔量(这是易知的)。实际上,$\ln\left(f_i/x_i\right)$是对应于$n\ln f$的偏摩尔量。下面是推导过程。考虑从$p\rightarrow 0$极限到当前压强$p$的等温过程,由$f$的定义式,
\begin{align*}
    \lim_{p^\prime\to 0}\left[G_\text{m}\left(p;T,\left\{n_i\right\}\right)-G_\text{m}\left(p^\prime;T,\left\{n_i\right\}\right)\right] & =RT\lim_{p^\prime\to 0}\left[\ln\frac{f\left(p;T,\left\{n_i\right\}\right)}{f\left(p^\prime;T,\left\{n_i\right\}\right)}\right] \\
                                                                                                                                        & =RT\lim_{p^\prime\to 0}\ln\frac{f\left(p;T,\left\{n_i\right\}\right)}{p^\prime}
\end{align*}
其中用到了逸度定义的极限规定。乘以$n$,并对$n_i$求偏导数:
\begin{align*}
      & \left.\frac{\partial}{\partial n_i}\lim_{p^\prime\to 0}\left[G\left(p;T,\left\{n_k\right\}\right)-G\left(p^\prime;T,\left\{n_k\right\}\right)\right]\right|_{T,p,\left\{n_{j\neq i}\right\}} \\
    = & RT\left.\frac{\partial}{\partial n_i}\lim_{p^\prime\to 0}\left[n\ln f\left(p;T,\left\{n_k\right\}\right)-n\ln p^\prime\right]\right|_{T,p,\left\{n_{j\neq i}\right\}}
\end{align*}
这时,我们需要假定,热力学函数对组份的偏导函数具有一致收敛性,这样的话,我们才能交换求$p^\prime\rightarrow 0$极限与求$\partial/\partial n_i$偏导的顺序,得到
\begin{align*}
      & \lim_{p^\prime\to 0}\left[\mu_i\left(p;T,\left\{n_k\right\}\right)-\mu_i\left(p^\prime;T,\left\{n_k\right\}\right)\right]                                                                                                       \\
    = & RT\lim_{p^\prime\to 0}\left[\left.\frac{\partial\left(n\ln f\left(p;T,\left\{n_k\right\}\right)\right)}{\partial n_i}\right|_{T,p,\left\{n_{j\neq i}\right\}}-\frac{\mathrm{d}}{\mathrm{d}n_i}\left(n\ln p^\prime\right)\right]
\end{align*}
上式等号左边可用组份$i$在混合物中的逸度$f_i$的定义表示出来,即
\begin{align*}
    \lim_{p^\prime\to 0}\left[\mu_i\left(p;T,\left\{n_k\right\}\right)-\mu_i\left(p^\prime;T,\left\{n_k\right\}\right)\right] & =\int_0^p\mathrm{d}\mu_i                                                                                               \\
                                                                                                                              & =RT\lim_{p^\prime\to 0}\ln\frac{f_i\left(p;T,\left\{n_k\right\}\right)}{f_i\left(p^\prime;T,\left\{n_k\right\}\right)} \\
                                                                                                                              & =RT\lim_{p^\prime\to 0}\ln\frac{f_i\left(p;T,\left\{n_k\right\}\right)}{x_ip^\prime}
\end{align*}
其中利用到组份$i$在混合物中的逸度定义的极限规定。因此有
\begin{align*}
                    & \lim_{p^\prime\to 0}\ln\frac{f_i\left(p;T,\left\{n_k\right\}\right)}{x_ip^\prime}=\lim_{p^\prime\to 0}\left[\left.\frac{\partial\left(n\ln f\left(p;T,\left\{n_k\right\}\right)\right)}{\partial n_i}\right|_{T,p,\left\{n_{j\neq i}\right\}}-\ln p^\prime\right] \\
    \Leftrightarrow & \lim_{p^\prime\to 0}\left[\ln\frac{f_i\left(p;T,\left\{n_k\right\}\right)}{x_i}-\left.\frac{\partial}{\partial n_i}\left(n\ln f\left(p;T,\left\{n_k\right\}\right)\right)\right|_{T,p,\left\{n_{j\neq i}\right\}}\right]=0                                        \\
    \Leftrightarrow & \ln\frac{f_i\left(p;T,\left\{n_k\right\}\right)}{x_i}=\left.\frac{\partial}{\partial n_i}\left(n\ln f\left(p;T,\left\{n_k\right\}\right)\right)\right|_{T,p,\left\{n_{j\neq i}\right\}}
\end{align*}
可见,$\ln \left(f_i/x_i\right)$是对应于$n\ln f$的偏摩尔量。进一步,由偏摩尔量的加和性有
\[n\ln f=\sum_i n_i \ln\left(\frac{f_i}{x_i}\right)\Leftrightarrow\ln f=\sum_ix_i\ln\frac{f_i}{x_i}\]
其中为了简洁省去了函数自变量的表示。有了这一关系,由理想气体$f=p$可得出理想气体混合物中$f_i=x_ip$,即满足道尔顿分压定律——正如其所应当的那样。

考虑恒定$T$和$p$下,组份从纯物质$i$($x_i=1$)到组份$i$的摩尔分数为$x_i$的混合过程。这一过程的化学势变化
\begin{align*}
    \mu_i\left(x_i;T,p\right)-\mu_i^*\left(T,p\right) & =\int_1^{x_i}\mathrm{d}\mu_i                               \\
                                                      & =RT\ln\frac{f_i\left(x_i;T,p\right)}{fi^*\left(T,p\right)}
\end{align*}
因此有
\[\mu_i\left(x_i;T,p\right)=\mu_i^*\left(T,p\right)+RT\ln\frac{f_i\left(x_i;T,p\right)}{f_i^*\left(T,p\right)}\]
这是混合物的化学势的逸度表达式。在$p\rightarrow 0$极限下,
\begin{align*}
    \lim_{p\to 0}\left[\mu_i\left(x_i;T,p\right)-\mu_i^*\left(T,p\right)\right] & =RT\lim_{p\to 0}\ln\frac{f_i\left(x_i;T,p\right)}{f_i^*\left(T,p\right)}                                   \\
                                                                                & =RT\lim_{p\to 0}\left[\ln\frac{f_i\left(x_i;T,p\right)}{x_ip}-\ln\frac{f_i^*\left(T,p\right)}{x_ip}\right] \\
                                                                                & =RT\ln x_i,\quad p\rightarrow 0
\end{align*}
故在$p\rightarrow 0$极限,
\[\mu_i\left(x_i;T,p\right)\approx\mu_i^*\left(T,p\right)+RT\ln x_i,\quad p\rightarrow 0\]






\end{document}